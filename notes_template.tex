% !TEX root = ./notes_template.tex
%%%%%%%%%%%%%%%%%%%%%%%%%%%%%%%%%%%%%%%%%%%%%%%%%%
%%%%%%%%%%%%%%%%%%%%% preamble %%%%%%%%%%%%%%%%%%%
%%%%%%%%%%%%%%%%%%%%%%%%%%%%%%%%%%%%%%%%%%%%%%%%%%
\documentclass[11pt,twoside]{book}
\usepackage[mono=false]{libertine} % new linux font, ignore mono

\usepackage{luatex85}

%\renewcommand{\baselinestretch}{1.05}
\usepackage{amsmath,amsthm,amssymb,mathrsfs,amsfonts,dsfont}
\usepackage{epsfig,graphicx}
\usepackage{tabularx}
\usepackage{blkarray}
\usepackage{slashed}
\usepackage{scalerel}
\def\msquare{\mathord{\scalerel*{\Box}{gX}}}
\usepackage{color}
\usepackage{listings}
\usepackage{caption}
% \usepackage{fullpage}
\usepackage{lipsum} % provides dummy text for testing
\usepackage[toc,title,titletoc,header]{appendix}
\usepackage{minitoc}
\usepackage{color}
\usepackage{multicol} % two-col ToC
\usepackage{bm}
\usepackage{imakeidx} % before hyperref
\usepackage{hyperref}
\usepackage{mathtools}
\DeclarePairedDelimiter\ceil{\lceil}{\rceil}
\DeclarePairedDelimiter\floor{\lfloor}{\rfloor}

% link colors settings
\hypersetup{
    colorlinks=true,
    citecolor=magenta,
    linkcolor=blue,
    filecolor=green,      
    urlcolor=cyan,
    % hypertexnames=false,
}
\usepackage[capitalise]{cleveref}
\usepackage{subcaption}
\usepackage{enumitem}
\usepackage{mathtools}
\usepackage{pdfpages}
\usepackage{physics}
\usepackage[linesnumbered,ruled,vlined,algosection]{algorithm2e}
\SetCommentSty{textsf}
\usepackage{epigraph}
\epigraphwidth=1.0\linewidth
\epigraphrule=0pt

% adjust margin
\usepackage[margin=2.3cm]{geometry}
\headheight13.6pt

%%%%%%%%%%%%%%%% thmtools %%%%%%%%%%%%%%%%%%%%%
\usepackage{thmtools}
\declaretheorem[numberwithin=chapter]{theorem}
\declaretheorem[numberwithin=chapter]{axiom}
\declaretheorem[numberwithin=chapter]{lemma}
\declaretheorem[numberwithin=chapter]{proposition}
\declaretheorem[numberwithin=chapter]{claim}
\declaretheorem[numberwithin=chapter]{conjecture}
\declaretheorem[numberwithin=chapter]{assumption}
\declaretheorem[sibling=theorem]{corollary}
\declaretheorem[numberwithin=chapter, style=definition]{definition}
\declaretheorem[numberwithin=chapter, style=definition]{problem}
\declaretheorem[numberwithin=chapter, style=definition]{example}
\declaretheorem[numberwithin=chapter, style=definition]{exercise}
\declaretheorem[numberwithin=chapter, style=definition]{observation}
\declaretheorem[numberwithin=chapter, style=definition]{fact}
\declaretheorem[numberwithin=chapter, style=definition]{construction}
\declaretheorem[numberwithin=chapter, style=definition]{remark}
\declaretheorem[numberwithin=chapter, style=remark]{question}
%%%%%%%%%%%%%%%% thmtools %%%%%%%%%%%%%%%%%%%%%
\usepackage{changepage}
\newenvironment{solution}
    {\renewcommand\qedsymbol{$\square$}\color{blue}\begin{adjustwidth}{0em}{2em}\begin{proof}[\textit Solution.~]}
    {\end{proof}\end{adjustwidth}}

%%%%%%%%%%%%%%%% index %%%%%%%%%%%%%%%%%%%%%
\begin{filecontents}{index.ist}
% https://tex.stackexchange.com/questions/65247/index-with-an-initial-letter-of-the-group
headings_flag 1
heading_prefix "{\\centering\\large \\textbf{"
heading_suffix "}}\\nopagebreak\n"
delim_0 "\\nobreak\\dotfill"
\end{filecontents}
\newcommand{\myindex}[1]{\index{#1} \emph{#1}}
\makeindex[columns=3, intoc, title=Alphabetical Index, options= -s index.ist]
%%%%%%%%%%%%%%%% index %%%%%%%%%%%%%%%%%%%%%

%%%%%%%%%%%%%%%% ToC %%%%%%%%%%%%%%%%%%%%%
% Link Chapter title to ToC: https://tex.stackexchange.com/questions/32495/linking-the-section-text-to-the-toc
\usepackage[explicit]{titlesec}
\titleformat{\chapter}[display]
  {\normalfont\huge\bfseries}{\chaptertitlename\ {\thechapter}}{20pt}{\hyperlink{chap-\thechapter}{\Huge#1}
\addtocontents{toc}{\protect\hypertarget{chap-\thechapter}{}}}
\titleformat{name=\chapter,numberless}
  {\normalfont\huge\bfseries}{}{-20pt}{\Huge#1}

%%%%%%%%%%%%%%%%%%% fancyhdr %%%%%%%%%%%%%%%%%
\usepackage{fancyhdr}
\pagestyle{fancy} % enable fancy page style
\renewcommand{\headrulewidth}{0.0pt} % comment if you want the rule
\fancyhf{} % clear header and footer
\fancyhead[lo,le]{\leftmark}
\fancyhead[re,ro]{\rightmark}
\fancyfoot[CE,CO]{\hyperref[toc-contents]{\thepage}}

% https://tex.stackexchange.com/questions/550520/making-each-page-number-link-back-to-beginning-of-chapter-or-section
\makeatletter
\def\chaptermark#1{\markboth{\protect\hyper@linkstart{link}{\@currentHref}{Chapter \thechapter ~ #1}\protect\hyper@linkend}{}}
\def\sectionmark#1{\markright{\protect\hyper@linkstart{link}{\@currentHref}{\thesection ~ #1}\protect\hyper@linkend}}
\makeatother
%%%%%%%%%%%%%%%%%%% fancyhdr %%%%%%%%%%%%%%%%%


%%%%%%%%%%%%%%%%%%% biblatex %%%%%%%%%%%%%%%%%
\usepackage[doi=false,url=false,isbn=false,style=alphabetic,backend=biber,backref=true]{biblatex}
\addbibresource{bib.bib}

\newbibmacro{string+doiurlisbn}[1]{%
  \iffieldundef{doi}{%
    \iffieldundef{url}{%
      \iffieldundef{isbn}{%
        \iffieldundef{issn}{%
          #1%
        }{%
          \href{http://books.google.com/books?vid=ISSN\thefield{issn}}{#1}%
        }%
      }{%
        \href{http://books.google.com/books?vid=ISBN\thefield{isbn}}{#1}%
      }%
    }{%
      \href{\thefield{url}}{#1}%
    }%
  }{%
    \href{http://dx.doi.org/\thefield{doi}}{#1}%
  }%
}

% https://tex.stackexchange.com/questions/94089/remove-quotes-from-inbook-reference-title-with-biblatex
\DeclareFieldFormat[article,incollection,inproceedings,book,misc]{title}{\usebibmacro{string+doiurlisbn}{\mkbibemph{#1}}}
% https://tex.stackexchange.com/questions/454672/biblatex-journal-name-non-italic
\DeclareFieldFormat{journaltitle}{#1\isdot}
\DeclareFieldFormat{booktitle}{#1\isdot}
% https://tex.stackexchange.com/questions/10682/suppress-in-biblatex
\renewbibmacro{in:}{}
% add video field: https://tex.stackexchange.com/questions/111846/biblatex-2-custom-fields-only-one-is-working
\DeclareSourcemap{
    \maps[datatype=bibtex]{
      \map{
        \step[fieldsource=video]
        \step[fieldset=usera,origfieldval]
    }
  }
}
\DeclareFieldFormat{usera}{\href{#1}{\textsc{Online video}}}
\AtEveryBibitem{
    \csappto{blx@bbx@\thefield{entrytype}}{% put at end of entry
        \iffieldundef{usera}{}{\space \printfield{usera}}
    }
}
%%%%%%%%%%%%%%%%%%% biblatex %%%%%%%%%%%%%%%%%

%%%%%%%%%%%%%%%%%%%%% glossaries %%%%%%%%%%%%%%%%%
% !TEX root = ./notes_template.tex
% \usepackage[style=super]{glossaries}
% https://www.overleaf.com/learn/latex/Glossaries
\usepackage[style=super,toc,acronym]{glossaries}
\setlength{\glsdescwidth}{1\linewidth}
\makeglossaries

\renewcommand\glossaryname{List of Abbreviations and Symbols}

\newglossaryentry{Q2}{name={$Q_2(f)$},
%sort=Q2,
description={Two-side (bounded) error quantum query complexity}}

\newglossaryentry{real_number}{name={$\mathbb{R}$},description={Real number}}

% \newglossaryentry{gcd}{name={gcd},description={greatest common divisor}}

\newacronym{gcd}{GCD}{Greatest Common Divisor}


\newglossaryentry{svm}{name={SVM},description={Support Vector Machine}}

\newglossaryentry{gd}{name={GD},description={Gradient Descent}}

\newglossaryentry{qft}{name={QFT},description={Quantum Field Theory}}

\newglossaryentry{qm}{name={QM},description={Quantum Mechanics}}

\newglossaryentry{v}{name={$\vec{v}$},description={a vector}}

% physics
\newglossaryentry{hamiltonian}{name={$\hat{H}$},description={Hamiltonian}}

\newglossaryentry{lagrangian}{name={$L$},description={Lagrangian}}
%%%%%%%%%%%%%%%%%%%%% glossaries %%%%%%%%%%%%%%%%%

%%%%%%%%%%%%%%%%%%%%% glossaries-extra %%%%%%%%%%%%%%%%%
% \usepackage[record,abbreviations,symbols,stylemods={list,tree,mcols}]{glossaries-extra}
%%%%%%%%%%%%%%%%%%%%% glossaries-extra %%%%%%%%%%%%%%%%%


% !TEX root = ./notes_template.tex

%%%%%%%%%%%%%%%%%%%%%%%%%%%%%%%%%%%%
%%%%%%%%%%%%%%%%%%%%%%%%%%%%%%%%%%%%
% math
\let\iff\relax
\newcommand{\iff}{\text{ iff }}
\newcommand{\OPT}{\textup{OPT}}

% physics
\newcommand{\acreation}{a^\dagger}



\let\biconditional\leftrightarrow
%%%%%%%%%%%%%%%%%%%%%%%%%%%%%%%%%%%%%%%%%%%%%%%%%%
%%%%%%%%%%%%%%%% begin of document %%%%%%%%%%%%%%%
%%%%%%%%%%%%%%%%%%%%%%%%%%%%%%%%%%%%%%%%%%%%%%%%%%

\begin{document}

\title{\bf \huge PAPER NOTES}
\author{Xuyuan Zhang}
\date{Update on \today}
\maketitle
\setcounter{tocdepth}{2}
\setcounter{minitocdepth}{1} 

\begin{multicols}{2}
    \dominitoc% Initialization
    \adjustmtc[2]% chp number shift for mini-toc
    \tableofcontents
    \label{toc-contents}
\end{multicols}


	% \listoffigures
	% \listoftables
\begin{multicols}{2}
    \dominitoc% Initialization	
    \adjustmtc[2]% chp number shift for mini-toc
 \listoftheorems[ignoreall,show={theorem}]
\end{multicols}

\renewcommand{\listtheoremname}{List of Definitions}
\begin{multicols}{2}
	\listoftheorems[ignoreall,show={definition}]
\end{multicols}

	% \printglossaries
	% \printglossary[type=\acronymtype]
	\printglossary
	% \printglossary[title=List of terms, toctitle=List of terms]

	% bib2gls
	% \printunsrtglossaries % print all types
	% \printunsrtglossary[type={abbreviations},title=List of Abbreviations,style=listgroup]
	% \printunsrtglossary[type={abbreviations},title=List of Abbreviations,style=listhypergroup] % doesn't work
	% \printunsrtglossary[type={symbols},title=List of Symbols,style=listgroup]
	% \printunsrtglossary % main entry

%%%%%%%%%%%%%%%Content%%%%%%%%%%%%%%%
% \mainmatter % separat the number of toc and mainmatter
\part{urban economics}

\chapter{Scale Economics, Product Differentiation, and the Pattern of Trade}

\section{Model}

\subsection{Assumptions of the Model}

There are assumed to be a large number of potential goods, all of which enter systemtrically into demand. Specifically, we assume that all individuals in the economy have the same utility function:

\begin{equation}
    U = \sum_{i} c_i^{\theta}, \quad 0 < \theta < 1
\end{equation}

where $c_i$ is consumption of the $i$th good. The number of goods actually produced, $n$, will be assumed to be large, although smaller than the potential range of products.

There will be assumed to be only one factor of production, labor. All goods will be produced with the same cost function,

\begin{equation}
    l_i = \alpha + \beta x_i, \quad \alpha, \beta > 0, i = 1, \ldots, n
\end{equation}

where $l_i$ is labor used in producing the $i$th good, and $x_i$ is the output of that good. Average cost declines at all level of output, although at a diminishing rate.

Output of each good must equal the su of individual consumption. If we can identify individuals with workers, output must euqal consumption of a representative individualtimes the labor force:

\begin{equation}
    x_i = Lc_i, \quad i = 1, \ldots, n
\end{equation}

We assume full employment, so that the total labor force must just be exhausted by labor used in production:

\begin{equation}
    L = \sum_{i=1}^n (\alpha + \beta x_i)
\end{equation}

Finally, we assume that firms maximize profits, but that there is free entry and exit of firms, so that in equilibrium profits will always be zero.

\subsection{Equilibrium in a Closed Economy}

First, I analyze consumer behavior to derive demand functions. Then profit-maximizing behavior by firms is derived, treating the number of firms as given. Finally, the assumption of free entry is used to determine the equilibrium number of firms.

The reason that a Chamberlinian approach is useful here is that, in spite of imperfect competition, the equilibrium of the model is determinate in all essential respect because the special nature of demand rules out strategic interdependence among firms. Because firms can costlessly differentiate their products, and all products enter systemtrically into demand, two firms will never want to produce the same product; each good will be produced by only one firm.

COnsider, an individual maximizing utility subject to a budget constraint. The FOC from the maximum problem have the form:

\begin{equation}
    \theta c_i^{\theta - 1} = \lambda p_i, \quad i = 1, \ldots, n
\end{equation}

where $p_i$ is the price on the $i$th good and $\lambda$ is the shadow price on the budget constraint, that is, the marginal utility of income. The single firm faces the demand curve by:

\begin{equation}
    p_i = \theta \lambda^{-1} (x_i / L_i)^{\theta - 1} \quad i = 1, \ldots, n
\end{equation}

Each firm faces a demand curve with an elasticity $1 / (1 - \theta)$, and the profit-maximizing price is therefore,

\begin{equation}
    p_i = \theta^{-1} \beta w
\end{equation}

where $w$ is the wage rate and prices and wages can be defined in terms of any unit. Note that since $\theta, \beta$ and $w$ are the same for all firms, prices are the same for all godos and we can adopt the shorthand $p = p_i$ for all $i$.

The price $p$ is independent of output given the special assumption about cost and utility. To determine profitability, however, we need to look at outputs. Profits of the firm producing good $i$ are:

\begin{equation}
    \pi_i = p x_i - \{\alpha + \beta x_i\}w \quad i = 1, \ldots, n
\end{equation}

its profit are positive, new firms will enter, causing the marginal utility of income to rise and profits to fall until profits are driven to zero. In equilibrium, then $\pi = 0$, implying for the output of a representative firm:

\begin{equation}
    x_i = \alpha / (p / w - \beta) = \alpha \theta / \beta (1 - \theta) \quad i = 1, \ldots, n
\end{equation}

Thus output per firm is determined by the zero-profit condition. Again, since $\alpha, \beta$ and $\theta$ are the same for all firms we can use the shorthand $x = x_i$ for all $i$.

Finally, we can determine the number of goods produced by using the condition of full employment. Then we have:

\begin{equation}
    n = \frac{L}{\alpha + \beta x} = \frac{L(1 - \theta)}{\alpha}
\end{equation}

\subsection{Effects of Trade}

Trades can occur because in the presence of increasing returns, each good will be produced in only one country - for the same reasons that each good is produced by only one firm. The symmetry of the situation ensures that the two countries will have the same wage rate, and that the price of any good produced in either country will be the same. The number of goods produced in each country can be determined from the full employment condition:

\begin{equation}
    n = L(1 - \theta) / \alpha; \quad n^* = L^*(1 - \theta) / \alpha
\end{equation}

where $L^*$ is the labor force of the second country and $n^*$ is the number of goods produced there.

Individuals will maximize their utility but they will now distribute their expenditure over both the $n$ goods produced in the home country and the $n^*$ goods produced in the foreign country. Because of the extended range of choice, welfare will increase even though the ``real wage'' $w/p$ remains unchanged. Also, the symmetry of the problem allows us to determine trade flows. It is apparent that individuals in the home country will spend a fraction $n^* / (n + n^*)$ of their income on foreign goods, while foreigners will spend $n/(n + n^*)$ of their income on home goods. Thus the value of home country imports measured in wage units is $Ln^* / (n + n^*) = LL^* / (L + L^*)$. This equals the value of foriegn country imports, confirming that with equal wage rates in the two conutries we will have balance-of-payments equilibrium.

\section{Transport Costs}

\subsection{Individual Behavior}

Transportation costs will be assumed to be of the ``iceberg'' type, that is, only a fraction $g$ of any good shipped arrives, with $1 - g$ lost in transit.

An individual in the home country will have a choice over $n$ products produced at home and $n^*$ products produced at home and $n^*$ products produced abroad. The price of a domestic product will be the same as that received by the producer $p$. Foreign products, however, will cost more than the producer's price; if foreign firms charge $p^*$, home country consumers will have to pay the c.i.f. price $\hat{p}^* = p^* / g$. Similarly, foreign buyers of domestic products will pay $\hat{p} = p / g$.

Since the price to consumers of goods of differnet countries will in general not be the same, consumption of each imported good will differ from consumption of each domestic good.

To determine world equilibrium, we must also take into account the quantities of goods used up in transit. For determining total demand, then, we need to kow the ratio of total demand by domestic residents for each foreign product to demand for each domestic product. Letting $\sigma$ denote this ratio, and $\sigma^*$ the corresponding ratio for the other country, we can show that:

\begin{equation}
    \begin{aligned}
        \sigma & = (p/p^*)^{1/(1 - \theta)}g^{\theta / (1 - \theta)} \\
        \sigma^* & = (p^*/p)^{1/(1 - \theta)}g^{\theta / (1 - \theta)}
    \end{aligned}
\end{equation}

The overall demand pattern of each individual can then be derived from the requirement that this spending just equal his wage; that is, in the home country we must have $(np + \sigma np^*)d = w$, where $d$ is the consumption of a representative domestic good; and similarly in the foreign country. The behavior of individuals can now be used to analyze the behavior of firms. The important point to notice is that the elasticity of export demand facing any given firm is $1 / (1 - \theta)$, which is the same as the elasticity of domestic demand. Thus transportation costs have no effect on firm's policy/ Writing out these conditions again, we have:

\begin{equation}
    \begin{aligned}
        p & = w\beta / \theta; p^* = w^* \beta / \theta \\
        n & = L(1 - \theta) / \alpha; n^* = L^*(1 - \theta) / \alpha
    \end{aligned}
\end{equation}

\subsection{Determination of Equilibrium}

The only variable which can be affected is the relative wage rate $w / w^* = w$, which no longer need be equal to one. We can determine $w$ by looking at any one of three equivalent market-clearing conditions: (i) equality of demand and supply for home country labor; (ii) equality of demand and supply for foreign country labor; (iii) balance-of-payments equilibrium. It can be shown that the home country's balance of payments measued in wage units of the other country, is:

\begin{equation}
    \begin{aligned}
        B & = \frac{\sigma^* nw}{\sigma^*n + n^*}L^* - \frac{\sigma n^*}{n + \sigma n^*}wL \\
        & = wLL^* \left[ \frac{\sigma^*}{\sigma^* L + L^*} - \frac{\sigma}{L + \sigma L^*}\right]
    \end{aligned}
\end{equation}

Since $\sigma$ and $\sigma^*$ are both functions of $p/p^* = w$, the condition $B = 0$ can be used to determine the relative wage. Since $\sigma$ is an increasing function of $w$ and $\sigma^*$ a decreasing function of $w$, $B(w)$ will be negative (positive) if and only if $w$ is greater (less) than $\overline{w}$, where $\overline{w}$ is the relative wage that the full employment condition holds, and the pattern shows that $\overline{w}$ is the unique equilibrium relative wage.

The simple proposition: the larger the country, other things equal, will have the higher wage. Suppose that we were to compute $B(w)$ for $w = 1$. In that case, we have $\sigma = \sigma^* < 1$. The expression for the balance of payments reduces to:

\begin{equation}
  B = LL^* \left[ \frac{1}{\sigma L + L^*} - \frac{1}{L + \sigma L^*}]
\end{equation}

This equation will be positive if $L > L^*$, negative if $L < L^*$. This means that the equilibrium relative wage $w$ must be greater than one if $L > L^*$, less than one if $L < L^*$.

\section{"Home Market" Effects on the Pattern of Trade}

\subsection{Two-Industry Economy}

Asssume that there are two classes of products, alpha and beta, with many potential products within each class. A tilde will distinguish beta products from alpha products. Demand for the two classes of products will be assumed to arise from the presence of two groups in the production. There will be one group with $L$ members, which derives utility only from consumption of alpha products; and another group with $\tilde{L}$ memebrs, deriving utility only from beta products. The utility functions of representative members of the two classes may be written:

\begin{equation}
  U = \sum_i c_i^{\theta}; \tilde{U} = \sum_j \tilde{c}_j^{\theta} \quad 0 < \theta < 1.
\end{equation}

On the cost side, the two kinds of products will be assumed to have idnetial cost

\begin{equation}
  \begin{aligned}
  l_i & = \alpha + \beta x_i, \quad i = 1, \ldots, n \\
  \tilde{l}_j & = \alpha + \beta \tilde{x}_j j = 1, \ldots, \tilde{n}
\end{aligned}
\end{equation}

where $l_i, \tilde{l}_j$ are labor used in production on typical goods in each class, and $x_i, \tilde{x}_j$ are total outputs of the goods.

The demand conditions now depend on the population shares. We have:

\begin{equation}
  \begin{aligend}
  x_i & = Lc_i \quad i = 1, \ldots, n \\
  \tilde{x}_j & = \tilde{L}\tilde{c}_j \quad j = 1, \ldots, \tilde{n}
  \end{aligned}
\end{equation}

The full-employment condition, however, applies to the economy as a whole:

\begin{equation}
  \sum_{i=1}^n l_i + \sum_{j=1}^{\tilde{n}} \tilde{l}_j = L + \tilde{L}
\end{equation}

Finally, we continue to assume free entry, driving profits to zero.


\chapter{Growth, speculation and sprawl in a monocentric city}

\section{The Model}

A single composite commodity, $Q$, is produced by a competitive industry in the city in both periods. Using subscripts to designate time periods, the endogenous quantities produced are $Q_1$ and $Q_2$. Some $Q$ is produced for export and the rest for local consumption. Regardless of whether it is bound for export or local use, every unit of $Q$ must be transported to the central point at a cost of $t$ per unit of $Q$, per unit of $x$. It is sold there at the (endogenous) price $P_i, i = 1, 2$. The $Q$ production function has fixed factor proportions. It requires exactly $\lambda$ units of land, $\mu$ of labor and $v$ of capital to produce each unit of $Q$. Land rent and labor wages are endogenous to the model, but the cost of capital is exogenous; it is $s$ per unit throughout the city in both periods.

Each provides a single unit of labor to the $Q$ industry per period, receives (endogenous) wages of $w$ per period, and has the same utility function. Because the city is open, migration will assure that every resident household attains the (exogenous) level of utility attained elsewhere in the naional economy. Arguments of the utility function are $Q$, residential land, and another composite commodity, $Z$, that is imported from outside the city and sold at the central point at unitary price in both periods. Nonland components of housing services are absorbed in $Q$ and $Z$. Households must consume exactly $q$ units of $Q, z$ of $Z$, and $1$ unit of residential land in each period. To ensure that local consumption of $Q$ does not exhaust production, we assume that:

\begin{equation*}
    1 / \mu > q
\end{equation*}

Households value residential proximity to the central point because workers must commute to work by passing through it and shoppers must shop at it. The combined cost of these trips  is $T$ per household, per unit of $x$, per period.

The exogenous growth mechanism in the model is an increase in export demand for $Q$ between two periods. The export demand function is $f(P_i, \Gamma_i)$, where $\Gamma_i \in 1, 2$, is an exogenous demand-shift variable that encodes all the information required to predict the usual price-quantity demand relation. Further,

\begin{equation*}
    \frac{\partial f}{\partial \Gamma_i} > 0, \frac{\partial f}{\partial P_i} < 0, i = 1, 2 \text{ and } \Gamma_1 < \Gamma_2
\end{equation*}

No other exogenous variables change between periods.

\section{Perfect Foresight Planning}

Suppose landowners have perfect foresight and know at the outset what their land rent will be in both periods for all possible development strategies. This requires that $\Gamma_2$ be known with certainty in period $1$, and means that all development decisions - whether executed in the first or second period - are made simultanously in period $1$. We call this perfect-foresight planning. It is to be distinguished from speculation, examined later, which occurs when landowners are uncertain about $\Gamma_2$ and future land rents.

In first-period equilibrium, households will reside in a Von-Thunen ring outside another where $Q$ is produced if only:

\begin{equation*}
    t / \lambda > T
\end{equation*}

Similarly, second-hand residential development will lie more distant from the central point than second-period industrial development. Depending on the relative magnitudes of $t / \lambda$ and $T$, it may be advantageous for landowners to withhold from development in period $1$ a ring of land between the industrial and residential zones, and preserve it for second-period industrial use.

In particular, it will always occur if:

\begin{equation*}
    t / \lambda > (2 + r) T,
\end{equation*}

where $r$ is the (universal) discount rate between periods. Since $-t/\lambda$ and $-T$ are shown below to be the slope of industrial and residential bid-rent functions, this condition requires that industrial bid-rents decrease more than twice as rapidly with $x$ as residential bid-rents.

\subsection{First-Period Equations}

equilibrium in period $1$ is characterized by eight endogenous variables. Three of them are $Q_1$, $P_1$ and $w_1$. Two other indicate land rents in the residential and industrial zones. The remaining three are spatial boundaries: $x_a$, the outer edge of the industrial zone; and $x_b$ and $x_c$, the inner and outer boundary of the residential zone. From previous assumptions, these will satisfy:

\begin{equation*}
    0 \leq x_a \leq x_b \leq x_c
\end{equation*}

To identify the industrial land rent variable, note that firms in the competitive $Q$ industry must earn zero profit. Thus land rent and transportation charges must exhaust revenues after payments are made to capital and labor. Since the latter costs are the same for every firm, on a per-unit-of-$Q$ basis, land rent and transportation charges per-unit-of-$Q$ must be the same for every firm. And since each unit of $Q$ requires exactly $\lambda$ units of land, this sum is the same for every firm on a per-unit-of-land basis as well. In period $1$, we call this amount $R_1^Q$ per unit of land. Because total transportation charges per-unit-of-land are $tx / \lambda$ at $x$, the rent on industrial land is indicated by the linear function $R_1^Q - tx / \lambda$.

To  identify the residential land rent variable, note that each household has the same income, consumes the same consumption bundle, and faces the same prices for $Q$ and $Z$. Their expenditure on land and transportation charges must therefore be the same at every location. In period $1$, we call this amount $R_1^H$. Because household transportation charges are $Tx$ at $x$, the rent on residential land is indicated by the linear function $R_1^H - Tx$.

First-period equilibrium is characterized by eight conditions. One of them equates the supply and demand for $Q_1$. This means local production of $Q$ must equal the sum of quantities demanded for export and local consumption. Where $Q_1$ units are produced, $\mu Q_1$ households are required to supply the corresponding amount of labor. This means local consumption of $Q_1$ must be $\mu qQ_1$. $Q$-market equilibrium requires that:

\begin{equation}
    Q_1 (1 - \mu q) = f(P_1, \Gamma_1)
\end{equation}

A second condition is that household budgets balance:

\begin{equation}
    w_1 = R_1^H + P_1 q + z.
\end{equation}

A third is that competitive Q-firms earn zero profit:

\begin{equation}
    P_1 = \mu w_1 + vs + \lambda R_1^Q
\end{equation}

Each unit of $Q$ requires $\lambda$ unit of land, so $Q_1$ is related to $x_a$ by:

\begin{equation}
    Q_1 = \pi x_a^2 / \lambda
\end{equation}

The last three conditions concern equilibrium in the land market. At $x_a$ industrial land rent must be zero, since otherwise land beyond it would be offered for first-period industiral development or land inside it withheld from development until later. This requires that:

\begin{equation}
    R_1^Q - tx_a / \lambda.
\end{equation}

For similar reasons, residential land rent at $x_c$ must be zero:

\begin{equation}
    R_1^H - Tx_c = 0.
\end{equation}

The equilibrium condition concerning $x_b$ is less easily stated. Owners of land near $x_a$ must decide only whether to develop industrially in the first or second period. Those near $x_c$ make a similar decision for residential development. The present value of the first option is:

\begin{equation*}
    R_1^H - Tx + \frac{1}{1 + r}(R_2^H - Tx)
\end{equation*}

The present value of the second option is:

\begin{equation*}
    \frac{1}{1 + r}(R_2^Q - tx / \lambda)
\end{equation*}

$x_b$ is the location where these strategies are equally profitable:

\begin{equation*}
    R_1^H - Tx_b = \frac{1}{1 + r}[(R_2^Q - tx_b / \lambda) - (R_2^H - Tx_b)]
\end{equation*}

Those eight equations are not a closed system since the last equation includes the second period, endogenous variables, $R_2^H, R_2^Q$. To solve this equalibrium, it is necessary to solve both periods' equations simultanously, as in a two-period dynamic problem.

\subsection{Second-Period Equations}

Equations in period $2$ is characterized by six variables, five of which are $Q_2, P_2, w_2, R_2^Q, R_2^H$. The other one is $x_d$, the outer boundary of second-period, residential expansion. Of course,

\begin{equation*}
    x_d \geq x_c
\end{equation*}

The second period conditions corresponding to (1) - (3) are:

\begin{equation}
    \begin{aligned}
        Q_2(1 - \mu q) & = f(P_2, \Gamma_2) \\
        w_2 & = R_2^H + P_2 q + z \\
        P_2 & = \mu w_2 + vs + \lambda R_2^Q
    \end{aligned}
\end{equation}

Since $\Gamma_2 > \Gamma_1$, $Q$-production will be greater in period $2$ and the industrial zone will expand from $x_a$ to $x_b$. Thus, $Q_2$ is related to $x_b$ by:

\begin{equation}
    Q_2 = \pi x_b^2 / \lambda
\end{equation}

The increase in production requires an expansion in the residential zone from $x_c$ to $x_d$. In order that this expansion allow a total of $Q_2$ households to reside in the city:

\begin{equation}
    Q_2 = \pi(x_d^2 - x_b^2) / \mu
\end{equation}

The final condition is that land rent at $x_d$ is zero:

\begin{equation}
    R_2^H - Tx_d = 0
\end{equation}

condition (1) - (14) is a closed system of fourteen, independent conditions in fourteen endogenous variables.

\subsection{The Equilibrium}

The equilibrium land rent function in the first period is:

\begin{equation*}
    \begin{aligned}
        ER_1(x) & = R_1^Q - tx / \lambda \text{ for } x \in [0, x_a] \\
        & = R_1^H - Tx \text{ for } x \in [x_a, x_c] \\
        & = 0 \text{ for } x \in [x_c, \infty)
    \end{aligned}
\end{equation*}

It is continuous at every $x$ except $x_b$. Land just beyond $x_b$ earns a positive rent while land inside it earns none. The equilibrium land-rent in the second-period is:

\begin{equation*}
    \begin{aligned}
        ER_2(x) & = R_2^Q - tx / \lambda \text{ for } x \in [0, x_b] \\
        & = R_2^H - Tx \text{ for } x \in [x_b, x_d] \\
        & = 0 \text{ for } x \in [x_d, \infty)
    \end{aligned}
\end{equation*}

Note first that $ER_2(x) \geq ER_1(x), \forall x \geq 0$ and second that $ER_2(x)$ is also continuous at every $x$ but $x_b$. In the second period, however, land just beyond $x_b$ earns less than land just inside it. The function indicating equilibrium present value of both period's land rent,

\begin{equation*}
    PV(x) = ER_1(x) + \frac{1}{1 + r}ER_2(x)
\end{equation*}

is continuous at every $x$.

\subsection{Comparative Statics}

Consider now the effect on both periods' spatial equilibrium of a change in $\Gamma_2$, all other exogenous variables remaining the same. We have:

\begin{equation*}
    \frac{\partial x_a}{\partial \Gamma_2} < 0, \frac{\partial x_b}{\partial \Gamma_2} > 0, \frac{\partial x_c}{\partial \Gamma_2} > 0, \frac{\partial x_d}{\partial \Gamma_2} > 0
\end{equation*}

All boundaries move outward as $\Gamma_2$ increases except $x_a$, which moves inward. Since $x_b$ increases and $x_a$ decreases with $\Gamma_2$, the second period industrial zone is larger to accomodate increased production of $Q$. Since the residential-to-individual land ratio is constant, the second-period residential zone must be greater. Thus the annulus of land added by the increase in $x_d$ is greater than that lost by the increase in $x_b$.

\section{Uncertainty and Speculation}

We now drop the perfect foresight assumption and suppose instead that landowners are uncertain in period $1$ about second-period land rent. In particular, they share the probability distribution $g(\Gamma_2)$ over $\Gamma_2$ in the first period where the true value of $\Gamma_2$ is resolved in period $2$. We retain the assumption that $\Gamma_2 > \Gamma_1$, so:

\begin{equation*}
    \int_0^{\Gamma_1} g(\Gamma_2) d\Gamma_2 = 0
\end{equation*}

We assume landowners are risk neutral. In this environment, landowners' first-period decisions are speculative. This means the monocentric city model equilibrium is the result of a sequential decision proess. The smaller the value of $\Gamma_2$ resolved, the less will be needed. A related difference is that depending upon the value of $\Gamma_2$ resolved, the outer boundary of industrial expansion may not be $x_b$. Because of this, we introduce $x_e$ as the boundary. While $x_b$ is determined in period $1$ on speculation, $x_e$ is determined along with $x_d$ in period $2$ once $\Gamma_2$ is resolved. If $\Gamma_2$ is small, then $x_e < x_b$ indicating that more land than necessary was preserved. If $\Gamma_2$ is sufficiently large, then $x_e > x_b$, indicating all preserved land is used and that a second industiral ring occurs beyond the first-period residential zone.

A final difference is that if $\Gamma_2$ is sufficiently small, land rents can be negative at some locations in the second period where residential development occured in the first period.

\subsection{Second-Period Equilibrium}

The second-period equilibrium is characterized by seven variables, six from perfect-foresight planning and one from speculation. The six are $Q_2, P_2, w_2, R_2^Q, R_2^H, x_d$. The seventh is $x_e$. 

Case 1:

We begin with the lower tail of $g(\Gamma_2)$. Suppose:

\begin{equation*}
    \Gamma_2 = \Gamma_1 + \delta
\end{equation*}

where $\delta$ is positive but very small, indicating miniscule growth in export demand for $Q$ between periods. The amount of land needed for both industrial and residential expansion is much less than that preserved by speculation between $x_a$ and $x_b$. It will occupy an annulus of land between $x_a$ and $x_e$. The expanded workforce will be accomodated by an annulus of residential development between $x_e$ and $x_d$ where:

\begin{equation*}
    x_a < x_e < x_d < x_b < x_c.
\end{equation*}

The amount of land in the industrial zone must equal $\lambda Q_2$, and that in the two residential zone $\mu Q_2$. This provides two of the remaining four equilibrium conditions:

\begin{equation}
    Q_2 = \pi x_e^2 / \lambda
\end{equation}

\begin{equation}
    Q_2 = \pi(x_c^2 - x_b^2 + x_d^2 - x_e^2) / \mu
\end{equation}

The other two conditions concern equilibrium in the land market. In other that land beyond $x_d$ not be offered for residential development, or land inside it withheld, residential land rent must be zero at $x_d$:

\begin{equation}
    R_2^H = Tx_d. \labeL{sepculation_residential_land_rent}
\end{equation}

In order that land beyond $x_e$ not be developed industrially,or land inside it residentially, land rent must be the same for both uses at $x_e$:

\begin{equation}
    R_2^Q - R_2^H = (t / \lambda - T)x_e.
\end{equation}

An important feature of equilibrium in Case 1, and a direct implication of the above, is that second-period land rent is negative throughout the residential zone developed in the first period.

As $\Gamma_2$ increases, more land is required for both kinds of expansion, so $x_e$ and $x_d$ increase. Once $x_d$ reaches $x_b$, there is no longer any vacant land in the leapfrog zone to keep residential land rent zero at $x_d$. Thus the above Equation \eqref{sepculation_residential_land_rent} is no longer valid and Case 1 no longer applies. Let the value of $\Gamma_2$ for which:

\begin{equation*}
    R_2^H = Tx_b
\end{equation*}

be called $\Gamma_2^1$. That is the minimum value of $\Gamma_2$ for which

\begin{equation}
    x_d = x_b
\end{equation}

Case 1 and set of conditions above apply only when

\begin{equation*}
    \Gamma_1 < \Gamma_2 < \Gamma_2^1
\end{equation*}

Case 2:

Now consider:

\begin{equation*}
    \Gamma_2 = \Gamma_2^1
\end{equation*}

All land in the leapfrog will be developed here. The equilibrium shows that:

\begin{equation*}
    x_e = x_c \sqrt{\lambda / (\mu + \lambda)}
\end{equation*}

The case is distinguished from the previous one in two ways. First, as $\Gamma_2$ increases above $\Gamma_2^1$, $x_d$ and $x_e$ remain stationary. Second, land rent is negative in only part of the residential zone. As $\Gamma_2$ and consequently $R_2^H$ increase, the point beyond which rent is negative becomes more remote. Let the value of $\Gamma_2$ for which:

\begin{equation*}
    R_2^H = Tx_c.
\end{equation*}

be called $\Gamma_2^2$. That is the minimum value of $\Gamma_2$ for which $x_d = x_b$ holds. Thus the result applies $\Gamma_2^1 \leq \Gamma_2 \leq \Gamma_2^2$.

Case 3:

Suppose

\begin{equation*}
    \Gamma_2 = \Gamma_2^2 + \delta.
\end{equation*}

We now have:

\begin{equation*}
    x_a < x_e < a_b < x_c < x_d.
\end{equation*}

Since the amount of land in the industrial zone must be $\lambda Q_2$ and that in the residential zone must be $\mu Q_2$, we have:

\begin{equation}
    Q_2 = \pi (x_d^2 - x_e^2) / \mu
\end{equation}

As $\Gamma_2$ increases, both zones expand as $x_e$ and $x_d$ increase. but this expansion cannot continue indefinitely because $x_e$ eventually reaches $x_b$. Once this happens, there is no longer any residential development in the leapfrog zone to keep both rents equal at $x_e$. Let the value of $\Gamma_2$ for which:

\begin{equation*}
    R_2^Q - R_2^H = (t / \lambda - T)x_b
\end{equation*}

be called $\Gamma_2^3$. That is the minimum value of $\Gamma_2$ for which

\begin{equation}
    x_e = x_b
\end{equation}

Therefore, $\Gamma_2^2 < \Gamma_2 < \Gamma_2^3$.

Case 4:

\begin{equation*}
    \Gamma_2 = \Gamma_2^3
\end{equation*}

so that all land in the leapforg zone is developed industrially. Taken together, we have $x_d$ can be solved for as a function of $x_b$:

\begin{equation*}
    x_d = x_b \sqrt{1 + \mu / \lambda}
\end{equation*}

Let the value of $\Gamma_2$ for which:

\begin{equation*}
    R_2^Q - R_2^H = (t / \lambda - T)x_c
\end{equation*}

be called $\Gamma_2^4$. That is the minimum value of $\Gamma_2$ for which $x_e = x_b$.

Then $\Gamma_2^3 \leq \Gamma_2 \leq \Gamma_2^4$.

Case 5:

\begin{equation*}
    \Gamma_2 > \Gamma_2^4
\end{equation*}



% !TEX root = ../notes_template.tex
\chapter{}

\section{Some Theory}

% \includepdf[pages=-]{homework/IOE516_HW1_Solutions.pdf}

\chapter{THE MAKING OF THE MODERN METROPOLIS: EVIDENCE FROM LONDON}

\section{Theoretical Framework}

We consider a city embedded in a wider economy (Great Britain). The economy as a whole consists of a discrete set of locations $\mathbb{M}$. Greater London is a subset of these locations $\mathbb{N} \subset \mathbb{M}$, Time is discrete and is indexed by $t$. The economy as a whole is populated by an exogenous continuous measure $L_{\mathbb{M}t}$ of workers, who are geographically mobile and wndowed with one unit of labor that is supplied inelastically. Workers simultaneously choose their preferred residence $n$ and workplace $i$ given their idiosyncratic draws. We denote the endogenous measure of workers who choose a residence-workplace pair in Greater London by $L_{\mathbb{N}t}$. We allow locations to differ from one another in terms of their attractiveness for production and residence, as determined by productivity, amenities, the supply of floor space, and transport connections, where each of these location characteristics can evolve over time.

\subsection{Preferences}

We assume that preferences take the CD-form, such that the indirect utility for a worker $\omega$ residing in $n$ and working in $i$ is:

\begin{equation}
    U_{ni}(\omega) = \frac{B_{ni}b_{ni}(\omega)w_i}{\kappa_{ni}P_n^{\alpha} Q_n^{1 - \alpha}}, 0 < \alpha < 1
\end{equation}

where we suppress the time subscript from now on; $P_n$ is the price index for consumption goods, which may include both tradeable and nontradeable consumption goods; $Q_n$ is the price of residential floor space; $w_i$ is the wage, $\kappa_{ni}$ is an iceberg commuting cost; $B_{ni}$ captures amenities from the bilateral commute from residence $n$ to workplace $i$ that are common across all workers; and $b_{ni}(\omega)$ is an idiosyncratic amenity draw that captures all the idiosyncratic factors that can cause an individual to live and work in particular locations in the city.

We assume that idiosyncratic amenities $(b_{ni}(\omega))$ are drawn from an independent extreme value (Frechet) distribution for each residence-workplace pair and each worker:

\begin{equation}
    G(b) = e^{-b^{-\varepsilon}}, \varepsilon > 1 \label{eq:frechet}
\end{equation}

where we normalize the Frechet scale parameter in Equation \eqref{eq:frechet} to $1$ because it enters worker choice probabilities isomorphically to common bilateral amenities $B_{ni}$. The Frechet shape parameter $\varepsilon$ regulates the dispersion of idiosyncratic amenities, which controls the sensitivity of worker location decisions to economic variables. The smaller the shape parameter $\varepsilon$, the greater the heterogeneity in idiosyncratic amenities, and the less sensitive are worker location decisions to economic variables.

We decompose the bilateral common amenities parameter $(B_{ni})$ into a residence component common across all workplaces $(B_n^{\mathcal{R}})$, a workplace component common across all residences $(B_i^L)$, and an idiosyncratic component $(B_{ni}^I)$ specific to an individual residence-workplace pair:

\begin{equation}
    B_{ni} = B_n^{\mathcal{R}}B_i^L B_{ni}^I, \quad B_n^{\mathcal{R}}, B_i^L, B_{ni}^I > 0
\end{equation}

We allow the levels of $B_n^{\mathcal{R}}, B_i^I$ and $B_{ni}^I$ to differ across residences $n$ and workplace $i$, although when we examine the impact of the construction of railway network, we assume that $B_i^L$ and $B_{ni}^I$ are time-invariant. In contrast, we allow $B_n^{\mathcal{R}}$ to change over time, and for those changes to be potentially endogenous to the evolution of the surrounding concentration of economic activity through agglomeration forces.

Conditional on choosing a residence-workplace pair in Greater London, we know that the probability a worker chooses to reside in location $n \in \mathbb{N}$ and work in location $i \in \mathbb{N}$ is given by:

\begin{equation}
    \begin{aligned}
        \lambda_{ni} & = \frac{L_{ni}}{L_{\mathbb{M}}} \frac{L_{\mathbb{M}}}{L_{\mathbb{N}}} = \frac{L_{ni}}{L_{\mathbb{N}}} \\
        & = \frac{(B_{ni} w_i)^{\varepsilon} (\kappa_{ni} P_n^{\alpha} Q_n^{1 - \alpha})^{-\varepsilon}}{\sum_{k \in \mathbb{N}} \sum_{\ell \in \mathbb{N}} (B_{k\ell}w_{\ell})^{\varepsilon} (\kappa_{k\ell} P_{k}^{\alpha} Q_k^{1 - \alpha})^{-\varepsilon} }, n,  i \in \mathbb{N}
    \end{aligned}
\end{equation}

where $L_{ni}$ is the measure of commuters from $n$ to $i$.

The probability of commuting between residence $n$ and workplace $i$ depends on the characteristics of that residence $n$, the attributes of that workplace $i$ and bilateral commuting costs and amenities. Summing across workplaces $i \in \mathbb{N}$, we obtain the probability that a worker lives in residence $n \in \mathbb{N}$, conditional on choosing a residence-workplace pair in Greater London $(\lambda_n^R = \frac{R_n}{L_{\mathbb{N}}})$. Similarly, summing across residences $n \in \mathbb{N}$, we obtain the probability that a worker is employed in workplace $i \in \mathbb{N}$, conditional on choosing a residence-workplace pair in Greater London $(\lambda_i^L = \frac{L_i}{L_{\mathbb{N}}})$

\begin{equation}
    \begin{aligned}
        \lambda_n^R & = \frac{\sum_{i \in \mathbb{N}} (B_{ni} w_i)^\varepsilon (\kappa_{ni} P_n^{\alpha} Q_n^{1 - \alpha})^{-\varepsilon}}{\sum_{k \in \mathbb{N}} \sum_{\ell \in \mathbb{N}} (B_{k\ell} w_{\ell})^{\varepsilon} (\kappa_{k\ell} P_k^{\alpha} Q_k^{1 - \alpha})^{-\varepsilon}     } \\
        \lambda_i^L & = \frac{\sum_{n \in \mathbb{N}} (B_{ni} w_i)^{\varepsilon} (\kappa_{ni} P_n^{\alpha} Q_n^{1 - \alpha})^{-\varepsilon}}{\sum_{k \in \mathbb{N}} \sum_{\ell \in \mathbb{N}} (B_{k\ell} w_{\ell})^{\varepsilon} (\kappa_{k\ell} P_k^{\alpha} Q_k^{1 - \alpha})^{-\varepsilon}}
    \end{aligned}
\end{equation}

where $R_n$ denotes employment by residence in location $n$ and $L_i$ denotes employment by workplace in location $i$. A second implication of our extreme value specification is that expected utility conditional on choosing a residence workplace pair $(\overline{U})$ is the same across all residence-workplace pairs in the economy:

\begin{equation}
    \overline{U} = v\left[ \sum_{k \in \mathbb{M}} \sum_{\ell \in \mathbb{M}} (B_{k\ell}w_{k\ell})^{\varepsilon} (\kappa_{k\ell} P_k^{\alpha} Q_k^{1 - \alpha})^{-\varepsilon} \right]^{\frac{1}{\varepsilon}} 
\end{equation}

where the expectation is taken over the distribution for idiosyncratic amenities; $v \equiv \Gamma(\frac{\varepsilon - 1}{\varepsilon})$; $\Gamma(\cdot)$ is the gamma function. Using the probability that a worker chooses a residence-workplace pair in Greater London $(\frac{L_{\mathbb{N}}}{L_{\mathbb{M}}})$, we can rewrite this probability mobility condition as:

\begin{equation}
    \overline{U}(\frac{L_{\mathbb{N}}}{L_{\mathbb{M}}})^{\frac{1}{\varepsilon}} = v\left[ \sum_{k \in \mathbb{N}} \sum_{\ell \in \mathbb{N}} (B_{k\ell} w_{k\ell})^{\varepsilon} (\kappa_{k\ell} P_k^{\alpha} Q_k^{1 - \alpha})^{-\varepsilon} \right]^{\frac{1}{\varepsilon}}
\end{equation}

where only the limits of the summations differ on the right hand sides of the equations.

Intuitively, for a given common level of expected utility in the economy $(\overline{U})$, locations in Greater London must offer higher real wages adjusted for common amenities $(B_{ni})$ and commuting costs $(\kappa_{ni})$ to attract workers with lower idiosyncratic draws with an elasticity determined by the parameter $\varepsilon$.

\subsection{Production}

We assume that consumption goods are produced according to a Cobb-Douglas technology using labor, machinery capital, and commercial floor space, where commercial floor space includes both building capital and land. Cost minimization and zero profits imply that payments for labor, commercial floor space, and machinery are constant shares of revenue ($X_i$):

\begin{equation}
    w_i L_i = \beta^L X_i, q_i H_i^L = \beta^H X_i, rM_i = \beta^M X_i, \beta^L + \beta^H + \beta^M = 1
\end{equation}

where $q_i$ is the price of commercial floor space; $H_i^L$ is commercial floor space use; $M_i$ is machinery use; and machinery is assumed to be perfectly mobile across locations with a common price $r$ determined in the wider economy. We allow the price of commercial floor space $(q_i)$ to potentially depart from the price of residential floor space $(Q_i)$ in each location $i$ through a location-specific wedge $(\xi_i)$:

\begin{equation}
    q_i = \xi_i Q_i.
\end{equation}

From the relationship between factor payments and revenue in equation, payments for commercial floor space are proportional to workplace income $(w_i L_i)$:

\begin{equation}
    q_i H_i^L = \frac{\beta^H}{\beta^L} w_i L_i
\end{equation}

\subsection{Commuter Market Clearing}

commuter market clearing implies that the measure of workers employed in each location $(L_i)$ equals the measure of workers chooosing to commute to that location:

\begin{equation}
    L_i = \sum_{n \in \mathbb{N}} \lambda_{ni \mid n}^R R_n
\end{equation}

where $\lambda_{ni \mid n}^R$ is the probability of commuting to workplace $i$ conditional on living in residence $n$:

\begin{equation}
    \lambda_{ni \mid n}^R = \frac{ \left(\frac{B_{ni} w_i}{\kappa_{ni}}\right)^{\varepsilon}}{\sum_{\ell \in \mathbb{N}} \left(\frac{B_{n\ell} w_{\ell}}{\kappa_{n\ell}}\right)^{\varepsilon}}
\end{equation}

where all characteristics of residence $n$ have canceled from the above equation because they do not vary across workplaces for a given residence.

Commuter market clearing also implies that per capita income by residence $(v_n)$ is a weighted average of the wages in all locations, where the weights are these conditional commuting probabilities by residences $(\lambda_{ni \mid n}^R)$:

\begin{equation}
    v_n = \sum_{i \in \mathbb{N}} \lambda_{ni \mid n}^R w_i.
\end{equation}

\subsection{Land Market Clearing}

We assume that floor space is owned by landlords, who receive payments from the residential and commercial use of floor space and consume only consumption goods. Land market clearing implies that total income from the ownership of floor space equals the sum of payments for residential and commercial floor space use:

\begin{equation}
    \mathbb{Q}_n = Q_n H_n^R q_n H_n^L = (1 - \alpha) \left[ \sum_{i \in \mathbb{N}} \lambda_{ni \mid n}^R w_i \right] R_n + \frac{\beta^H}{\beta^L} w_n L_n
\end{equation}

where $H_n^R$ is residential floor space use; rateable values $(\mathbb{Q}_n)$ equals the sum of prices times quantities for residential floor space $(Q_n H_n^R)$ and commercial floor space $(q_n H_n^L)$; and we have used the expression for per capita income by residence $(v_n)$ from commuter market clearing.

From the combined land and commuter market-clearing condition, payments for residential floor space are a constant multiple of residence income $(v_n R_n)$, and payments for commercial floor space are a constant multiple of workplace income $(w_n L_n)$. Importantly, we allow the supplies of residential floor space $(H_n^R)$ and commercial floor space $(H_n^L)$ to be endogenous, and we allow the prices of residential and commercial floor space to potentially differ from one another through the location-specific wedge $\xi_i(q_i = \xi_i Q_i)$. In our baseline quantitative analysis below, we are not required to make assumptions about these supplies of residential and commercial floor space or this wedge between commercial and residential floor prices. The reason is that we condition on the observed rateable values in the data $(\mathbb{Q}_n)$ and the supplies and prices for residential and commercial floor space $(H_n^R, H_n^L, Q_n, q_n)$ only after the land market-clearing condition.

\section{Quantitative Analysis}

\subsection{Combined Land and Commuter Market Clearing}

We evaluate the effect of changes in the transport network by using an "exact hat algebra" approach. In particular, we rewrite our combined land and commuter market clearing condition for another year $\tau \neq t$ in terms of the values of variables in a baseline year $t$ and the relative changes of variables between years $\tau$ and $t$:

\begin{equation}
    \hat{\mathbb{Q}}_{nt} \mathbb{Q}_{nt} = (1 - \alpha) \hat{v}_{nt} v_{nt} \hat{R}_{nt} R_{nt} + \frac{\beta^H}{\beta^L} \hat{w}_{nt} w_{nt} \hat{L}_{nt} L_{nt}
\end{equation}

where $\hat{x}_{nt} = \frac{x_{n\tau}}{x_{nt}}$ for the variable $x_{nt}$ and we now make explicit the time subscripts. The relative change in employment $(\hat{L}_{it})$ and the relative change in average per capita income by residence $(\hat{v}_{nt})$ for year $\tau$ can be expressed as:

\begin{equation}
    \hat{L}_{it} L_{it} = \sum_{n \in \mathbb{N}} \frac{\lambda_{nit\mid n}^R \hat{w}_{it}^{\varepsilon} \hat{\kappa}_{nit}^{-\varepsilon}}{\sum_{\ell \in \mathbb{N}} \lambda_{n\ell t \mid n}^R \hat{w}_{\ell t}^{\varepsilon} \hat{\kappa}_{n\ell t}^{-\varepsilon}  } \hat{R}_{nt} R_{nt} 
\end{equation}

\begin{equation}
    \hat{v}_{nt} v_{nt} = \sum_{i \in \mathbb{N}} \frac{\lambda_{nit \mid n}^R \hat{w}_{it}}{\sum_{\ell \in \mathbb{N}} \lambda_{n\ell t \mid n}^R \hat{w}_{\ell t}^{\varepsilon} \hat{\kappa}_{n\ell t}^{-\varepsilon}  } \hat{w}_{it} w_{it}
\end{equation}

where these equations include terms in change in wages $(\hat{w}_{n})$ and commuting costs $(\hat{\kappa}_{ni})$ but not in amenities, because we assume that the workplace and bilateral components of amenities are constant $(\hat{B}_{it}^L = 1$ and $\hat{B}_{nit}^I = 1)$, and changes in the residential component of amenities $(\hat{B}_{nt}^R \neq 1)$ cancel from the numerator and denominator of the fractions.

substituting the expressions to the market clearing conditions for year $\tau$ we can get the result.

\begin{lemma}
    Suppose that $(\hat{\mathbb{Q}}_{nt}, \hat{R}_{nt}, L_{nit}, \lambda_{nit \mid n}^R, \mathbb{Q}_{nt}, v_{nt}, R_{nt}, w_{nt}, L_{nt})$ are known. Given known values for model parameters $\{\alpha, \beta^L, \beta^H, \varepsilon\}$ and the change in bilateral commuting costs $(\hat{\kappa}_{nit}^{-\varepsilon})$, the combined land and commuter market clearing condition determines a unique vector of relative changes in wages $(\hat{w}_{it})$ in each location.
\end{lemma}
% Now we start to lecture 12

\chapter{Urban Diversity, Process Innovation, and the Life Cycle of Products}

\section{The model}

There are $N$ cities in the economy, where $N$ is endogenous, and a continuum $L$ of infinitely lived workers, each of which has one of $m$ possible discrete aptitudes. There are equal proportion of workers with each aptitude in the economy, but their distribution across cities is endogenously determined through migration. Let us index cities by $i$ and worker aptitudes by subcript $j$ so that $l_i^j$ denotes the supply of labor with aptitude $j$ in city $i$. Time is discrete and indexed by $t$.

\subsection{Technology}

The ideal production process is firm specific and randomly drawn from a set of $m$ possible discrete processes, with equal probability for each. Each of the $m$ possible processes for each firm requires process specific intermediate inputs from a local sector employing workers of a specific aptitude. We say that two production processes for different firms are of the same type if they require intermediates produced using workers with the same aptitude.

A newly created firm  does not know its ideal production process, but it can find this by trying, one at a time, different processes in the production of prototypes. After producing a prototype with a certain process, the firm knows whether this process is its ideal one or not. Thus in order to switch from prototype to mass production a firm needs to have produced a prototype with its ideal process first, or to have tried all of its $m$ possible processes except one. Furthermore, we allow for the possibility that a firm decides to stop searching before learning its ideal process. Firms have an exogenous probability $\delta$ of closing down each period. Firms also lose one period of production whenever they relocate from one city to another. Thus, the cost of firm relocation increases with the exogenous probability of closure $\delta$.

THe intermediates specific to each type of process are produced by a monopolistically competitive intermediate sector, each such intermediate sector hires workers of aptitude $j$ and sells process-specific nontradable intermediate services to final-good firms using a process of type $j$. These differentiated services enter the production function of final good producers with the same constant elasticity of substitution $\frac{\varepsilon + 1}{\varepsilon}$. The production is:

\begin{equation}
  \overset{C}{?}_i^j(h) = Q_i^j \overset{x}{?}_i^j (h)
\end{equation}

\begin{equation}
  \text{ where } Q_i^j = (l_i^j)^{-\varepsilon} w_i^j, \varepsilon > 0 
\end{equation}

We distinguish variables corresponding to prototypes from those corresponding to mass-produced goods by an accent in the form of a question mark, ?. INdexing the differentiated varieties of goods by $h$, we denote ouput of prototype $h$ made with a process of type $j$ in city $i$ by $\overset{x}{?}_i^j(h)$. $Q_i^j$ is the unit cost for firms producing prototypes using a process of type $j$  in city $i$ and $w_i^j$ is the wage per unit of labor for the corresponding workers. Note that $Q_i^j$ decreases as $l_i^j$ increases: there are localization economies that reduce unit costs when there is a larger supply of labor with the relevant aptitude in the same city.

When a firm finds its ideal production process, it can engage in mass production at a fraction $\rho$ of the cost of producing a prototype, where $0 < \rho < 1$. Thus the cost function for a firm engaged in mass production is:

\begin{equation}
  C_i^j(h) = \rho Q_i^j x_i^j(h)
\end{equation}

where $x_i^j(h)$ denotes the ouput of mass produced good $h$, made with a process of type $j$, in city $i$.

With respect to the internal structure of cities, there are congestion costs in each city incurred in labor time and parameterized by $\tau (> 0)$. Labor supply, $l_i^j$, and production, $L_i^j$, with aptitude $j$ in city $i$ are related by the following expression:

\begin{equation}
  l_i^j = L_i^j(1 - \tau \sum_{j=1}^m L_i^j).
\end{equation}

THe expected wage income of a worker with aptitude $j$ in city $i$ is then $(1 - \tau \sum_{j=1}^m L_i^j)w_i^j$, where the higher land rents pair by those living closer to the city center are offset by lower commuting costs.

\subsection{Preferences}

Each period consumers allocate a fraction $\mu$ of their expenditure to prototypes and a fraction of $1 - \mu$ to mass-produced goods. The instantaneous indirect utility of a consumer in city $i$ is:

\begin{equation}
  V_i = \overset{P}{?}^{-\mu} P^{-(1 - \mu)}e_i^j 
\end{equation}

where $e_i$ denotes individual expenditure.

\begin{equation}
  \overset{P}{?} = \left\{ \sum_{j=1}^m \int \int [\overset{p}{?}_i^j (h)]^{1 - \sigma} dh di \right\}^{1/(1 - \sigma)} 
\end{equation}

\begin{equation}
  P = \left\{ \sum_{j=1}^m \int \int [p_i^j (h)]^{1 - \sigma} dh di \right\}^{1/(1 - \sigma)} 
\end{equation}

and the appropriate price indices of prototypes and mass-produced goods respectively, and $\overset{p}{?}_i^j(h)$ and $p_i^j(h)$ denote the price of individual varieties of prototypes and mass-produced goods respectively. All prototypes enter consumer preferences with the same elasticity of substitution $\sigma (> 2)$, and so do all mass produced goods.

\subsection{Income and Migration}

National income, $Y$, is the sum of expenditure and investment:

\begin{equation}
  Y = \sum_{j=1}^m \int L_i^j e_i^j di + \overset{P}{?}^{\mu} P^{1 - \mu} F \overset{n}{\circ}
\end{equation}

$L_i^j$ denotes population with aptitude $j$ in city $i$. Investment $\overset{P}{?}^{\mu} P^{1 - \mu}F\overset{n}{\circ}$ comes from aggregation of the start-up costs incurred by newly created firms. To come up with a new product, the firm must spend $F$ on market research, purchasing the same combination of goods bought by the representative consumer. Finally, $\overset{n}{\circ}$ denotes the total number of new firms.

\begin{definition}[Specialized City]
  A city is said to be fully specialzied if all its workers have the same aptitude, so that all local firms use the same type of production process.
\end{definition}

\begin{definition}[Diversified City]
 A city is said to be (fully) diversified if it has the same proportion of workers with each of the $m$ aptitudes, so that there are equal proportions of firms using each of the $m$ types of production process. 
\end{definition}

\subsection{City Formation}

Each potential site for a city is controlled by a different land development company or land developer, not all of which will be active in equilibrium. Developers have the ability to tax local land rents and to make transfers to local workers. When active, each land developer commits to a contract with any potential worker in its city that specifies the size of the city, whether it has a dominant sector and if so which, and any transfers.

\subsection{Equilibrium Definition}

Finally, a steady state equilibrium in this model is a configuration such that all of the following are true. Each developer offers a contract designed so as to maximize its profits. Each firm chooses a location/production strategy and prices so as to maximize its expected lifetime profits. All profit opportunities are exploited and then urban structure is constant over time.

\section{Equilibrium City Sizes}

\begin{lemma}[Output per Worker]
  In equilibrium, output per worker by firms using processes of type $j$ in city $i$ in a given period is:

  \begin{equation*}
    \frac{\overset{n}{?}_i^j \overset{x}{?}_i^j + \rho n_i^j x_i^j}{L_i^j} = (L_i^j)^{\varepsilon} (1 - \tau \sum_{j=1}^m L_i^j)^{\varepsilon + 1}.
  \end{equation*}
\end{lemma}


\chapter{Spatial Sorting and Inequality}

\section{Change in SKill Sourting: Framework}

\subsection{Setup}

On the production side, rather than modelling imperfect trade between locations, we consider an economy that is more stylized spatially, with two types of goods: (a) a homogeneous manufactured good that is freely traded across space and (b) housing, a local nontraded good.

\subsubsection{Preferences}

Consider a spatial equilibrium framework with two skill groups $\theta = U, S$, who choose where to live among locations $i \in [1, \ldots, N]$. Aggregate skill supply for each group, $L^{\theta}$, is exogenously given, and each worker supplies one unit of labor for wage $w_i^{\theta}$ in location $i$. The utility of worker $w$, who is type $\theta$ and lives in location $i$, is:

\begin{equation}
  u_i^{\theta}(w) = \max_{c, b} \log U^{\theta}(A_i, c, b) + \varepsilon_i^{\theta}(w), \text{ such that } c + r_i b = w_i^{\theta} 
\end{equation}

Here $\log U^{\theta}(\cdot)$ is the representative utility of a worker of type $\theta$; $c$ is the consumption of the freely traded good and is taken as the numeraire; $b$ denote housing, with price $r_i$ in location $i$; and $A_i$ is a vector of amenities in location $i$. Finally, $\varepsilon_i^{\theta}(w)$ is a worker-specific preference shock for living in location $i$.

First, we make assumption of CD type preferences over traded and nontraded goods. Second, we assume that amenities are separable from consumption. We allow amenities in location $i$ to be valued differently by the two groups, as caputured by a group-specific amenity level $A_i^{\theta}$. Third, preference shocks are typically chosen to be extreme value (EV) distributed. Papers in the tradition of urban and labor economics or industrial organization tend to use logit shocks, with normalized variance $\frac{\pi^2}{6}$ shifted by a factor $\frac{1}{\kappa^{\theta})}$, which together with CD utility lead to the following indirect utility of worker $\theta$ in location $i$:

\begin{equation*}
  v_i^{\theta}(w) = \log A_i^{\theta} + \log w_i^{\theta} - \alpha^{\theta} \log r_i + \frac{1}{\kappa^{\theta}} \varepsilon_i^{\theta}(w)
\end{equation*}

Equivalently, papers in the tradition of trade and economic geography typically choose Frechet shocks for $\varepsilon_i^{\theta}(w)$ with scale parameter $\kappa^{\theta} > 1$ that enter utility in a multiplicatively separable way. In that case, the indirect utility of worker $\theta$ in location $i$ is:

\begin{equation*}
  v_i^{\theta}(w) = \frac{A_i^{\theta} w_i^{\theta}}{r_i^{\alpha^{\theta}}} \varepsilon_i^{\theta}(w).
\end{equation*}

In either case, location choices in group $\theta$ can be summarized with $\lambda_i^{\theta}$, the share of $\theta$ workers who choose location $i$:

\begin{equation}
  \lambda_i^{\theta} = \frac{ (\frac{A_i^{\theta} w_i^{\theta}}{r_i^{\alpha^{\theta}}})^{\kappa^{\theta}}}{ \sum_{j=1}^N ( \frac{A_j^{\theta} w_j^{\theta}}{r_j^{\alpha^{\theta}}})^{\kappa^{\theta}} }
\end{equation}

The parameter $\kappa^{\theta}$ captures the elasticity of population shares with respect to amenity-adjusted real wages and is therefore a measure of mobility of group $\theta$, which we allow to be group specific. Expected utility for a worker in group $\theta$ across locations is:

\begin{equation}
  W^{\theta} = \delta^{\theta} \left[ \sum_{k=1}^N (\frac{A_i^{\theta} w_i^{\theta}}{r_i^{\alpha^{\theta}}})^{\kappa^{\theta}} \right]^{\frac{1}{\kappa^{\theta}}}
\end{equation}

where $\delta^{\theta} = \Gamma(\frac{\kappa^{\theta} - 1}{\kappa^{\theta}})$ and $\Gamma(\cdot)$ is the gamma function in the Frechet case.

\subsubsection{Supply of goods, amenities, and housing}

We first write down the labor demand side of the economy. In location $i$, output is produced by perfectly competitive firms. They combine skilled and unskilled labor, who are imperfect substitutes in production:

\begin{equation}
  Y_i = \left[ (z_i^U)^{\frac{1}{\rho}} (L_i^U)^{\frac{\rho - 1}{\rho}} + (z_i^S)^{\frac{1}{\rho}} (L_i^S)^{\frac{\rho - 1}{\rho}} \right]^{\frac{\rho}{\rho - 1}}
\end{equation}

In the CES production function, $\rho \geq 1$ is the elasticity of substitution between skills and $z_i^{\theta}$ are location- and skill specific productivity shifters. The shifters can be in part exogenous and in part endogenous, reflecting externalities. We assume that, for $\theta = \{U, S\}$ and $\forall i$,

\begin{equation}
  z_i^{\theta} = z^{\theta}(\overline{Z}_i, L_i^U, L_i^S) 
\end{equation}

where $\overline{Z}_i$ is the exogenous productivity component in city $i$. Local productivity spillovers are allowed here to depend not just on city size or density but also on its composition $(L_i^U, L_i^S)$. Given equation, relative labor demand in location $i$ is:

\begin{equation}
  \log (\frac{L_i^S}{L_i^U}) = \log (\frac{z_i^S}{z_i^U}) - \rho \log(\frac{w_i^S}{w_i^U}) 
\end{equation}

Furthermore, competition across cities ensures that the unit cost of production in all cities is $1$, the common price of the freely traded good:

\begin{equation*}
  \sum_{\theta} z_i^{\theta} (w_i^{\theta})^{1 - \rho} = 1, \forall i
\end{equation*}

Similar to productivity, amenities $A_i^{\theta}$ are assumed to be driven by both exogenous differences, and endogenous differences between cities, that is,

\begin{equation}
  A_i^{\theta} = A^{\theta}(\overline{A}_i, L_i^U, L_i^S)
\end{equation}

where $A_i$ is the exogenous amenity component of city $i$. Endogenous amenities capture elements of quality of life that change when the size or composition of cities changes.



\chapter{Urban Accounting and Welfare}

\section{The Model}

\subsection{Technology}

Consider a model of a system of cities in an economy with $N_t$ workers. Goods are produced in $I$ monocentric circular cities. Cities have a local level of productivity. Production in city $i$ in period $t$ is given by:

\begin{equation*}
    Y_{it} = A_{it}K_{it}^{\theta}H_{it}^{1 - \theta}
\end{equation*}

where $A_{it}$ denotes city productivity, $K_{it}$ denotes total capital, and $H_{it}$ denotes total hours worked in the city. We denote the population size of city $i$ by $N_{it}$. The standard first-order conditions of this problem are:

\begin{equation}
    w_{it} = (1 - \theta)\frac{Y_{it}}{H_{it}} = (1 - \theta) \frac{y_{it}}{h_{it}} \text{ and } r_t = \theta \frac{Y_{it}}{K_{it}} = \theta \frac{y_{it}}{k_{it}}
\end{equation}

where lowercase letters denote per capita variables. Note that capital is freely mobile across locations so there is a national interest rate $r_t$. Mobility patterns will not be determined solely by the wage, $w_{it}$, so there may be equilibrium differences in wages across cities at any point in time. We can then write down the "efficiency wedge", which is identical to the level of productivity, $A_{it}$, as:

\begin{equation}
    A_{it} = \frac{Y_{it}}{K_{it}^{\theta}H_{it}^{1 - \theta}} = \frac{y_{it}}{k_{it}^{\theta} h_{it}^{1 - \theta}}
\end{equation}

\subsection{Preferences}

Agents order consumption and hour sequenecs according to the following utility function:

\begin{equation*}
    \sum_{t=0}^\infty \beta^t \left[ \log c_{it} + \psi \log(1 - h_{it}) + \gamma_i \right],
\end{equation*}

where $\gamma_i$ is a city-specific amenity and $\psi$ is a parameter that governs the relative preference for leisure. Each agent lives on one unit of land and commutes from his home to work. Commuting is costly in terms of goods. The problem of an agent in city $i_0$ with cpaital $k_0$ is therefore,

\begin{equation*}
    \max_{\{c_{i_t, t}, h_{i_t, t}, k_{i_t, t}, i_{t}\}_{t=0}^\infty} \sum_{t=0}^\infty \beta^t \left[ \log c_{it} + \psi \log (1 - h_{it}) + \gamma_i \right]
\end{equation*}

subject to the budget constraint:

\begin{equation*}
    \begin{aligned}
        c_{it} + x_{it} & = r_{t} k_{it} + w_{it} h_{it} (1 - \tau_{it}) - R_{it} - T_{it} \\
        k_{it + 1} & = (1 - \delta) k_{it} + x_{it}
    \end{aligned}
\end{equation*}

where $x_{it}$ is investment, $\tau_{it}$ is a labor tax or friction associated with the cost of building the commuting infrastructure, $R_{it}$ are land rents, and $T_{it}$ are commuting costs.

We assume that we are in steady state so $k_{it + 1} = k_{it}$ and $x_{it} = \delta k_{it}$. Furthermore, we assume $k_{it}$ is such that $r_t = \delta$. The simplified budget constraint of the agent becomes:

\begin{equation}
    c_{it} = w_{it} h_{it} (1 - \tau_{it}) - R_{it} - T_{it}.
\end{equation}

The first-order conditions of this problem imply $\psi \frac{c_{it}}{1 - h_{it}} = (1 - \tau_{it}) w_{it}$. Combining the expression with the first equation, we obtain:

\begin{equation}
    (1 - \tau_{it}) = \frac{\psi_{it}}{(1 - \theta)} \frac{c_{it}}{1 - h_{it}} \frac{h_{it}}{y_{it}}
\end{equation}

We refer to $\tau_{it}$ as the "labor wedge". Although $\tau_{it}$ is modeled as a labor tax, it should be interpreted more broadly as anything that distorts an agent's optimal labor supply decision. Agents can move freely across cities so utility in each period has to be determined by:

\begin{equation}
    \overline{u} = \log c_{it} + \psi \log(1 - h_{it}) + \gamma_i
\end{equation}

for all cities with $N_{it} > 0$, where $\overline{u}$ is the economy-wide per period utility of living in a city.

\subsection{Commuting Cost, Land Rents, and City Equilibrium}

Cities are monocentric, all production happens at the center, and people live in surrounding areas characterized by their distance to the center, $d$. Cities are surroundedby a vast amount of agricultural land that can be freely converted into urban land. We normalize the price of agricultural land to zero. Since land rents are continuous in equilibrium, this implies that at the boundary of a city, $\overline{d}_{it}$, land rents should be zero as well, namely, $R(\overline{d}_{id}) = 0$. Since, all agents in a city are identical, in equilibrium they must be indifferent between they live in a city, which implies that the total cost of rent plus commuting costs should be identical in all areas of a city. So,

\begin{equation*}
    R_{it}(d) + T(d) = T(\overline{d}_{it}) = \kappa \overline{d}_{it}, \forall d \in [0, \overline{d}_{it}]
\end{equation*}

since $T(d) = \kappa d$ where $\kappa$ denotes commuting costs per mile.

Everyone lives on one unit of land, $N_{it} = \overline{d}_{it}^2 \pi$, and so $\overline{d}_{it} = (N_{it} / \pi)^{\frac{1}{2}}$. Thus $R_{it} + T(d) = \kappa (N_{it} / \pi)^{\frac{1}{2}}$. This implies that $R_{it}(d) = \kappa(\overline{d}_{it} - d)$ and so total land rents in a city of size $N_{it}$ are given by $TR_{it} = \int_0^{\overline{d}_{it}} (\kappa (\overline{d}_{it} - d) d2\pi)dd = \frac{\kappa}{3} \pi^{-\frac{1}{2}} N_{it}^{\frac{3}{2}}$. Hence, arrange land rents are equal to $AR_{it} = \frac{2\kappa}{3} (\frac{N_{it}}{\pi})^{\frac{1}{2}}$. Taking logs and rearranging terms, we obtain that:

\begin{equation}
    \log (N_{it}) = o_1 + 2 \log AR_{it}.
\end{equation}

where $o_1$ is a constant. We can also compute the total miles traveled by commuters in the city, which is given by:

\begin{equation}
    TC_{it} = \int_0^{\overline{d}_{it}} (d^2 2\pi) dd = \frac{2}{3} \pi^{-\frac{1}{2}} N_{it}^{\frac{3}{2}}
\end{equation}

\subsection{Government Budget Constraint}

The government levies a labor tax, $\tau_{it}$, to pay for the transportation infrastructure. Let government expenditure be a function of total commuting costs and wages such that:

\begin{equation*}
    G(h_{it} w_{it}, TC_{it}) = g_{it} h_{it} w_{it} \kappa TC_{it} = g_{it} h_{it} w_{it} \kappa \frac{2}{3} \pi^{-\frac{1}{2}} N_{it}^{\frac{3}{2}}
\end{equation*}

where $g_{it}$ is a measure of government inefficiency. That is, the government requires $\kappa g_{it}$ workers per mile commuted to build and maintain urban infrastructure. The government budget constraint is then given by:

\begin{equation}
    \tau_{it} h_{it} N_{it} w_{it} = g_{it} h_{it} w_{it} \kappa \frac{2}{3} \pi^{-\frac{1}{2}} N_{it}^{\frac{3}{2}}
\end{equation}

which implies that the labor wedge can be written as:

\begin{equation}
    \tau_{it} = g_{it} \kappa \frac{2}{3} (\frac{N_{it}}{\pi})^{\frac{1}{2}}
\end{equation}

or

\begin{equation}
    \log \tau_{it} = o_2 + \frac{1}{2} \log N_{it} + \log g_{it}
\end{equation}

\subsection{Equilibrium}

The consumer budget constraint is given by:

\begin{equation*}
    c_{it} = w_{it} h_{it} (1 - \tau_{it}) - R_{it} - T_{it} = (1 - \theta) (1 - \tau_{it}) y_{it} - \kappa (\frac{N_{it}}{\pi})^{\frac{1}{2}}
\end{equation*}

To determine output we know that the proudction function is given by $y_{it} = A_{it} k_{it}^{\theta} h_{it}^{1 - \theta}$ and the decision of firms to rent capital implies that $r_t k_{it} = \theta y_{it}$. Hence,

\begin{equation*}
    y_{it} = A_{it} (\frac{\theta y_{it}}{r_t})^{\theta} h_{it}^{1 - \theta} = A_{it}^{\frac{1}{1 - \theta}} (\frac{\theta}{r_t})^{\frac{\theta}{1 - \theta}} h_{it}.
\end{equation*}

Using the above result, we can derive

\begin{equation*}
    h_{it} = \frac{1}{1 + \psi} (1 + \frac{\psi(R_{it} + T_{it})}{(1 - \theta)(1 - \tau_{it})} \frac{(\frac{r_t}{\theta})^{\frac{\theta}{1 - \theta}}}{A_{it}^{\frac{1}{1 - \theta}}})
\end{equation*}

and

\begin{equation*}
    c_{it} = \frac{1}{1 + \psi} \left[(1 - \theta)(1 - \tau_{it}) (\frac{\theta}{r_t})^{\frac{\theta}{1 - \theta}}A_{it}^{\frac{1}{1 - \theta}} - (R_{it} + T_{it})\right]
\end{equation*}

The free mobilility assumption in the result implies that $\overline{u}_t = \log c_{it} + \psi \log(1 - h_{it}) + \gamma_{it}$ for some $\overline{u}_{t}$ determined in general equilibrium so:

\begin{equation}
    \begin{aligned}
        \overline{u}_{it} & + (1 + \psi) \log(1 + \psi) - \psi \log \psi \\
        & = \log \left( (1 - \theta) (1 - \kappa g_{it} \frac{2}{3} (\frac{N_{it}}{\pi})^{\frac{1}{2}}\right) \frac{A_{it}^{\frac{1}{1 - \theta}}}{(r_t / \theta)^{\frac{\theta}{1 - \theta}}} - \kappa (\frac{N_{it}}{\pi})^{\frac{1}{2}} \right) \\
        & + \psi \log \left( 1 - \frac{\kappa (\frac{N_{it}}{\pi})^{\frac{1}{2}}}{(1 - \theta)(1 - \kappa g_{it} \frac{2}{3} (\frac{N_{it}}{\pi})^{\frac{1}{2}})} \frac{(\frac{r_t}{\theta})^{\frac{\theta}{1 - \theta}}}{A_{it}^{\frac{1}{1 - \theta}}} \right) + \gamma_{it}.
    \end{aligned}
\end{equation}

The last equation determines the size of the city $N_{it}$ as implicit function of city productivity $A_{it}$, city amenities, $\gamma_{i}$, government inefficiency, $g_{it}$, and economy wide variables like $r_t$ and $\overline{u}_{it}$. In the above euqation, the LHS is decreasing in $N_{it}$. THe LHS is also increasing in $A_{it}$ and $\gamma_i$ and decreasing in $g_{it}$. Hence, we can prove immediately that:

\begin{equation}
    \frac{\partial N_{it}}{\partial A_{it}} > 0, \frac{\partial N_{it}}{\partial \gamma_i} > 0, \frac{\partial N_{it}}{\partial g_{it}} < 0 \label{eq:decreasing_increasing_condition}
\end{equation}

The economy wide utility level $\overline{u}_t$ is determined by the labor market clearing conditions

\begin{equation}
    \sum_{i=1}^I N_{it} = N_t
\end{equation}

This last equation clarifies that our urban system is closed; we do not consider urban-rural migration.

\section{Evidence of Efficiency, Amenities, and Frictions}

\subsection{Empirical Approach}

We start by estimating the following equation:

\begin{equation}
    \log N_{it} = \alpha_1 + \beta_1 \log A_{it} + \varepsilon_{it}
\end{equation}

THe value of $\beta_1$ yields the effect of the "efficiency wedge" on city population. According to the model, $\beta_1 > 0$ by the Equation \eqref{eq:decreasing_increasing_condition}. Furthermore, $\log N_{it}(A_{it}) = \beta_1 \log A_{it}$ is the population size explained by the size of the "efficiency wedge". In contrast, $\varepsilon_{1it}$ is the part of the observed population in the city that is unrelated to the productivity; according to the model it is related to both $g_{it}$ and $\gamma_i$. We can then estimate the following equation: $\tilde{\varepsilon}_{1}(g_{it}, \gamma_{it}) \equiv \varepsilon_{1it}$.

Since the "efficiency wedge" increases population size, total commuting increases, which affects the "labor wedge". This is the standard urban trade-off between productivity and agglomeration. We can estimate the effect of producitvity on the labor wedge and the decomposition of $\log N_{it}$ into $\log \tilde{N}_{it}(A_{it})$ and $\varepsilon_{1it}$. Hence, we estiamte:

\begin{equation}
    \log \tau_{it} = \alpha_2 + \beta_2 \log \tilde{N}_{it}(A_{it}) + \varepsilon_{2it}
\end{equation}

We denote the effect of efficiency on distortions by $\log \tilde{\tau}_{it} = \beta_2 \log \tilde{N}_{it}(A_{it})$. We can then estimate the following equation: $\tilde{\varepsilon}_{2it} \equiv \varepsilon_{2it}$. The equation also implies that the error term $\varepsilon_{2it}$ is related to $g_{it}$ and to $\tilde{\varepsilon}_{1}(g_{it}, \gamma_{it})$. Hence, we define $\tilde{\varepsilon}_{2}(g_{it}, \tilde{\varepsilon}_{1}(g_{it}, \gamma_{it})) \equiv \varepsilon_{2it}$.

We now can decompose the effect from all three elements of $(A_{it}, \gamma_{it}, g_{it})$. To do so, we estimate:

\begin{equation}
    \log (AR_{it}) = \alpha_3 + \beta_3 \log \tilde{\tau}_{it} + \beta_4 + \varepsilon_{1it} + \beta_5 \varepsilon_{2it} + \varepsilon_{3it}
\end{equation}

using median rents for $AR_{it}$. The effect of $\gamma_{it}$ and $g_{it}$ are determined by the estimates of $\beta_4$ and $\beta_5$. Note that $\varepsilon_{1it}$ and $\varepsilon_{2it}$ depend on both $\gamma_{it}$ and $g_{it}$. However, since $\varepsilon_{2it} = \tilde{\varepsilon}_{2}(g_{it}, \tilde{\varepsilon}_{1}(g_{it}, \gamma_{it}))$ depend only on $\gamma_{it}$ through $\varepsilon_{1it}$ and we are including $\varepsilon_{1it}$ directly in the regression, $\beta_5$ capture only the effect of changes in $g_{it}$ on land rents.

Note that we can then use the equation to relate average rents and population sizes. So we can estiamte the model using 

\begin{equation}
    \log (N_{it}) = \alpha_4 + \beta_6 \log (AR_{it}) + \varepsilon_{4it}
\end{equation}

In a circular city $\beta_6 = 2 > 0$.

\subsection{Effects of Efficiency, Amenities, and Frictions on City Size}

We can decompose the labor wedge into taxes and other distortions such that

\begin{equation}
    (1 - \tau_{it}) = (1 - \tau_{it}^\prime) (\frac{1 - \tau_{ith}}{1 + \tau_{ith}})
\end{equation}

where $\tau_{it}$ is our measure of the labor wedge, $\tau_{ith}$ is the labor tax rate, $\tau_{itc}$ is the consumption tax rate, and $\tau_{it}^\prime$ are other distortions. Thus expect our measure of the total labor wedge, $(1 - \tau_{it})$, to be correlated with $(1 - \tau_{ith}) / (1 + \tau_{itc})$.

\chapter{Interacting Agents, Spatial Externalities and the Evolution of residential land use patterns}

\section{Optimal Timing of Development}

Define $A(i, t)$ as the returns to the original, unsubdivided parcel (denoted $i$) in the undeveloped use in time period $t$. We will refer to this as agriculture, broadly defined to include any uses of the land in an undeveloped state. Conversion of parcel $i$ at time $T$ will require the agent to incur costs to reap expected one-time gross returns. Costs include the provision of subdivision infrastructure, as well as permitting and other administrative fees. We denote $\delta$ as the discount factor, defined as $\frac{1}{1 + r}$ where $r$ is the interest rate, and the one-time returns from development minus costs of conversion in time $T$ as $V(i, T)$. Then the net returns from developing parcel $i$ in time $T$ equals the one time net returns minus the present value of foregone agricultural returns and is given by:

\begin{equation}
    V(i, T) - \sum_{t=0}^\infty A(i, T + t) \delta^t \label{eq:one_time_returns}
\end{equation}

The net returns from keeping parcel $i$ in an agricultural use in period $T$ and developing in time period $T + 1$, discounted to time $T$, are:

\begin{equation}
    A(i, T) + \delta V(i, T+ 1) - \sum_{t=0}^\infty A(i, T + 1) \delta^t \label{eq:agricultural_returns}
\end{equation}

The optimal development time will occur in period $T$ only if the Equation \eqref{eq:one_time_returns} is positive and if Equation \eqref{eq:one_time_returns} is greater than Equation \eqref{eq:agricultural_returns}. The optimal development time is the time period $T$ if:

\begin{equation}
    V(i, T) - \sum_{t=0}^\infty A(i, T + t) \delta^{T + t} > 0 \label{eq:optimal_development}
\end{equation}

and

\begin{equation}
    V(i, T) - A(i, T) \geq \delta V(i, T + 1) \label{eq:optimal_development2}
\end{equation}

The agent develops in periop $T$ only if (a) the net value of conversion is positive and (b)

\begin{equation*}
    \frac{V(i, T + 1) - \{V(i, T) - A(i, T)\}}{V(i, T) - A(i, T)} < r
\end{equation*}

where $r$ is the interest rate. Here we are interested in explaining the scattered pattern of exurban development. Such a pattern would result if net negative interactions were present, e.g., due to congestion externalities, and if these effects were sufficiently strong as to create a `repelling' effect among residential development.

Let $\delta_{s} I_s(i, t)$ represent this spillover effect, where $I_s(i, t)$ is the proportion of neighboring parcels that are in a developed state at time the development decision is made, $\delta_s$ is the interaction parameter, and $s$ indexes the order of the spaital lag, which increases with increasing distance from parcel $i$. Because neighboring developed lands could conceivably have positive and/or negative spillover effects, the parameter $\lambda_s$, which represents the net effect of these spillovers, could be either positive or negative at any given distance $s$.

Rewritting the Equation \eqref{eq:one_time_returns} to incorporate the effect of interactions, the net returns from developing parcel $i$ in time period $T$ equals:

\begin{equation}
    V(i, T) + \sum_{s} \lambda_s I_s(i, T) - \sum_{t=0}^\infty A(i, T + t) \delta^{T + t} \label{eq:one_time_returns_with_interactions}
\end{equation}

Rewritting  the conversion rule in Equation \eqref{eq:optimal_development2}, development now occurs in the first period in which:

\begin{equation}
    V(i, T) - \delta V(i, T + 1) - A(i, T) + \sum_s (1 - \delta) \lambda_s I_s(i, T) \geq 0 \label{eq:optimal_development_with_interactions}
\end{equation}

where the value of $I(i, T)$ is assumed to be constant between periods $T$ and $T + 1$. 

\section{The Empirical Model}

\subsection{Hazard model of development}

To take account of these differences across agents, define $\varepsilon_i$ as these unobservable factors associated with the owner of parcel $i$. Given that $\varepsilon$ is viewed by the research as a stochastic variable, the following gives the probability that parcel $i$, with surrounding land use pattern $\sum_s I_s(i, T)$ will be converted by period $T$:

\begin{equation}
    \text{Prob} \{\varepsilon_i < \frac{1}{1 - \delta} (V(i, T) - \delta V(i, T + 1) - A(i, T)) + \sum_s \lambda_s I(i, T)\}
\end{equation}

This implies that agents with large $\varepsilon$'s, such as those who are particularly good farmers or those that place a particular high value on their undeveloped land as a source of direct utility, will convert later than agents with the same type of parcel but smalle values of $\varepsilon$.

The probability that a parcel with a given set of characteristics will be converted in period $T$ is its hazard rate for period $T$, which is given by:

\begin{equation}
    h(T) = \frac{F[\varepsilon^*(T + 1)] - F[\varepsilon^*(T)]}{1 - F[\varepsilon^*(T)]}
\end{equation}

where $F$ is the cumulative distribution function for $\varepsilon$ and define $\varepsilon^*$ as the $\varepsilon$ that makes the equation an equality. In this analysis, we choose Cox's partial likelihood method because it can accomodate time-varying covariates. Assuming that $V(i, t)$ is separable in these factors that are time varying, but spatially constant, we can apply Cox's model to our problem by defining a baseline hazard rate that is a function of time only. Let $\omega(T)$ represent the exponential of this baseline hazard rate and assume that the log of the hazard rate it linear in other arguments. Then the hazard rate for parcel $i$ is given by:

\begin{equation}
    h(i, T) = \omega(T) \times \exp(Z\beta).
\end{equation}

where $Z$ is a vector of parcel $i$'s attributes, including $I_s(i, T)$, and $\beta$ is a corresponding parameter vector. Cox's method is a semiparametric approach that relies on formulating the likelihood in such a way that the baseline hazard, $\omega(T)$, drops out and therefore specification of an error distribution is unnecessary. It is the product of $N$ contributions to the likelihood function, where $N$ is the number of developable parcels, and the form of the $n$th contribution is given by:

\begin{equation}
    L_n = \frac{h(n, T_n)}{\sum_{j=1}^{J_n} h(j, T_n)}
\end{equation}

By definition, $T_n$ is the time at which the $n$th parcel is converted. In the above equation, $h(n, T_n)$ is the hazard rate for the $n$th parcel, $h(j, T_n)$ is the hazard rate for the $j$th parcel, but evaluated at time $T_n$, and $J_n$ is the set of parcels that have `survived' in the undeveloped state until time $T_n$.

\subsection{The identification Problem}

The vector of $Z(i)$ contains all attributes associated with parcel $i$ and the stochastic term $\varepsilon_i$ captures the existence of idiosyncratic factors associated with agent $i$.



\chapter{When is the economy monocentric}

\section{A formal model of a spatial economy}

Consider a long-narrow country, in which area is represented by one-dimensional unbounded location space, $X = \mathbb{R}$. The quality of land is homogenous and density of land is equal to $1$ everywhere. The country has a continuum of homogenous workers with a given size, $N$. Each worker is endowed with a unit of labor, and is free to chooose any location and job in the country. The consumers of the country consist of the workers and landlords.

Each consumer consumes a homogenous agricultural good (A-good) together with a continuum of differentiated manufacturing goods (M-goods) of size $n$. Here, $n$ is to be determined endogenously. All consumers have the same utililty function given by:

\begin{equation}
    u = \alpha_A \log z + \alpha_M \log \left\{ \int_0^n q(\omega)^{\rho} d\omega \right\}^{1 / \rho}
\end{equation}

where $z$ represents the amount of A-good consumed, $q(\omega)$ is the consumption of each variety $\omega \in [0, n]$ of M-good; and $\alpha_A, \alpha_M$ and $\rho$ are positive constants such that $\alpha_A + \alpha_M = 1$ and $0 < \rho < 1$. Note that a smaller $\rho$ means that consumers have a stronger preference for variety in M-goods.

Suppose that a consumer has an income, $Y$, and faces a set of prices $p_A$ and $p_M(\omega)$. Then for choosing the consumption bundle that maximizes the equation above, subject to the budget constraint $p_A z + \int_0^n p_M(\omega) q(\omega) d\omega = Y$, demand functions of the consumer can be obtained as:

\begin{equation}
    z = (\alpha_A Y) / p_A
\end{equation}

\begin{equation}
    q(\omega) = (\alpha_M Y / p_M(\omega)) \left(p_M(\omega)^{-\gamma} / \int_0^n p_M(\omega)^{-\gamma} d\omega \right) \label{eq:demand_for_q}
\end{equation}

for each $\omega \in [0, n]$, where $\gamma = \rho / (1 - \rho)$. Note that the Equation \eqref{eq:demand_for_q} that the demand for any variety in M-good has the same price elasticity, $E$, given by:

\begin{equation}
    E = 1 / (1 - \rho) = 1 + \gamma
\end{equation}

Thus $E$ increases as $\rho$ increases. Substituting $z$ and $q(\omega)$ into the utility function yields the following indirect utility function:

\begin{equation}
    u = \log\{\alpha_A^{\alpha_A} \alpha_M^{\alpha_M}Yp_A^{\alpha_A}\} + \frac{\alpha_M}{\gamma} \log \left\{ \int_0^n (\omega)^{-\gamma} d\omega \right\}
\end{equation}

Next, the A-good is assumed to be produced under constant returns, where each unit of A-good consumes a unit of land and $a_A$ units of labor.

\chapter{The endogenous formation of a city: population agglomeration and marketplaces in a location-specific production economy}

\section{A general model}

\subsection{An overview of the economy}

There are $I$ (finite and integer) produced assumption commodities in the economy. There is a finite number of different locations in the economy. The location set is denoted by $J \subset \mathbb{R}^m$, where $m$ is a positive integer representing the dimension of the location space and $J$ is finite. Each location $j \in J$ is just a point, but it contains a positive amount of homogeneous land. Marketplaces can be established in a feasible marketplace location set $D \subset \mathbb{R}^m$. The set $D$ is assumed to be compact. A consumer chooses her location $j$ from $J$ taking into account, for each location, the wage, rent, and commuting cost to the closest marketplace.

\subsection{Marketplaces}

The location of a marketplace can be anywhere in $D$. No marketplace requires land: a marketplace is a point $d \in D$. The locations of the marketplaces are denoted by $\{d_1, d_2, \ldots, d_k\} = \bf{d} \subset D$. Let $K$ be the collection of finite sets in $D(\emptyset \in K)$. We assume that the transportation cost of commodities between marketplaces is negligible, while to access marketplaces consumers have to use their time endowments.

\subsection{Consumers}

There is a continuum of consumers. The set of consumers is denoted $A = [0, 1]$, and a representative element of $A$ is $a$. The consumers from an atomless measure space $(A, \mathcal{A}, v)$, where $\mathcal{A}$ is the Borel sigma algebra of $A$ and $v$ is the Lebesgue measure on $A$. By definition, $v(A) = 1$.

\subsection{Individual transportation}

We assume that travel cost from location $j$ to a marketplace $d$ is represented by $\tilde{\delta}_j(d)$ where $\tilde{\delta}_j(d): D \to \mathbb{R}_+$ is a continuous function. Travel cost is assumed to be independent of the quantity of goods transported. Since a consumer accesses a marketplace that is most convenient, time to travel to a marketplace in a marketplace structure $\bf{d}$ for a consumer residing at $j \in J$ is given by $\delta_j(\bf{d})$, where $\delta_j: K \to \mathbb{R}_+$ is such that $\delta_j(\bf{d}) = \min_{d \in \bf{d}} \tilde{\delta}_j(d)$. The Euclidean distance from location $j$ to the closest marketplace given marketplace structure $\bf{d} \in K$ is an example of $\delta_j(\bf{d})$, i.e., $\delta_j(\bf{d}) = 2 \min_{d \in \bf{d}} \norm{j - d}$.

\subsection{Mass transportation among marketplace}

When there is more than one marketplace, it is necessary to build a mass transportation system to have commodity flows among them. One can imagine the situation where there is a railroad station in front of each marketplace.

\subsection{The trading cost}

It is assumed that each consumer must reside in exactly one location in $J$. Since consumers can choose their locations freely, consumption sets are non-convex, and even disconnected. This fact does not depend on the finitess of the location set $J$. Even if $J$ is a convex subsset of $\mathbb{R}^m$, the trading cost is always non-convex since each consumer needs to choose one location as her residence. Since production is location-specific, we cannot treat labor as a homogeneous good, since it has a different effect on output at different locations. A trading set is a consumption set where the endowment point is normalized to the origin. Let $\Omega = \Omega_c \times \Omega_{\ell} \times \Omega_L \times \Omega_J$ be the potential trading set, which denote potential commodity, leisure, and land trading sets, while $\Omega_J = J$ denotes the potential location choice set, respectively. Consumers' trading sets with no transportation costs are represented by the closed-valued measurable correspondence $X: A \mapsto \Omega$.

\chapter{ON THE INTERNAL STRUCTURE OF CITIES}

\section{The Model}

We consider a circular city of fixed radius $S$, located in a large economy. A single traded good is produced within the city, which is sold to the larger economy at a competitive price. Labor is supplied elastically at the reservation utility $\overline{u}$ that prevails in the larger economy. Workers have preferences over units of the produced good and the quantity of residential land that they consume. We would treat land as available at the boundary of the city at a price $q_f$, determined by its value in an agricultural use. We take the radius $S$ of the city as given.

The total land area of the city, $\pi S^2$, is divided between production use and residential use. We describe locations within the city by their pollar coordinates $(r, \phi)$, but for most purposes we consider only symmetric equilibria, where nothing depends on $\phi$, and refer simply to "location $r$". Let $\theta(r)$ be the fraction used for production. Let the employment density - employment per unit of production land - at location $r$ be $n(r)$, implying that total employment at $r$ is $2\pi r \theta(r) n(r)$. Let $N(r)$ be the number of workers housed at $r$, per unit of residential land. Then if each such person occupies $\ell(r)$ units of land, we have $\ell(r) N(r) = 1$.

Production technology involves: ordinary CRS that relates land, labor and the technology level to goods production. There is the external effect that relates technology level at any one location to the employment, weighted by distance, at other locations. Finally, there is a cost - in units of lost labor time - to commuting to and from work.

Production of the traded good at location $r$ is assumed to be a CRS of land, $2 \pi r \theta(r)$, and labor, $2\pi r \theta(r) n(r)$, at that location. Production per unit of land at location $r$ can thus be written as:

\begin{equation}
    x(r) = g(z(r))f(n(r))
\end{equation}

the functions $g$ and $f$ are taken to be CD:

\begin{equation}
    g(z) = z^{\gamma}
\end{equation}

and

\begin{equation}
    f(n) = An^{\alpha}
\end{equation}

The intercept term $g(z(r))$ is a productivity term that reflects an external effect on production at location $(r, 0)$ of employment at neighboring locations $(s, \phi)$. This production externality is assumed to be linear, and to decay exponentially at a rate $\delta$ with the distance between $(r, 0)$ and $(s, \phi)$:

\begin{equation*}
    z(r) = \delta \int_0^S \int_0^{2\pi} s\theta(s, \phi)n(s, \phi) e^{-\delta x(r, s, \phi)} d\phi ds,
\end{equation*}

where $x(r, s, \phi) = [r^2 - 2\cos(\phi)rs + s^2]^{1/2}$. Since allocations are assumed to be symmetric, we can write:

\begin{equation}
    z(r) = \int_0^S \psi(r, s)s \theta(s)n(s)ds,
\end{equation}

where

\begin{equation}
    \psi(r, s) = \delta \int_0^{2\pi} e^{-\delta x(r, s, \phi)}d\phi.
\end{equation}

Each worker is endowed with one unit of labor, which he supplies inelastically to the composite activity producing-and-commuting. The third aspect of the technology is a commuting cost that takes the form of a loss of labor time that depends on the distance traveled to and from work each day. Specifically, if a worker lives at location $s$ and works at location $r$, he delivers:

\begin{equation*}
    e^{-\kappa |r - s|}
\end{equation*}

hours of labor at location $r$.

Workers have identical preferences $U(c, \ell)$ over consumption of the produced good $c$ and residential land $\ell$. The function $U$ is CD:

\begin{equation}
    U(c, \ell) = c^{\beta}\ell^{1 - \beta}
\end{equation}

Let $c(r)$ and $\ell(r)$ denote the goods and land consumption of everyone housed at $r$. Every consumer-worker at every location must receive the reservation utility level:

\begin{equation}
    U(c(r), \ell(r)) = \overline{u}
\end{equation}

\chapter{Market thickness and the impact of unemployment on housing market outcomes}

\section{The Model}

\subsection{The basic setup}

In our model, the number of households in a city, denoted by $M$, is given. A household either lives in her own house or rents an apartment. A house cannot become an apartment. The total number of houses $^H$ in the city is fixed. All houses have the same quality but they differ in characteristics such as design, yard, etc. We use a unit circle to model the characteristics space of houses. Each point on the circle represents a unique characteristic. All households differ in their preferences regarding housing characteristics. They are uniformly distributed around the circle. A buyer's location on the circle means that any house at the location would be a perfect match for the buyer. The buyer's location is private information.

At the beginning of each period, sellers post advertisements and announce the characteristics of their house to the public. We assume each buyer can visit at most one house per period. It is thus optimal for each buyer to choose to visit the house that best matches her. A seller may have multiple visitors. We assume each seller can negotiate with at most one buyer per period; it is then optimal for the seller to choose to negotiate with the visitor who best matches the seller's house and hence shows the strongest interest in the house.

During each period, every hosuehold who lives in her own house may be hit by a shock. When hit by such a shock, the household may choose whether or not to move to a different house. If she moves, she will need to sell her current house and buy a new one. We assume that when a household moves, she will move out of her current house and rent a place to live during the transition if she cannot immediately find a new house to move into. This assumption allows us not to consider the situation in which a household continues living in her current house while it is for sale.

First, at the beginning of time $t$, the number of owner-occupied houses in the city $N_t^{H}$ plus the number of renters, is equal to the number of households in the city; namely $M = N_t^R + N_t^H$. Second, during time $t$, the number of houses for sale on the market $N_t^S$, reduced by the number of sales made $N_t^{sales}$, is equal to the number of houses left unsold at the end of this period: $U_t = N_t^S - N_t^{sales}$. Third, the number of houses for sale during this period is euqal to the number of unsold houses from the previous period $U_{t-1}$, plus those from homeowners who move out of their houses in this period $\mu_t N_t^H$, where $\mu_t$ is a homeowner's probability of moving in this period. Namely, $N_t^S = U_{t - 1} + \mu_t N_t^H$. Finally, the sum of the number of owner-occupied houses at the beginning of this period, $N_t^H$, and the number of unsold houses for sale from last period, $U_{t - 1}$, is equal to the total number of houses in the city; that is, $T^H = N_t^H + U_{t - 1}$.

Next, we introduce the unemployment rate, denoted $urate$, into the model. Both renters and home occupiers are assumed to have the same probability of being unemployed in each period. We assume that unemployed people are not in the market to buy houses because it is difficult for them to obtain mortgages. Therefore, the probability that a potential buyer will actually enter the market as a buyer, denoted as $\gamma$, cannot exceed the employment rate. Moreover, because of the financial constraint, only those households whose income is above a certain fraction of the expected house price will enter the housing market because they expect to be able to afford to buy a house.

\begin{equation}
    \gamma_t = (1 - urate) \times prob(y_t > (\tau_0 + \tau_1 urate) \times E(P_t) \mid employed).
\end{equation}

where $y_t$ is a household income, which is assumed to be i.i.d. drawn from a Pareto distribution conditional on being employed; that is, the cdf of income is $F(y) = 1 - (y / y_{min})^{-1/\sigma}$ if employed. $E(P_t)$ is the expected house price; $y> (\tau_0 + \tau_1 urate) \times E(P)$ reflects the financial constraint. A positive $\tau_1$ means the unemployment rate has an additional discouraging effect on the household's probability of entering the market as a buyer.

The total number of buyers in the market during time $t$, therefore, is $\gamma_t$ times the sum of those homeowners who move out of their current houses, $\mu_t N_t^H$, and those people who are currently renters, $N_t^R$; namely, $N_t^B = \gamma_t \mu_t N_t^H + \gamma_t N_t^R$.

\subsection{The seller's problem}

Next, we study the decision of sellers in the search-and-matching process. Suppose at time $t$, seller $i$ meets with buyer $j$ and they negotiate the sale price. The seller's value function is as follows:

\begin{equation}
    J_R^S
\end{equation}
% \begin{appendices}
% \input{./chapter/appendix_formula.tex}

% \includepdf[angle=90, pages=-]{./cheetsheet/probability.pdf}
% \end{appendices}

\backmatter

%%%%%%%%%%%%%%% Reference %%%%%%%%%%%%%%%

\printbibliography[heading=bibintoc]
\printindex

\end{document}


