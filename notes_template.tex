% !TEX root = ./notes_template.tex
%%%%%%%%%%%%%%%%%%%%%%%%%%%%%%%%%%%%%%%%%%%%%%%%%%
%%%%%%%%%%%%%%%%%%%%% preamble %%%%%%%%%%%%%%%%%%%
%%%%%%%%%%%%%%%%%%%%%%%%%%%%%%%%%%%%%%%%%%%%%%%%%%
\documentclass[11pt,twoside]{book}
\usepackage[mono=false]{libertine} % new linux font, ignore mono

\usepackage{luatex85}

%\renewcommand{\baselinestretch}{1.05}
\usepackage{amsmath,amsthm,amssymb,mathrsfs,amsfonts,dsfont}
\usepackage{epsfig,graphicx}
\usepackage{tabularx}
\usepackage{blkarray}
\usepackage{slashed}
\usepackage{scalerel}
\def\msquare{\mathord{\scalerel*{\Box}{gX}}}
\usepackage{color}
\usepackage{listings}
\usepackage{caption}
% \usepackage{fullpage}
\usepackage{lipsum} % provides dummy text for testing
\usepackage[toc,title,titletoc,header]{appendix}
\usepackage{minitoc}
\usepackage{color}
\usepackage{multicol} % two-col ToC
\usepackage{bm}
\usepackage{imakeidx} % before hyperref
\usepackage{hyperref}
\usepackage{mathtools}
\DeclarePairedDelimiter\ceil{\lceil}{\rceil}
\DeclarePairedDelimiter\floor{\lfloor}{\rfloor}

% link colors settings
\hypersetup{
    colorlinks=true,
    citecolor=magenta,
    linkcolor=blue,
    filecolor=green,      
    urlcolor=cyan,
    % hypertexnames=false,
}
\usepackage[capitalise]{cleveref}
\usepackage{subcaption}
\usepackage{enumitem}
\usepackage{mathtools}
\usepackage{pdfpages}
\usepackage{physics}
\usepackage[linesnumbered,ruled,vlined,algosection]{algorithm2e}
\SetCommentSty{textsf}
\usepackage{epigraph}
\epigraphwidth=1.0\linewidth
\epigraphrule=0pt

% adjust margin
\usepackage[margin=2.3cm]{geometry}
\headheight13.6pt

%%%%%%%%%%%%%%%% thmtools %%%%%%%%%%%%%%%%%%%%%
\usepackage{thmtools}
\declaretheorem[numberwithin=chapter]{theorem}
\declaretheorem[numberwithin=chapter]{axiom}
\declaretheorem[numberwithin=chapter]{lemma}
\declaretheorem[numberwithin=chapter]{proposition}
\declaretheorem[numberwithin=chapter]{claim}
\declaretheorem[numberwithin=chapter]{conjecture}
\declaretheorem[numberwithin=chapter]{assumption}
\declaretheorem[sibling=theorem]{corollary}
\declaretheorem[numberwithin=chapter, style=definition]{definition}
\declaretheorem[numberwithin=chapter, style=definition]{problem}
\declaretheorem[numberwithin=chapter, style=definition]{example}
\declaretheorem[numberwithin=chapter, style=definition]{exercise}
\declaretheorem[numberwithin=chapter, style=definition]{observation}
\declaretheorem[numberwithin=chapter, style=definition]{fact}
\declaretheorem[numberwithin=chapter, style=definition]{construction}
\declaretheorem[numberwithin=chapter, style=definition]{remark}
\declaretheorem[numberwithin=chapter, style=remark]{question}
%%%%%%%%%%%%%%%% thmtools %%%%%%%%%%%%%%%%%%%%%
\usepackage{changepage}
\newenvironment{solution}
    {\renewcommand\qedsymbol{$\square$}\color{blue}\begin{adjustwidth}{0em}{2em}\begin{proof}[\textit Solution.~]}
    {\end{proof}\end{adjustwidth}}

%%%%%%%%%%%%%%%% index %%%%%%%%%%%%%%%%%%%%%
\begin{filecontents}{index.ist}
% https://tex.stackexchange.com/questions/65247/index-with-an-initial-letter-of-the-group
headings_flag 1
heading_prefix "{\\centering\\large \\textbf{"
heading_suffix "}}\\nopagebreak\n"
delim_0 "\\nobreak\\dotfill"
\end{filecontents}
\newcommand{\myindex}[1]{\index{#1} \emph{#1}}
\makeindex[columns=3, intoc, title=Alphabetical Index, options= -s index.ist]
%%%%%%%%%%%%%%%% index %%%%%%%%%%%%%%%%%%%%%

%%%%%%%%%%%%%%%% ToC %%%%%%%%%%%%%%%%%%%%%
% Link Chapter title to ToC: https://tex.stackexchange.com/questions/32495/linking-the-section-text-to-the-toc
\usepackage[explicit]{titlesec}
\titleformat{\chapter}[display]
  {\normalfont\huge\bfseries}{\chaptertitlename\ {\thechapter}}{20pt}{\hyperlink{chap-\thechapter}{\Huge#1}
\addtocontents{toc}{\protect\hypertarget{chap-\thechapter}{}}}
\titleformat{name=\chapter,numberless}
  {\normalfont\huge\bfseries}{}{-20pt}{\Huge#1}

%%%%%%%%%%%%%%%%%%% fancyhdr %%%%%%%%%%%%%%%%%
\usepackage{fancyhdr}
\pagestyle{fancy} % enable fancy page style
\renewcommand{\headrulewidth}{0.0pt} % comment if you want the rule
\fancyhf{} % clear header and footer
\fancyhead[lo,le]{\leftmark}
\fancyhead[re,ro]{\rightmark}
\fancyfoot[CE,CO]{\hyperref[toc-contents]{\thepage}}

% https://tex.stackexchange.com/questions/550520/making-each-page-number-link-back-to-beginning-of-chapter-or-section
\makeatletter
\def\chaptermark#1{\markboth{\protect\hyper@linkstart{link}{\@currentHref}{Chapter \thechapter ~ #1}\protect\hyper@linkend}{}}
\def\sectionmark#1{\markright{\protect\hyper@linkstart{link}{\@currentHref}{\thesection ~ #1}\protect\hyper@linkend}}
\makeatother
%%%%%%%%%%%%%%%%%%% fancyhdr %%%%%%%%%%%%%%%%%


%%%%%%%%%%%%%%%%%%% biblatex %%%%%%%%%%%%%%%%%
\usepackage[doi=false,url=false,isbn=false,style=alphabetic,backend=biber,backref=true]{biblatex}
\addbibresource{bib.bib}

\newbibmacro{string+doiurlisbn}[1]{%
  \iffieldundef{doi}{%
    \iffieldundef{url}{%
      \iffieldundef{isbn}{%
        \iffieldundef{issn}{%
          #1%
        }{%
          \href{http://books.google.com/books?vid=ISSN\thefield{issn}}{#1}%
        }%
      }{%
        \href{http://books.google.com/books?vid=ISBN\thefield{isbn}}{#1}%
      }%
    }{%
      \href{\thefield{url}}{#1}%
    }%
  }{%
    \href{http://dx.doi.org/\thefield{doi}}{#1}%
  }%
}

% https://tex.stackexchange.com/questions/94089/remove-quotes-from-inbook-reference-title-with-biblatex
\DeclareFieldFormat[article,incollection,inproceedings,book,misc]{title}{\usebibmacro{string+doiurlisbn}{\mkbibemph{#1}}}
% https://tex.stackexchange.com/questions/454672/biblatex-journal-name-non-italic
\DeclareFieldFormat{journaltitle}{#1\isdot}
\DeclareFieldFormat{booktitle}{#1\isdot}
% https://tex.stackexchange.com/questions/10682/suppress-in-biblatex
\renewbibmacro{in:}{}
% add video field: https://tex.stackexchange.com/questions/111846/biblatex-2-custom-fields-only-one-is-working
\DeclareSourcemap{
    \maps[datatype=bibtex]{
      \map{
        \step[fieldsource=video]
        \step[fieldset=usera,origfieldval]
    }
  }
}
\DeclareFieldFormat{usera}{\href{#1}{\textsc{Online video}}}
\AtEveryBibitem{
    \csappto{blx@bbx@\thefield{entrytype}}{% put at end of entry
        \iffieldundef{usera}{}{\space \printfield{usera}}
    }
}
%%%%%%%%%%%%%%%%%%% biblatex %%%%%%%%%%%%%%%%%

%%%%%%%%%%%%%%%%%%%%% glossaries %%%%%%%%%%%%%%%%%
% !TEX root = ./notes_template.tex
% \usepackage[style=super]{glossaries}
% https://www.overleaf.com/learn/latex/Glossaries
\usepackage[style=super,toc,acronym]{glossaries}
\setlength{\glsdescwidth}{1\linewidth}
\makeglossaries

\renewcommand\glossaryname{List of Abbreviations and Symbols}

\newglossaryentry{Q2}{name={$Q_2(f)$},
%sort=Q2,
description={Two-side (bounded) error quantum query complexity}}

\newglossaryentry{real_number}{name={$\mathbb{R}$},description={Real number}}

% \newglossaryentry{gcd}{name={gcd},description={greatest common divisor}}

\newacronym{gcd}{GCD}{Greatest Common Divisor}


\newglossaryentry{svm}{name={SVM},description={Support Vector Machine}}

\newglossaryentry{gd}{name={GD},description={Gradient Descent}}

\newglossaryentry{qft}{name={QFT},description={Quantum Field Theory}}

\newglossaryentry{qm}{name={QM},description={Quantum Mechanics}}

\newglossaryentry{v}{name={$\vec{v}$},description={a vector}}

% physics
\newglossaryentry{hamiltonian}{name={$\hat{H}$},description={Hamiltonian}}

\newglossaryentry{lagrangian}{name={$L$},description={Lagrangian}}
%%%%%%%%%%%%%%%%%%%%% glossaries %%%%%%%%%%%%%%%%%

%%%%%%%%%%%%%%%%%%%%% glossaries-extra %%%%%%%%%%%%%%%%%
% \usepackage[record,abbreviations,symbols,stylemods={list,tree,mcols}]{glossaries-extra}
%%%%%%%%%%%%%%%%%%%%% glossaries-extra %%%%%%%%%%%%%%%%%


% !TEX root = ./notes_template.tex

%%%%%%%%%%%%%%%%%%%%%%%%%%%%%%%%%%%%
%%%%%%%%%%%%%%%%%%%%%%%%%%%%%%%%%%%%
% math
\let\iff\relax
\newcommand{\iff}{\text{ iff }}
\newcommand{\OPT}{\textup{OPT}}

% physics
\newcommand{\acreation}{a^\dagger}



\let\biconditional\leftrightarrow
%%%%%%%%%%%%%%%%%%%%%%%%%%%%%%%%%%%%%%%%%%%%%%%%%%
%%%%%%%%%%%%%%%% begin of document %%%%%%%%%%%%%%%
%%%%%%%%%%%%%%%%%%%%%%%%%%%%%%%%%%%%%%%%%%%%%%%%%%

\begin{document}

\title{\bf \huge PAPER NOTES}
\author{Xuyuan Zhang}
\date{Update on \today}
\maketitle
\setcounter{tocdepth}{2}
\setcounter{minitocdepth}{1} 

\begin{multicols}{2}
    \dominitoc% Initialization
    \adjustmtc[2]% chp number shift for mini-toc
    \tableofcontents
    \label{toc-contents}
\end{multicols}


	% \listoffigures
	% \listoftables
\begin{multicols}{2}
    \dominitoc% Initialization	
    \adjustmtc[2]% chp number shift for mini-toc
 \listoftheorems[ignoreall,show={theorem}]
\end{multicols}

\renewcommand{\listtheoremname}{List of Definitions}
\begin{multicols}{2}
	\listoftheorems[ignoreall,show={definition}]
\end{multicols}

	% \printglossaries
	% \printglossary[type=\acronymtype]
	\printglossary
	% \printglossary[title=List of terms, toctitle=List of terms]

	% bib2gls
	% \printunsrtglossaries % print all types
	% \printunsrtglossary[type={abbreviations},title=List of Abbreviations,style=listgroup]
	% \printunsrtglossary[type={abbreviations},title=List of Abbreviations,style=listhypergroup] % doesn't work
	% \printunsrtglossary[type={symbols},title=List of Symbols,style=listgroup]
	% \printunsrtglossary % main entry

%%%%%%%%%%%%%%%Content%%%%%%%%%%%%%%%
% \mainmatter % separat the number of toc and mainmatter
\part{urban economics}

% !TEX root = ../notes_template.tex
\chapter{}

\section{Some Theory}

% \includepdf[pages=-]{homework/IOE516_HW1_Solutions.pdf}

\chapter{THE MAKING OF THE MODERN METROPOLIS: EVIDENCE FROM LONDON}

\section{Theoretical Framework}

We consider a city embedded in a wider economy (Great Britain). The economy as a whole consists of a discrete set of locations $\mathbb{M}$. Greater London is a subset of these locations $\mathbb{N} \subset \mathbb{M}$, Time is discrete and is indexed by $t$. The economy as a whole is populated by an exogenous continuous measure $L_{\mathbb{M}t}$ of workers, who are geographically mobile and wndowed with one unit of labor that is supplied inelastically. Workers simultaneously choose their preferred residence $n$ and workplace $i$ given their idiosyncratic draws. We denote the endogenous measure of workers who choose a residence-workplace pair in Greater London by $L_{\mathbb{N}t}$. We allow locations to differ from one another in terms of their attractiveness for production and residence, as determined by productivity, amenities, the supply of floor space, and transport connections, where each of these location characteristics can evolve over time.

\subsection{Preferences}

We assume that preferences take the CD-form, such that the indirect utility for a worker $\omega$ residing in $n$ and working in $i$ is:

\begin{equation}
    U_{ni}(\omega) = \frac{B_{ni}b_{ni}(\omega)w_i}{\kappa_{ni}P_n^{\alpha} Q_n^{1 - \alpha}}, 0 < \alpha < 1
\end{equation}

where we suppress the time subscript from now on; $P_n$ is the price index for consumption goods, which may include both tradeable and nontradeable consumption goods; $Q_n$ is the price of residential floor space; $w_i$ is the wage, $\kappa_{ni}$ is an iceberg commuting cost; $B_{ni}$ captures amenities from the bilateral commute from residence $n$ to workplace $i$ that are common across all workers; and $b_{ni}(\omega)$ is an idiosyncratic amenity draw that captures all the idiosyncratic factors that can cause an individual to live and work in particular locations in the city.

We assume that idiosyncratic amenities $(b_{ni}(\omega))$ are drawn from an independent extreme value (Frechet) distribution for each residence-workplace pair and each worker:

\begin{equation}
    G(b) = e^{-b^{-\varepsilon}}, \varepsilon > 1 \label{eq:frechet}
\end{equation}

where we normalize the Frechet scale parameter in Equation \eqref{eq:frechet} to $1$ because it enters worker choice probabilities isomorphically to common bilateral amenities $B_{ni}$. The Frechet shape parameter $\varepsilon$ regulates the dispersion of idiosyncratic amenities, which controls the sensitivity of worker location decisions to economic variables. The smaller the shape parameter $\varepsilon$, the greater the heterogeneity in idiosyncratic amenities, and the less sensitive are worker location decisions to economic variables.

We decompose the bilateral common amenities parameter $(B_{ni})$ into a residence component common across all workplaces $(B_n^{\mathcal{R}})$, a workplace component common across all residences $(B_i^L)$, and an idiosyncratic component $(B_{ni}^I)$ specific to an individual residence-workplace pair:

\begin{equation}
    B_{ni} = B_n^{\mathcal{R}}B_i^L B_{ni}^I, \quad B_n^{\mathcal{R}}, B_i^L, B_{ni}^I > 0
\end{equation}

We allow the levels of $B_n^{\mathcal{R}}, B_i^I$ and $B_{ni}^I$ to differ across residences $n$ and workplace $i$, although when we examine the impact of the construction of railway network, we assume that $B_i^L$ and $B_{ni}^I$ are time-invariant. In contrast, we allow $B_n^{\mathcal{R}}$ to change over time, and for those changes to be potentially endogenous to the evolution of the surrounding concentration of economic activity through agglomeration forces.

Conditional on choosing a residence-workplace pair in Greater London, we know that the probability a worker chooses to reside in location $n \in \mathbb{N}$ and work in location $i \in \mathbb{N}$ is given by:

\begin{equation}
    \begin{aligned}
        \lambda_{ni} & = \frac{L_{ni}}{L_{\mathbb{M}}} \frac{L_{\mathbb{M}}}{L_{\mathbb{N}}} = \frac{L_{ni}}{L_{\mathbb{N}}} \\
        & = \frac{(B_{ni} w_i)^{\varepsilon} (\kappa_{ni} P_n^{\alpha} Q_n^{1 - \alpha})^{-\varepsilon}}{\sum_{k \in \mathbb{N}} \sum_{\ell \in \mathbb{N}} (B_{k\ell}w_{\ell})^{\varepsilon} (\kappa_{k\ell} P_{k}^{\alpha} Q_k^{1 - \alpha})^{-\varepsilon} }, n,  i \in \mathbb{N}
    \end{aligned}
\end{equation}

where $L_{ni}$ is the measure of commuters from $n$ to $i$.

The probability of commuting between residence $n$ and workplace $i$ depends on the characteristics of that residence $n$, the attributes of that workplace $i$ and bilateral commuting costs and amenities. Summing across workplaces $i \in \mathbb{N}$, we obtain the probability that a worker lives in residence $n \in \mathbb{N}$, conditional on choosing a residence-workplace pair in Greater London $(\lambda_n^R = \frac{R_n}{L_{\mathbb{N}}})$. Similarly, summing across residences $n \in \mathbb{N}$, we obtain the probability that a worker is employed in workplace $i \in \mathbb{N}$, conditional on choosing a residence-workplace pair in Greater London $(\lambda_i^L = \frac{L_i}{L_{\mathbb{N}}})$

\begin{equation}
    \begin{aligned}
        \lambda_n^R & = \frac{\sum_{i \in \mathbb{N}} (B_{ni} w_i)^\varepsilon (\kappa_{ni} P_n^{\alpha} Q_n^{1 - \alpha})^{-\varepsilon}}{\sum_{k \in \mathbb{N}} \sum_{\ell \in \mathbb{N}} (B_{k\ell} w_{\ell})^{\varepsilon} (\kappa_{k\ell} P_k^{\alpha} Q_k^{1 - \alpha})^{-\varepsilon}     } \\
        \lambda_i^L & = \frac{\sum_{n \in \mathbb{N}} (B_{ni} w_i)^{\varepsilon} (\kappa_{ni} P_n^{\alpha} Q_n^{1 - \alpha})^{-\varepsilon}}{\sum_{k \in \mathbb{N}} \sum_{\ell \in \mathbb{N}} (B_{k\ell} w_{\ell})^{\varepsilon} (\kappa_{k\ell} P_k^{\alpha} Q_k^{1 - \alpha})^{-\varepsilon}}
    \end{aligned}
\end{equation}

where $R_n$ denotes employment by residence in location $n$ and $L_i$ denotes employment by workplace in location $i$. A second implication of our extreme value specification is that expected utility conditional on choosing a residence workplace pair $(\overline{U})$ is the same across all residence-workplace pairs in the economy:

\begin{equation}
    \overline{U} = v\left[ \sum_{k \in \mathbb{M}} \sum_{\ell \in \mathbb{M}} (B_{k\ell}w_{k\ell})^{\varepsilon} (\kappa_{k\ell} P_k^{\alpha} Q_k^{1 - \alpha})^{-\varepsilon} \right]^{\frac{1}{\varepsilon}} 
\end{equation}

where the expectation is taken over the distribution for idiosyncratic amenities; $v \equiv \Gamma(\frac{\varepsilon - 1}{\varepsilon})$; $\Gamma(\cdot)$ is the gamma function. Using the probability that a worker chooses a residence-workplace pair in Greater London $(\frac{L_{\mathbb{N}}}{L_{\mathbb{M}}})$, we can rewrite this probability mobility condition as:

\begin{equation}
    \overline{U}(\frac{L_{\mathbb{N}}}{L_{\mathbb{M}}})^{\frac{1}{\varepsilon}} = v\left[ \sum_{k \in \mathbb{N}} \sum_{\ell \in \mathbb{N}} (B_{k\ell} w_{k\ell})^{\varepsilon} (\kappa_{k\ell} P_k^{\alpha} Q_k^{1 - \alpha})^{-\varepsilon} \right]^{\frac{1}{\varepsilon}}
\end{equation}

where only the limits of the summations differ on the right hand sides of the equations.

Intuitively, for a given common level of expected utility in the economy $(\overline{U})$, locations in Greater London must offer higher real wages adjusted for common amenities $(B_{ni})$ and commuting costs $(\kappa_{ni})$ to attract workers with lower idiosyncratic draws with an elasticity determined by the parameter $\varepsilon$.

\subsection{Production}

We assume that consumption goods are produced according to a Cobb-Douglas technology using labor, machinery capital, and commercial floor space, where commercial floor space includes both building capital and land. Cost minimization and zero profits imply that payments for labor, commercial floor space, and machinery are constant shares of revenue ($X_i$):

\begin{equation}
    w_i L_i = \beta^L X_i, q_i H_i^L = \beta^H X_i, rM_i = \beta^M X_i, \beta^L + \beta^H + \beta^M = 1
\end{equation}

where $q_i$ is the price of commercial floor space; $H_i^L$ is commercial floor space use; $M_i$ is machinery use; and machinery is assumed to be perfectly mobile across locations with a common price $r$ determined in the wider economy. We allow the price of commercial floor space $(q_i)$ to potentially depart from the price of residential floor space $(Q_i)$ in each location $i$ through a location-specific wedge $(\xi_i)$:

\begin{equation}
    q_i = \xi_i Q_i.
\end{equation}

From the relationship between factor payments and revenue in equation, payments for commercial floor space are proportional to workplace income $(w_i L_i)$:

\begin{equation}
    q_i H_i^L = \frac{\beta^H}{\beta^L} w_i L_i
\end{equation}

\subsection{Commuter Market Clearing}

commuter market clearing implies that the measure of workers employed in each location $(L_i)$ equals the measure of workers chooosing to commute to that location:

\begin{equation}
    L_i = \sum_{n \in \mathbb{N}} \lambda_{ni \mid n}^R R_n
\end{equation}

where $\lambda_{ni \mid n}^R$ is the probability of commuting to workplace $i$ conditional on living in residence $n$:

\begin{equation}
    \lambda_{ni \mid n}^R = \frac{ \left(\frac{B_{ni} w_i}{\kappa_{ni}}\right)^{\varepsilon}}{\sum_{\ell \in \mathbb{N}} \left(\frac{B_{n\ell} w_{\ell}}{\kappa_{n\ell}}\right)^{\varepsilon}}
\end{equation}

where all characteristics of residence $n$ have canceled from the above equation because they do not vary across workplaces for a given residence.

Commuter market clearing also implies that per capita income by residence $(v_n)$ is a weighted average of the wages in all locations, where the weights are these conditional commuting probabilities by residences $(\lambda_{ni \mid n}^R)$:

\begin{equation}
    v_n = \sum_{i \in \mathbb{N}} \lambda_{ni \mid n}^R w_i.
\end{equation}

\subsection{Land Market Clearing}

We assume that floor space is owned by landlords, who receive payments from the residential and commercial use of floor space and consume only consumption goods. Land market clearing implies that total income from the ownership of floor space equals the sum of payments for residential and commercial floor space use:

\begin{equation}
    \mathbb{Q}_n = Q_n H_n^R q_n H_n^L = (1 - \alpha) \left[ \sum_{i \in \mathbb{N}} \lambda_{ni \mid n}^R w_i \right] R_n + \frac{\beta^H}{\beta^L} w_n L_n
\end{equation}

where $H_n^R$ is residential floor space use; rateable values $(\mathbb{Q}_n)$ equals the sum of prices times quantities for residential floor space $(Q_n H_n^R)$ and commercial floor space $(q_n H_n^L)$; and we have used the expression for per capita income by residence $(v_n)$ from commuter market clearing.

From the combined land and commuter market-clearing condition, payments for residential floor space are a constant multiple of residence income $(v_n R_n)$, and payments for commercial floor space are a constant multiple of workplace income $(w_n L_n)$. Importantly, we allow the supplies of residential floor space $(H_n^R)$ and commercial floor space $(H_n^L)$ to be endogenous, and we allow the prices of residential and commercial floor space to potentially differ from one another through the location-specific wedge $\xi_i(q_i = \xi_i Q_i)$. In our baseline quantitative analysis below, we are not required to make assumptions about these supplies of residential and commercial floor space or this wedge between commercial and residential floor prices. The reason is that we condition on the observed rateable values in the data $(\mathbb{Q}_n)$ and the supplies and prices for residential and commercial floor space $(H_n^R, H_n^L, Q_n, q_n)$ only after the land market-clearing condition.

\section{Quantitative Analysis}

\subsection{Combined Land and Commuter Market Clearing}

We evaluate the effect of changes in the transport network by using an "exact hat algebra" approach. In particular, we rewrite our combined land and commuter market clearing condition for another year $\tau \neq t$ in terms of the values of variables in a baseline year $t$ and the relative changes of variables between years $\tau$ and $t$:

\begin{equation}
    \hat{\mathbb{Q}}_{nt} \mathbb{Q}_{nt} = (1 - \alpha) \hat{v}_{nt} v_{nt} \hat{R}_{nt} R_{nt} + \frac{\beta^H}{\beta^L} \hat{w}_{nt} w_{nt} \hat{L}_{nt} L_{nt}
\end{equation}

where $\hat{x}_{nt} = \frac{x_{n\tau}}{x_{nt}}$ for the variable $x_{nt}$ and we now make explicit the time subscripts. The relative change in employment $(\hat{L}_{it})$ and the relative change in average per capita income by residence $(\hat{v}_{nt})$ for year $\tau$ can be expressed as:

\begin{equation}
    \hat{L}_{it} L_{it} = \sum_{n \in \mathbb{N}} \frac{\lambda_{nit\mid n}^R \hat{w}_{it}^{\varepsilon} \hat{\kappa}_{nit}^{-\varepsilon}}{\sum_{\ell \in \mathbb{N}} \lambda_{n\ell t \mid n}^R \hat{w}_{\ell t}^{\varepsilon} \hat{\kappa}_{n\ell t}^{-\varepsilon}  } \hat{R}_{nt} R_{nt} 
\end{equation}

\begin{equation}
    \hat{v}_{nt} v_{nt} = \sum_{i \in \mathbb{N}} \frac{\lambda_{nit \mid n}^R \hat{w}_{it}}{\sum_{\ell \in \mathbb{N}} \lambda_{n\ell t \mid n}^R \hat{w}_{\ell t}^{\varepsilon} \hat{\kappa}_{n\ell t}^{-\varepsilon}  } \hat{w}_{it} w_{it}
\end{equation}

where these equations include terms in change in wages $(\hat{w}_{n})$ and commuting costs $(\hat{\kappa}_{ni})$ but not in amenities, because we assume that the workplace and bilateral components of amenities are constant $(\hat{B}_{it}^L = 1$ and $\hat{B}_{nit}^I = 1)$, and changes in the residential component of amenities $(\hat{B}_{nt}^R \neq 1)$ cancel from the numerator and denominator of the fractions.

substituting the expressions to the market clearing conditions for year $\tau$ we can get the result.

\begin{lemma}
    Suppose that $(\hat{\mathbb{Q}}_{nt}, \hat{R}_{nt}, L_{nit}, \lambda_{nit \mid n}^R, \mathbb{Q}_{nt}, v_{nt}, R_{nt}, w_{nt}, L_{nt})$ are known. Given known values for model parameters $\{\alpha, \beta^L, \beta^H, \varepsilon\}$ and the change in bilateral commuting costs $(\hat{\kappa}_{nit}^{-\varepsilon})$, the combined land and commuter market clearing condition determines a unique vector of relative changes in wages $(\hat{w}_{it})$ in each location.
\end{lemma}
% Now we start to lecture 12

\chapter{Urban Diversity, Process Innovation, and the Life Cycle of Products}

\section{The model}

There are $N$ cities in the economy, where $N$ is endogenous, and a continuum $L$ of infinitely lived workers, each of which has one of $m$ possible discrete aptitudes. There are equal proportion of workers with each aptitude in the economy, but their distribution across cities is endogenously determined through migration. Let us index cities by $i$ and worker aptitudes by subcript $j$ so that $l_i^j$ denotes the supply of labor with aptitude $j$ in city $i$. Time is discrete and indexed by $t$.

\subsection{Technology}

The ideal production process is firm specific and randomly drawn from a set of $m$ possible discrete processes, with equal probability for each. Each of the $m$ possible processes for each firm requires process specific intermediate inputs from a local sector employing workers of a specific aptitude. We say that two production processes for different firms are of the same type if they require intermediates produced using workers with the same aptitude.

A newly created firm  does not know its ideal production process, but it can find this by trying, one at a time, different processes in the production of prototypes. After producing a prototype with a certain process, the firm knows whether this process is its ideal one or not. Thus in order to switch from prototype to mass production a firm needs to have produced a prototype with its ideal process first, or to have tried all of its $m$ possible processes except one. Furthermore, we allow for the possibility that a firm decides to stop searching before learning its ideal process. Firms have an exogenous probability $\delta$ of closing down each period. Firms also lose one period of production whenever they relocate from one city to another. Thus, the cost of firm relocation increases with the exogenous probability of closure $\delta$.

THe intermediates specific to each type of process are produced by a monopolistically competitive intermediate sector, each such intermediate sector hires workers of aptitude $j$ and sells process-specific nontradable intermediate services to final-good firms using a process of type $j$. These differentiated services enter the production function of final good producers with the same constant elasticity of substitution $\frac{\varepsilon + 1}{\varepsilon}$. The production is:

\begin{equation}
  \overset{C}{?}_i^j(h) = Q_i^j \overset{x}{?}_i^j (h)
\end{equation}

\begin{equation}
  \text{ where } Q_i^j = (l_i^j)^{-\varepsilon} w_i^j, \varepsilon > 0 
\end{equation}

We distinguish variables corresponding to prototypes from those corresponding to mass-produced goods by an accent in the form of a question mark, ?. INdexing the differentiated varieties of goods by $h$, we denote ouput of prototype $h$ made with a process of type $j$ in city $i$ by $\overset{x}{?}_i^j(h)$. $Q_i^j$ is the unit cost for firms producing prototypes using a process of type $j$  in city $i$ and $w_i^j$ is the wage per unit of labor for the corresponding workers. Note that $Q_i^j$ decreases as $l_i^j$ increases: there are localization economies that reduce unit costs when there is a larger supply of labor with the relevant aptitude in the same city.

When a firm finds its ideal production process, it can engage in mass production at a fraction $\rho$ of the cost of producing a prototype, where $0 < \rho < 1$. Thus the cost function for a firm engaged in mass production is:

\begin{equation}
  C_i^j(h) = \rho Q_i^j x_i^j(h)
\end{equation}

where $x_i^j(h)$ denotes the ouput of mass produced good $h$, made with a process of type $j$, in city $i$.

With respect to the internal structure of cities, there are congestion costs in each city incurred in labor time and parameterized by $\tau (> 0)$. Labor supply, $l_i^j$, and production, $L_i^j$, with aptitude $j$ in city $i$ are related by the following expression:

\begin{equation}
  l_i^j = L_i^j(1 - \tau \sum_{j=1}^m L_i^j).
\end{equation}

THe expected wage income of a worker with aptitude $j$ in city $i$ is then $(1 - \tau \sum_{j=1}^m L_i^j)w_i^j$, where the higher land rents pair by those living closer to the city center are offset by lower commuting costs.

\subsection{Preferences}

Each period consumers allocate a fraction $\mu$ of their expenditure to prototypes and a fraction of $1 - \mu$ to mass-produced goods. The instantaneous indirect utility of a consumer in city $i$ is:

\begin{equation}
  V_i = \overset{P}{?}^{-\mu} P^{-(1 - \mu)}e_i^j 
\end{equation}

where $e_i$ denotes individual expenditure.

\begin{equation}
  \overset{P}{?} = \left\{ \sum_{j=1}^m \int \int [\overset{p}{?}_i^j (h)]^{1 - \sigma} dh di \right\}^{1/(1 - \sigma)} 
\end{equation}

\begin{equation}
  P = \left\{ \sum_{j=1}^m \int \int [p_i^j (h)]^{1 - \sigma} dh di \right\}^{1/(1 - \sigma)} 
\end{equation}

and the appropriate price indices of prototypes and mass-produced goods respectively, and $\overset{p}{?}_i^j(h)$ and $p_i^j(h)$ denote the price of individual varieties of prototypes and mass-produced goods respectively. All prototypes enter consumer preferences with the same elasticity of substitution $\sigma (> 2)$, and so do all mass produced goods.

\subsection{Income and Migration}

National income, $Y$, is the sum of expenditure and investment:

\begin{equation}
  Y = \sum_{j=1}^m \int L_i^j e_i^j di + \overset{P}{?}^{\mu} P^{1 - \mu} F \overset{n}{\circ}
\end{equation}

$L_i^j$ denotes population with aptitude $j$ in city $i$. Investment $\overset{P}{?}^{\mu} P^{1 - \mu}F\overset{n}{\circ}$ comes from aggregation of the start-up costs incurred by newly created firms. To come up with a new product, the firm must spend $F$ on market research, purchasing the same combination of goods bought by the representative consumer. Finally, $\overset{n}{\circ}$ denotes the total number of new firms.

\begin{definition}[Specialized City]
  A city is said to be fully specialzied if all its workers have the same aptitude, so that all local firms use the same type of production process.
\end{definition}

\begin{definition}[Diversified City]
 A city is said to be (fully) diversified if it has the same proportion of workers with each of the $m$ aptitudes, so that there are equal proportions of firms using each of the $m$ types of production process. 
\end{definition}

\subsection{City Formation}

Each potential site for a city is controlled by a different land development company or land developer, not all of which will be active in equilibrium. Developers have the ability to tax local land rents and to make transfers to local workers. When active, each land developer commits to a contract with any potential worker in its city that specifies the size of the city, whether it has a dominant sector and if so which, and any transfers.

\subsection{Equilibrium Definition}

Finally, a steady state equilibrium in this model is a configuration such that all of the following are true. Each developer offers a contract designed so as to maximize its profits. Each firm chooses a location/production strategy and prices so as to maximize its expected lifetime profits. All profit opportunities are exploited and then urban structure is constant over time.

\section{Equilibrium City Sizes}

\begin{lemma}[Output per Worker]
  In equilibrium, output per worker by firms using processes of type $j$ in city $i$ in a given period is:

  \begin{equation*}
    \frac{\overset{n}{?}_i^j \overset{x}{?}_i^j + \rho n_i^j x_i^j}{L_i^j} = (L_i^j)^{\varepsilon} (1 - \tau \sum_{j=1}^m L_i^j)^{\varepsilon + 1}.
  \end{equation*}
\end{lemma}


\chapter{Spatial Sorting and Inequality}

\section{Change in SKill Sourting: Framework}

\subsection{Setup}

On the production side, rather than modelling imperfect trade between locations, we consider an economy that is more stylized spatially, with two types of goods: (a) a homogeneous manufactured good that is freely traded across space and (b) housing, a local nontraded good.

\subsubsection{Preferences}

Consider a spatial equilibrium framework with two skill groups $\theta = U, S$, who choose where to live among locations $i \in [1, \ldots, N]$. Aggregate skill supply for each group, $L^{\theta}$, is exogenously given, and each worker supplies one unit of labor for wage $w_i^{\theta}$ in location $i$. The utility of worker $w$, who is type $\theta$ and lives in location $i$, is:

\begin{equation}
  u_i^{\theta}(w) = \max_{c, b} \log U^{\theta}(A_i, c, b) + \varepsilon_i^{\theta}(w), \text{ such that } c + r_i b = w_i^{\theta} 
\end{equation}

Here $\log U^{\theta}(\cdot)$ is the representative utility of a worker of type $\theta$; $c$ is the consumption of the freely traded good and is taken as the numeraire; $b$ denote housing, with price $r_i$ in location $i$; and $A_i$ is a vector of amenities in location $i$. Finally, $\varepsilon_i^{\theta}(w)$ is a worker-specific preference shock for living in location $i$.

First, we make assumption of CD type preferences over traded and nontraded goods. Second, we assume that amenities are separable from consumption. We allow amenities in location $i$ to be valued differently by the two groups, as caputured by a group-specific amenity level $A_i^{\theta}$. Third, preference shocks are typically chosen to be extreme value (EV) distributed. Papers in the tradition of urban and labor economics or industrial organization tend to use logit shocks, with normalized variance $\frac{\pi^2}{6}$ shifted by a factor $\frac{1}{\kappa^{\theta})}$, which together with CD utility lead to the following indirect utility of worker $\theta$ in location $i$:

\begin{equation*}
  v_i^{\theta}(w) = \log A_i^{\theta} + \log w_i^{\theta} - \alpha^{\theta} \log r_i + \frac{1}{\kappa^{\theta}} \varepsilon_i^{\theta}(w)
\end{equation*}

Equivalently, papers in the tradition of trade and economic geography typically choose Frechet shocks for $\varepsilon_i^{\theta}(w)$ with scale parameter $\kappa^{\theta} > 1$ that enter utility in a multiplicatively separable way. In that case, the indirect utility of worker $\theta$ in location $i$ is:

\begin{equation*}
  v_i^{\theta}(w) = \frac{A_i^{\theta} w_i^{\theta}}{r_i^{\alpha^{\theta}}} \varepsilon_i^{\theta}(w).
\end{equation*}

In either case, location choices in group $\theta$ can be summarized with $\lambda_i^{\theta}$, the share of $\theta$ workers who choose location $i$:

\begin{equation}
  \lambda_i^{\theta} = \frac{ (\frac{A_i^{\theta} w_i^{\theta}}{r_i^{\alpha^{\theta}}})^{\kappa^{\theta}}}{ \sum_{j=1}^N ( \frac{A_j^{\theta} w_j^{\theta}}{r_j^{\alpha^{\theta}}})^{\kappa^{\theta}} }
\end{equation}

The parameter $\kappa^{\theta}$ captures the elasticity of population shares with respect to amenity-adjusted real wages and is therefore a measure of mobility of group $\theta$, which we allow to be group specific. Expected utility for a worker in group $\theta$ across locations is:

\begin{equation}
  W^{\theta} = \delta^{\theta} \left[ \sum_{k=1}^N (\frac{A_i^{\theta} w_i^{\theta}}{r_i^{\alpha^{\theta}}})^{\kappa^{\theta}} \right]^{\frac{1}{\kappa^{\theta}}}
\end{equation}

where $\delta^{\theta} = \Gamma(\frac{\kappa^{\theta} - 1}{\kappa^{\theta}})$ and $\Gamma(\cdot)$ is the gamma function in the Frechet case.

\subsubsection{Supply of goods, amenities, and housing}

We first write down the labor demand side of the economy. In location $i$, output is produced by perfectly competitive firms. They combine skilled and unskilled labor, who are imperfect substitutes in production:

\begin{equation}
  Y_i = \left[ (z_i^U)^{\frac{1}{\rho}} (L_i^U)^{\frac{\rho - 1}{\rho}} + (z_i^S)^{\frac{1}{\rho}} (L_i^S)^{\frac{\rho - 1}{\rho}} \right]^{\frac{\rho}{\rho - 1}}
\end{equation}

In the CES production function, $\rho \geq 1$ is the elasticity of substitution between skills and $z_i^{\theta}$ are location- and skill specific productivity shifters. The shifters can be in part exogenous and in part endogenous, reflecting externalities. We assume that, for $\theta = \{U, S\}$ and $\forall i$,

\begin{equation}
  z_i^{\theta} = z^{\theta}(\overline{Z}_i, L_i^U, L_i^S) 
\end{equation}

where $\overline{Z}_i$ is the exogenous productivity component in city $i$. Local productivity spillovers are allowed here to depend not just on city size or density but also on its composition $(L_i^U, L_i^S)$. Given equation, relative labor demand in location $i$ is:

\begin{equation}
  \log (\frac{L_i^S}{L_i^U}) = \log (\frac{z_i^S}{z_i^U}) - \rho \log(\frac{w_i^S}{w_i^U}) 
\end{equation}

Furthermore, competition across cities ensures that the unit cost of production in all cities is $1$, the common price of the freely traded good:

\begin{equation*}
  \sum_{\theta} z_i^{\theta} (w_i^{\theta})^{1 - \rho} = 1, \forall i
\end{equation*}

Similar to productivity, amenities $A_i^{\theta}$ are assumed to be driven by both exogenous differences, and endogenous differences between cities, that is,

\begin{equation}
  A_i^{\theta} = A^{\theta}(\overline{A}_i, L_i^U, L_i^S)
\end{equation}

where $A_i$ is the exogenous amenity component of city $i$. Endogenous amenities capture elements of quality of life that change when the size or composition of cities changes.



\chapter{Urban Accounting and Welfare}

\section{The Model}

\subsection{Technology}

Consider a model of a system of cities in an economy with $N_t$ workers. Goods are produced in $I$ monocentric circular cities. Cities have a local level of productivity. Production in city $i$ in period $t$ is given by:

\begin{equation*}
    Y_{it} = A_{it}K_{it}^{\theta}H_{it}^{1 - \theta}
\end{equation*}

where $A_{it}$ denotes city productivity, $K_{it}$ denotes total capital, and $H_{it}$ denotes total hours worked in the city. We denote the population size of city $i$ by $N_{it}$. The standard first-order conditions of this problem are:

\begin{equation}
    w_{it} = (1 - \theta)\frac{Y_{it}}{H_{it}} = (1 - \theta) \frac{y_{it}}{h_{it}} \text{ and } r_t = \theta \frac{Y_{it}}{K_{it}} = \theta \frac{y_{it}}{k_{it}}
\end{equation}

where lowercase letters denote per capita variables. Note that capital is freely mobile across locations so there is a national interest rate $r_t$. Mobility patterns will not be determined solely by the wage, $w_{it}$, so there may be equilibrium differences in wages across cities at any point in time. We can then write down the "efficiency wedge", which is identical to the level of productivity, $A_{it}$, as:

\begin{equation}
    A_{it} = \frac{Y_{it}}{K_{it}^{\theta}H_{it}^{1 - \theta}} = \frac{y_{it}}{k_{it}^{\theta} h_{it}^{1 - \theta}}
\end{equation}

\subsection{Preferences}

Agents order consumption and hour sequenecs according to the following utility function:

\begin{equation*}
    \sum_{t=0}^\infty \beta^t \left[ \log c_{it} + \psi \log(1 - h_{it}) + \gamma_i \right],
\end{equation*}

where $\gamma_i$ is a city-specific amenity and $\psi$ is a parameter that governs the relative preference for leisure. Each agent lives on one unit of land and commutes from his home to work. Commuting is costly in terms of goods. The problem of an agent in city $i_0$ with cpaital $k_0$ is therefore,

\begin{equation*}
    \max_{\{c_{i_t, t}, h_{i_t, t}, k_{i_t, t}, i_{t}\}_{t=0}^\infty} \sum_{t=0}^\infty \beta^t \left[ \log c_{it} + \psi \log (1 - h_{it}) + \gamma_i \right]
\end{equation*}

subject to the budget constraint:

\begin{equation*}
    \begin{aligned}
        c_{it} + x_{it} & = r_{t} k_{it} + w_{it} h_{it} (1 - \tau_{it}) - R_{it} - T_{it} \\
        k_{it + 1} & = (1 - \delta) k_{it} + x_{it}
    \end{aligned}
\end{equation*}

where $x_{it}$ is investment, $\tau_{it}$ is a labor tax or friction associated with the cost of building the commuting infrastructure, $R_{it}$ are land rents, and $T_{it}$ are commuting costs.

We assume that we are in steady state so $k_{it + 1} = k_{it}$ and $x_{it} = \delta k_{it}$. Furthermore, we assume $k_{it}$ is such that $r_t = \delta$. The simplified budget constraint of the agent becomes:

\begin{equation}
    c_{it} = w_{it} h_{it} (1 - \tau_{it}) - R_{it} - T_{it}.
\end{equation}

The first-order conditions of this problem imply $\psi \frac{c_{it}}{1 - h_{it}} = (1 - \tau_{it}) w_{it}$. Combining the expression with the first equation, we obtain:

\begin{equation}
    (1 - \tau_{it}) = \frac{\psi_{it}}{(1 - \theta)} \frac{c_{it}}{1 - h_{it}} \frac{h_{it}}{y_{it}}
\end{equation}

We refer to $\tau_{it}$ as the "labor wedge". Although $\tau_{it}$ is modeled as a labor tax, it should be interpreted more broadly as anything that distorts an agent's optimal labor supply decision. Agents can move freely across cities so utility in each period has to be determined by:

\begin{equation}
    \overline{u} = \log c_{it} + \psi \log(1 - h_{it}) + \gamma_i
\end{equation}

for all cities with $N_{it} > 0$, where $\overline{u}$ is the economy-wide per period utility of living in a city.

\subsection{Commuting Cost, Land Rents, and City Equilibrium}

Cities are monocentric, all production happens at the center, and people live in surrounding areas characterized by their distance to the center, $d$. Cities are surroundedby a vast amount of agricultural land that can be freely converted into urban land. We normalize the price of agricultural land to zero. Since land rents are continuous in equilibrium, this implies that at the boundary of a city, $\overline{d}_{it}$, land rents should be zero as well, namely, $R(\overline{d}_{id}) = 0$. Since, all agents in a city are identical, in equilibrium they must be indifferent between they live in a city, which implies that the total cost of rent plus commuting costs should be identical in all areas of a city. So,

\begin{equation*}
    R_{it}(d) + T(d) = T(\overline{d}_{it}) = \kappa \overline{d}_{it}, \forall d \in [0, \overline{d}_{it}]
\end{equation*}

since $T(d) = \kappa d$ where $\kappa$ denotes commuting costs per mile.

Everyone lives on one unit of land, $N_{it} = \overline{d}_{it}^2 \pi$, and so $\overline{d}_{it} = (N_{it} / \pi)^{\frac{1}{2}}$. Thus $R_{it} + T(d) = \kappa (N_{it} / \pi)^{\frac{1}{2}}$. This implies that $R_{it}(d) = \kappa(\overline{d}_{it} - d)$ and so total land rents in a city of size $N_{it}$ are given by $TR_{it} = \int_0^{\overline{d}_{it}} (\kappa (\overline{d}_{it} - d) d2\pi)dd = \frac{\kappa}{3} \pi^{-\frac{1}{2}} N_{it}^{\frac{3}{2}}$. Hence, arrange land rents are equal to $AR_{it} = \frac{2\kappa}{3} (\frac{N_{it}}{\pi})^{\frac{1}{2}}$. Taking logs and rearranging terms, we obtain that:

\begin{equation}
    \log (N_{it}) = o_1 + 2 \log AR_{it}.
\end{equation}

where $o_1$ is a constant. We can also compute the total miles traveled by commuters in the city, which is given by:

\begin{equation}
    TC_{it} = \int_0^{\overline{d}_{it}} (d^2 2\pi) dd = \frac{2}{3} \pi^{-\frac{1}{2}} N_{it}^{\frac{3}{2}}
\end{equation}

\subsection{Government Budget Constraint}

The government levies a labor tax, $\tau_{it}$, to pay for the transportation infrastructure. Let government expenditure be a function of total commuting costs and wages such that:

\begin{equation*}
    G(h_{it} w_{it}, TC_{it}) = g_{it} h_{it} w_{it} \kappa TC_{it} = g_{it} h_{it} w_{it} \kappa \frac{2}{3} \pi^{-\frac{1}{2}} N_{it}^{\frac{3}{2}}
\end{equation*}

where $g_{it}$ is a measure of government inefficiency. That is, the government requires $\kappa g_{it}$ workers per mile commuted to build and maintain urban infrastructure. The government budget constraint is then given by:

\begin{equation}
    \tau_{it} h_{it} N_{it} w_{it} = g_{it} h_{it} w_{it} \kappa \frac{2}{3} \pi^{-\frac{1}{2}} N_{it}^{\frac{3}{2}}
\end{equation}

which implies that the labor wedge can be written as:

\begin{equation}
    \tau_{it} = g_{it} \kappa \frac{2}{3} (\frac{N_{it}}{\pi})^{\frac{1}{2}}
\end{equation}

or

\begin{equation}
    \log \tau_{it} = o_2 + \frac{1}{2} \log N_{it} + \log g_{it}
\end{equation}

\subsection{Equilibrium}

The consumer budget constraint is given by:

\begin{equation*}
    c_{it} = w_{it} h_{it} (1 - \tau_{it}) - R_{it} - T_{it} = (1 - \theta) (1 - \tau_{it}) y_{it} - \kappa (\frac{N_{it}}{\pi})^{\frac{1}{2}}
\end{equation*}

To determine output we know that the proudction function is given by $y_{it} = A_{it} k_{it}^{\theta} h_{it}^{1 - \theta}$ and the decision of firms to rent capital implies that $r_t k_{it} = \theta y_{it}$. Hence,

\begin{equation*}
    y_{it} = A_{it} (\frac{\theta y_{it}}{r_t})^{\theta} h_{it}^{1 - \theta} = A_{it}^{\frac{1}{1 - \theta}} (\frac{\theta}{r_t})^{\frac{\theta}{1 - \theta}} h_{it}.
\end{equation*}

Using the above result, we can derive

\begin{equation*}
    h_{it} = \frac{1}{1 + \psi} (1 + \frac{\psi(R_{it} + T_{it})}{(1 - \theta)(1 - \tau_{it})} \frac{(\frac{r_t}{\theta})^{\frac{\theta}{1 - \theta}}}{A_{it}^{\frac{1}{1 - \theta}}})
\end{equation*}

and

\begin{equation*}
    c_{it} = \frac{1}{1 + \psi} \left[(1 - \theta)(1 - \tau_{it}) (\frac{\theta}{r_t})^{\frac{\theta}{1 - \theta}}A_{it}^{\frac{1}{1 - \theta}} - (R_{it} + T_{it})\right]
\end{equation*}

The free mobilility assumption in the result implies that $\overline{u}_t = \log c_{it} + \psi \log(1 - h_{it}) + \gamma_{it}$ for some $\overline{u}_{t}$ determined in general equilibrium so:

\begin{equation}
    \begin{aligned}
        \overline{u}_{it} & + (1 + \psi) \log(1 + \psi) - \psi \log \psi \\
        & = \log \left( (1 - \theta) (1 - \kappa g_{it} \frac{2}{3} (\frac{N_{it}}{\pi})^{\frac{1}{2}}\right) \frac{A_{it}^{\frac{1}{1 - \theta}}}{(r_t / \theta)^{\frac{\theta}{1 - \theta}}} - \kappa (\frac{N_{it}}{\pi})^{\frac{1}{2}} \right) \\
        & + \psi \log \left( 1 - \frac{\kappa (\frac{N_{it}}{\pi})^{\frac{1}{2}}}{(1 - \theta)(1 - \kappa g_{it} \frac{2}{3} (\frac{N_{it}}{\pi})^{\frac{1}{2}})} \frac{(\frac{r_t}{\theta})^{\frac{\theta}{1 - \theta}}}{A_{it}^{\frac{1}{1 - \theta}}} \right) + \gamma_{it}.
    \end{aligned}
\end{equation}

The last equation determines the size of the city $N_{it}$ as implicit function of city productivity $A_{it}$, city amenities, $\gamma_{i}$, government inefficiency, $g_{it}$, and economy wide variables like $r_t$ and $\overline{u}_{it}$. In the above euqation, the LHS is decreasing in $N_{it}$. THe LHS is also increasing in $A_{it}$ and $\gamma_i$ and decreasing in $g_{it}$. Hence, we can prove immediately that:

\begin{equation}
    \frac{\partial N_{it}}{\partial A_{it}} > 0, \frac{\partial N_{it}}{\partial \gamma_i} > 0, \frac{\partial N_{it}}{\partial g_{it}} < 0 \label{eq:decreasing_increasing_condition}
\end{equation}

The economy wide utility level $\overline{u}_t$ is determined by the labor market clearing conditions

\begin{equation}
    \sum_{i=1}^I N_{it} = N_t
\end{equation}

This last equation clarifies that our urban system is closed; we do not consider urban-rural migration.

\section{Evidence of Efficiency, Amenities, and Frictions}

\subsection{Empirical Approach}

We start by estimating the following equation:

\begin{equation}
    \log N_{it} = \alpha_1 + \beta_1 \log A_{it} + \varepsilon_{it}
\end{equation}

THe value of $\beta_1$ yields the effect of the "efficiency wedge" on city population. According to the model, $\beta_1 > 0$ by the Equation \eqref{eq:decreasing_increasing_condition}. Furthermore, $\log N_{it}(A_{it}) = \beta_1 \log A_{it}$ is the population size explained by the size of the "efficiency wedge". In contrast, $\varepsilon_{1it}$ is the part of the observed population in the city that is unrelated to the productivity; according to the model it is related to both $g_{it}$ and $\gamma_i$. We can then estimate the following equation: $\tilde{\varepsilon}_{1}(g_{it}, \gamma_{it}) \equiv \varepsilon_{1it}$.

Since the "efficiency wedge" increases population size, total commuting increases, which affects the "labor wedge". This is the standard urban trade-off between productivity and agglomeration. We can estimate the effect of producitvity on the labor wedge and the decomposition of $\log N_{it}$ into $\log \tilde{N}_{it}(A_{it})$ and $\varepsilon_{1it}$. Hence, we estiamte:

\begin{equation}
    \log \tau_{it} = \alpha_2 + \beta_2 \log \tilde{N}_{it}(A_{it}) + \varepsilon_{2it}
\end{equation}

We denote the effect of efficiency on distortions by $\log \tilde{\tau}_{it} = \beta_2 \log \tilde{N}_{it}(A_{it})$. We can then estimate the following equation: $\tilde{\varepsilon}_{2it} \equiv \varepsilon_{2it}$. The equation also implies that the error term $\varepsilon_{2it}$ is related to $g_{it}$ and to $\tilde{\varepsilon}_{1}(g_{it}, \gamma_{it})$. Hence, we define $\tilde{\varepsilon}_{2}(g_{it}, \tilde{\varepsilon}_{1}(g_{it}, \gamma_{it})) \equiv \varepsilon_{2it}$.

We now can decompose the effect from all three elements of $(A_{it}, \gamma_{it}, g_{it})$. To do so, we estimate:

\begin{equation}
    \log (AR_{it}) = \alpha_3 + \beta_3 \log \tilde{\tau}_{it} + \beta_4 + \varepsilon_{1it} + \beta_5 \varepsilon_{2it} + \varepsilon_{3it}
\end{equation}

using median rents for $AR_{it}$. The effect of $\gamma_{it}$ and $g_{it}$ are determined by the estimates of $\beta_4$ and $\beta_5$. Note that $\varepsilon_{1it}$ and $\varepsilon_{2it}$ depend on both $\gamma_{it}$ and $g_{it}$. However, since $\varepsilon_{2it} = \tilde{\varepsilon}_{2}(g_{it}, \tilde{\varepsilon}_{1}(g_{it}, \gamma_{it}))$ depend only on $\gamma_{it}$ through $\varepsilon_{1it}$ and we are including $\varepsilon_{1it}$ directly in the regression, $\beta_5$ capture only the effect of changes in $g_{it}$ on land rents.

Note that we can then use the equation to relate average rents and population sizes. So we can estiamte the model using 

\begin{equation}
    \log (N_{it}) = \alpha_4 + \beta_6 \log (AR_{it}) + \varepsilon_{4it}
\end{equation}

In a circular city $\beta_6 = 2 > 0$.

\subsection{Effects of Efficiency, Amenities, and Frictions on City Size}

We can decompose the labor wedge into taxes and other distortions such that

\begin{equation}
    (1 - \tau_{it}) = (1 - \tau_{it}^\prime) (\frac{1 - \tau_{ith}}{1 + \tau_{ith}})
\end{equation}

where $\tau_{it}$ is our measure of the labor wedge, $\tau_{ith}$ is the labor tax rate, $\tau_{itc}$ is the consumption tax rate, and $\tau_{it}^\prime$ are other distortions. Thus expect our measure of the total labor wedge, $(1 - \tau_{it})$, to be correlated with $(1 - \tau_{ith}) / (1 + \tau_{itc})$.
% \begin{appendices}
% \input{./chapter/appendix_formula.tex}

% \includepdf[angle=90, pages=-]{./cheetsheet/probability.pdf}
% \end{appendices}

\backmatter

%%%%%%%%%%%%%%% Reference %%%%%%%%%%%%%%%

\printbibliography[heading=bibintoc]
\printindex

\end{document}


