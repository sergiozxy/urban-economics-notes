\chapter{Growth, speculation and sprawl in a monocentric city}

\section{The Model}

A single composite commodity, $Q$, is produced by a competitive industry in the city in both periods. Using subscripts to designate time periods, the endogenous quantities produced are $Q_1$ and $Q_2$. Some $Q$ is produced for export and the rest for local consumption. Regardless of whether it is bound for export or local use, every unit of $Q$ must be transported to the central point at a cost of $t$ per unit of $Q$, per unit of $x$. It is sold there at the (endogenous) price $P_i, i = 1, 2$. The $Q$ production function has fixed factor proportions. It requires exactly $\lambda$ units of land, $\mu$ of labor and $v$ of capital to produce each unit of $Q$. Land rent and labor wages are endogenous to the model, but the cost of capital is exogenous; it is $s$ per unit throughout the city in both periods.

Each provides a single unit of labor to the $Q$ industry per period, receives (endogenous) wages of $w$ per period, and has the same utility function. Because the city is open, migration will assure that every resident household attains the (exogenous) level of utility attained elsewhere in the naional economy. Arguments of the utility function are $Q$, residential land, and another composite commodity, $Z$, that is imported from outside the city and sold at the central point at unitary price in both periods. Nonland components of housing services are absorbed in $Q$ and $Z$. Households must consume exactly $q$ units of $Q, z$ of $Z$, and $1$ unit of residential land in each period. To ensure that local consumption of $Q$ does not exhaust production, we assume that:

\begin{equation*}
    1 / \mu > q
\end{equation*}

Households value residential proximity to the central point because workers must commute to work by passing through it and shoppers must shop at it. The combined cost of these trips  is $T$ per household, per unit of $x$, per period.

The exogenous growth mechanism in the model is an increase in export demand for $Q$ between two periods. The export demand function is $f(P_i, \Gamma_i)$, where $\Gamma_i \in 1, 2$, is an exogenous demand-shift variable that encodes all the information required to predict the usual price-quantity demand relation. Further,

\begin{equation*}
    \frac{\partial f}{\partial \Gamma_i} > 0, \frac{\partial f}{\partial P_i} < 0, i = 1, 2 \text{ and } \Gamma_1 < \Gamma_2
\end{equation*}

No other exogenous variables change between periods.

\section{Perfect Foresight Planning}

Suppose landowners have perfect foresight and know at the outset what their land rent will be in both periods for all possible development strategies. This requires that $\Gamma_2$ be known with certainty in period $1$, and means that all development decisions - whether executed in the first or second period - are made simultanously in period $1$. We call this perfect-foresight planning. It is to be distinguished from speculation, examined later, which occurs when landowners are uncertain about $\Gamma_2$ and future land rents.

In first-period equilibrium, households will reside in a Von-Thunen ring outside another where $Q$ is produced if only:

\begin{equation*}
    t / \lambda > T
\end{equation*}

Similarly, second-hand residential development will lie more distant from the central point than second-period industrial development. Depending on the relative magnitudes of $t / \lambda$ and $T$, it may be advantageous for landowners to withhold from development in period $1$ a ring of land between the industrial and residential zones, and preserve it for second-period industrial use.

In particular, it will always occur if:

\begin{equation*}
    t / \lambda > (2 + r) T,
\end{equation*}

where $r$ is the (universal) discount rate between periods. Since $-t/\lambda$ and $-T$ are shown below to be the slope of industrial and residential bid-rent functions, this condition requires that industrial bid-rents decrease more than twice as rapidly with $x$ as residential bid-rents.

\subsection{First-Period Equations}

equilibrium in period $1$ is characterized by eight endogenous variables. Three of them are $Q_1$, $P_1$ and $w_1$. Two other indicate land rents in the residential and industrial zones. The remaining three are spatial boundaries: $x_a$, the outer edge of the industrial zone; and $x_b$ and $x_c$, the inner and outer boundary of the residential zone. From previous assumptions, these will satisfy:

\begin{equation*}
    0 \leq x_a \leq x_b \leq x_c
\end{equation*}

To identify the industrial land rent variable, note that firms in the competitive $Q$ industry must earn zero profit. Thus land rent and transportation charges must exhaust revenues after payments are made to capital and labor. Since the latter costs are the same for every firm, on a per-unit-of-$Q$ basis, land rent and transportation charges per-unit-of-$Q$ must be the same for every firm. And since each unit of $Q$ requires exactly $\lambda$ units of land, this sum is the same for every firm on a per-unit-of-land basis as well. In period $1$, we call this amount $R_1^Q$ per unit of land. Because total transportation charges per-unit-of-land are $tx / \lambda$ at $x$, the rent on industrial land is indicated by the linear function $R_1^Q - tx / \lambda$.

To  identify the residential land rent variable, note that each household has the same income, consumes the same consumption bundle, and faces the same prices for $Q$ and $Z$. Their expenditure on land and transportation charges must therefore be the same at every location. In period $1$, we call this amount $R_1^H$. Because household transportation charges are $Tx$ at $x$, the rent on residential land is indicated by the linear function $R_1^H - Tx$.

First-period equilibrium is characterized by eight conditions. One of them equates the supply and demand for $Q_1$. This means local production of $Q$ must equal the sum of quantities demanded for export and local consumption. Where $Q_1$ units are produced, $\mu Q_1$ households are required to supply the corresponding amount of labor. This means local consumption of $Q_1$ must be $\mu qQ_1$. $Q$-market equilibrium requires that:

\begin{equation}
    Q_1 (1 - \mu q) = f(P_1, \Gamma_1)
\end{equation}

A second condition is that household budgets balance:

\begin{equation}
    w_1 = R_1^H + P_1 q + z.
\end{equation}

A third is that competitive Q-firms earn zero profit:

\begin{equation}
    P_1 = \mu w_1 + vs + \lambda R_1^Q
\end{equation}

Each unit of $Q$ requires $\lambda$ unit of land, so $Q_1$ is related to $x_a$ by:

\begin{equation}
    Q_1 = \pi x_a^2 / \lambda
\end{equation}

The last three conditions concern equilibrium in the land market. At $x_a$ industrial land rent must be zero, since otherwise land beyond it would be offered for first-period industiral development or land inside it withheld from development until later. This requires that:

\begin{equation}
    R_1^Q - tx_a / \lambda.
\end{equation}

For similar reasons, residential land rent at $x_c$ must be zero:

\begin{equation}
    R_1^H - Tx_c = 0.
\end{equation}

The equilibrium condition concerning $x_b$ is less easily stated. Owners of land near $x_a$ must decide only whether to develop industrially in the first or second period. Those near $x_c$ make a similar decision for residential development. The present value of the first option is:

\begin{equation*}
    R_1^H - Tx + \frac{1}{1 + r}(R_2^H - Tx)
\end{equation*}

The present value of the second option is:

\begin{equation*}
    \frac{1}{1 + r}(R_2^Q - tx / \lambda)
\end{equation*}

$x_b$ is the location where these strategies are equally profitable:

\begin{equation*}
    R_1^H - Tx_b = \frac{1}{1 + r}[(R_2^Q - tx_b / \lambda) - (R_2^H - Tx_b)]
\end{equation*}

Those eight equations are not a closed system since the last equation includes the second period, endogenous variables, $R_2^H, R_2^Q$. To solve this equalibrium, it is necessary to solve both periods' equations simultanously, as in a two-period dynamic problem.

\subsection{Second-Period Equations}

Equations in period $2$ is characterized by six variables, five of which are $Q_2, P_2, w_2, R_2^Q, R_2^H$. The other one is $x_d$, the outer boundary of second-period, residential expansion. Of course,

\begin{equation*}
    x_d \geq x_c
\end{equation*}

The second period conditions corresponding to (1) - (3) are:

\begin{equation}
    \begin{aligned}
        Q_2(1 - \mu q) & = f(P_2, \Gamma_2) \\
        w_2 & = R_2^H + P_2 q + z \\
        P_2 & = \mu w_2 + vs + \lambda R_2^Q
    \end{aligned}
\end{equation}

Since $\Gamma_2 > \Gamma_1$, $Q$-production will be greater in period $2$ and the industrial zone will expand from $x_a$ to $x_b$. Thus, $Q_2$ is related to $x_b$ by:

\begin{equation}
    Q_2 = \pi x_b^2 / \lambda
\end{equation}

The increase in production requires an expansion in the residential zone from $x_c$ to $x_d$. In order that this expansion allow a total of $Q_2$ households to reside in the city:

\begin{equation}
    Q_2 = \pi(x_d^2 - x_b^2) / \mu
\end{equation}

The final condition is that land rent at $x_d$ is zero:

\begin{equation}
    R_2^H - Tx_d = 0
\end{equation}

condition (1) - (14) is a closed system of fourteen, independent conditions in fourteen endogenous variables.

\subsection{The Equilibrium}

The equilibrium land rent function in the first period is:

\begin{equation*}
    \begin{aligned}
        ER_1(x) & = R_1^Q - tx / \lambda \text{ for } x \in [0, x_a] \\
        & = R_1^H - Tx \text{ for } x \in [x_a, x_c] \\
        & = 0 \text{ for } x \in [x_c, \infty)
    \end{aligned}
\end{equation*}

It is continuous at every $x$ except $x_b$. Land just beyond $x_b$ earns a positive rent while land inside it earns none. The equilibrium land-rent in the second-period is:

\begin{equation*}
    \begin{aligned}
        ER_2(x) & = R_2^Q - tx / \lambda \text{ for } x \in [0, x_b] \\
        & = R_2^H - Tx \text{ for } x \in [x_b, x_d] \\
        & = 0 \text{ for } x \in [x_d, \infty)
    \end{aligned}
\end{equation*}

Note first that $ER_2(x) \geq ER_1(x), \forall x \geq 0$ and second that $ER_2(x)$ is also continuous at every $x$ but $x_b$. In the second period, however, land just beyond $x_b$ earns less than land just inside it. The function indicating equilibrium present value of both period's land rent,

\begin{equation*}
    PV(x) = ER_1(x) + \frac{1}{1 + r}ER_2(x)
\end{equation*}

is continuous at every $x$.

\subsection{Comparative Statics}

Consider now the effect on both periods' spatial equilibrium of a change in $\Gamma_2$, all other exogenous variables remaining the same. We have:

\begin{equation*}
    \frac{\partial x_a}{\partial \Gamma_2} < 0, \frac{\partial x_b}{\partial \Gamma_2} > 0, \frac{\partial x_c}{\partial \Gamma_2} > 0, \frac{\partial x_d}{\partial \Gamma_2} > 0
\end{equation*}

All boundaries move outward as $\Gamma_2$ increases except $x_a$, which moves inward. Since $x_b$ increases and $x_a$ decreases with $\Gamma_2$, the second period industrial zone is larger to accomodate increased production of $Q$. Since the residential-to-individual land ratio is constant, the second-period residential zone must be greater. Thus the annulus of land added by the increase in $x_d$ is greater than that lost by the increase in $x_b$.

\section{Uncertainty and Speculation}

We now drop the perfect foresight assumption and suppose instead that landowners are uncertain in period $1$ about second-period land rent. In particular, they share the probability distribution $g(\Gamma_2)$ over $\Gamma_2$ in the first period where the true value of $\Gamma_2$ is resolved in period $2$. We retain the assumption that $\Gamma_2 > \Gamma_1$, so:

\begin{equation*}
    \int_0^{\Gamma_1} g(\Gamma_2) d\Gamma_2 = 0
\end{equation*}

We assume landowners are risk neutral. In this environment, landowners' first-period decisions are speculative. This means the monocentric city model equilibrium is the result of a sequential decision proess. The smaller the value of $\Gamma_2$ resolved, the less will be needed. A related difference is that depending upon the value of $\Gamma_2$ resolved, the outer boundary of industrial expansion may not be $x_b$. Because of this, we introduce $x_e$ as the boundary. While $x_b$ is determined in period $1$ on speculation, $x_e$ is determined along with $x_d$ in period $2$ once $\Gamma_2$ is resolved. If $\Gamma_2$ is small, then $x_e < x_b$ indicating that more land than necessary was preserved. If $\Gamma_2$ is sufficiently large, then $x_e > x_b$, indicating all preserved land is used and that a second industiral ring occurs beyond the first-period residential zone.

A final difference is that if $\Gamma_2$ is sufficiently small, land rents can be negative at some locations in the second period where residential development occured in the first period.

\subsection{Second-Period Equilibrium}

The second-period equilibrium is characterized by seven variables, six from perfect-foresight planning and one from speculation. The six are $Q_2, P_2, w_2, R_2^Q, R_2^H, x_d$. The seventh is $x_e$. 

Case 1:

We begin with the lower tail of $g(\Gamma_2)$. Suppose:

\begin{equation*}
    \Gamma_2 = \Gamma_1 + \delta
\end{equation*}

where $\delta$ is positive but very small, indicating miniscule growth in export demand for $Q$ between periods. The amount of land needed for both industrial and residential expansion is much less than that preserved by speculation between $x_a$ and $x_b$. It will occupy an annulus of land between $x_a$ and $x_e$. The expanded workforce will be accomodated by an annulus of residential development between $x_e$ and $x_d$ where:

\begin{equation*}
    x_a < x_e < x_d < x_b < x_c.
\end{equation*}

The amount of land in the industrial zone must equal $\lambda Q_2$, and that in the two residential zone $\mu Q_2$. This provides two of the remaining four equilibrium conditions:

\begin{equation}
    Q_2 = \pi x_e^2 / \lambda
\end{equation}

\begin{equation}
    Q_2 = \pi(x_c^2 - x_b^2 + x_d^2 - x_e^2) / \mu
\end{equation}

The other two conditions concern equilibrium in the land market. In other that land beyond $x_d$ not be offered for residential development, or land inside it withheld, residential land rent must be zero at $x_d$:

\begin{equation}
    R_2^H = Tx_d. \labeL{sepculation_residential_land_rent}
\end{equation}

In order that land beyond $x_e$ not be developed industrially,or land inside it residentially, land rent must be the same for both uses at $x_e$:

\begin{equation}
    R_2^Q - R_2^H = (t / \lambda - T)x_e.
\end{equation}

An important feature of equilibrium in Case 1, and a direct implication of the above, is that second-period land rent is negative throughout the residential zone developed in the first period.

As $\Gamma_2$ increases, more land is required for both kinds of expansion, so $x_e$ and $x_d$ increase. Once $x_d$ reaches $x_b$, there is no longer any vacant land in the leapfrog zone to keep residential land rent zero at $x_d$. Thus the above Equation \eqref{sepculation_residential_land_rent} is no longer valid and Case 1 no longer applies. Let the value of $\Gamma_2$ for which:

\begin{equation*}
    R_2^H = Tx_b
\end{equation*}

be called $\Gamma_2^1$. That is the minimum value of $\Gamma_2$ for which

\begin{equation}
    x_d = x_b
\end{equation}

Case 1 and set of conditions above apply only when

\begin{equation*}
    \Gamma_1 < \Gamma_2 < \Gamma_2^1
\end{equation*}

Case 2:

Now consider:

\begin{equation*}
    \Gamma_2 = \Gamma_2^1
\end{equation*}

All land in the leapfrog will be developed here. The equilibrium shows that:

\begin{equation*}
    x_e = x_c \sqrt{\lambda / (\mu + \lambda)}
\end{equation*}

The case is distinguished from the previous one in two ways. First, as $\Gamma_2$ increases above $\Gamma_2^1$, $x_d$ and $x_e$ remain stationary. Second, land rent is negative in only part of the residential zone. As $\Gamma_2$ and consequently $R_2^H$ increase, the point beyond which rent is negative becomes more remote. Let the value of $\Gamma_2$ for which:

\begin{equation*}
    R_2^H = Tx_c.
\end{equation*}

be called $\Gamma_2^2$. That is the minimum value of $\Gamma_2$ for which $x_d = x_b$ holds. Thus the result applies $\Gamma_2^1 \leq \Gamma_2 \leq \Gamma_2^2$.

Case 3:

Suppose

\begin{equation*}
    \Gamma_2 = \Gamma_2^2 + \delta.
\end{equation*}

We now have:

\begin{equation*}
    x_a < x_e < a_b < x_c < x_d.
\end{equation*}

Since the amount of land in the industrial zone must be $\lambda Q_2$ and that in the residential zone must be $\mu Q_2$, we have:

\begin{equation}
    Q_2 = \pi (x_d^2 - x_e^2) / \mu
\end{equation}

As $\Gamma_2$ increases, both zones expand as $x_e$ and $x_d$ increase. but this expansion cannot continue indefinitely because $x_e$ eventually reaches $x_b$. Once this happens, there is no longer any residential development in the leapfrog zone to keep both rents equal at $x_e$. Let the value of $\Gamma_2$ for which:

\begin{equation*}
    R_2^Q - R_2^H = (t / \lambda - T)x_b
\end{equation*}

be called $\Gamma_2^3$. That is the minimum value of $\Gamma_2$ for which

\begin{equation}
    x_e = x_b
\end{equation}

Therefore, $\Gamma_2^2 < \Gamma_2 < \Gamma_2^3$.

Case 4:

\begin{equation*}
    \Gamma_2 = \Gamma_2^3
\end{equation*}

so that all land in the leapforg zone is developed industrially. Taken together, we have $x_d$ can be solved for as a function of $x_b$:

\begin{equation*}
    x_d = x_b \sqrt{1 + \mu / \lambda}
\end{equation*}

Let the value of $\Gamma_2$ for which:

\begin{equation*}
    R_2^Q - R_2^H = (t / \lambda - T)x_c
\end{equation*}

be called $\Gamma_2^4$. That is the minimum value of $\Gamma_2$ for which $x_e = x_b$.

Then $\Gamma_2^3 \leq \Gamma_2 \leq \Gamma_2^4$.

Case 5:

\begin{equation*}
    \Gamma_2 > \Gamma_2^4
\end{equation*}

Second period export demand for $Q$ is especially strong here, and landowners find it advantageous to develop industrially not only the entire leapfrog zone, but another ring between $x_c$ and $x_e$. Residential expansion occurs even further out between $x_e$ and $x_d$. Thus there are four von Thunen rings of developed land - two of each kind, alternating as $x$ increases.

Because the amount of land in the two industrial zones must be $\lambda Q_2$,

\begin{equation}
    Q_2 = \pi (x_b^2 + x_e^2 - x_c^2) / \lambda
\end{equation}

must hold. The second period land rent function, $ER_2(x)$, differs among the cases. In case 1, it is

\begin{equation*}
    \begin{aligned}
        ER_2(x) & = R_2^Q - tx / \lambda \text{ for } x \in [0, x_e] \\
        & = R_2^H - Tx \text{ for } x \in (x_e, x_d], x \in (x_b, x_c] \\
        & = 0 \text{ for } x \in (x_d, \infty)
    \end{aligned}
\end{equation*}

For case 2, it is:

\begin{equation*}
    \begin{aligned}
        ER_2(x) & = R_2^Q - tx / \lambda \text{ for } x \in [0, x_e] \\
        & = R_2^H - Tx \text{ for } x \in (x_c, x_e] \\
        & = 0 \text{ for } x \in (x_e, \infty)
    \end{aligned}
\end{equation*}

For case 3 and 4, it is:

\begin{equation*}
    \begin{aligned}
        ER_2(x) & = R_2^Q - tx / \lambda \text{ for } x \in [0, x_e] \\
        & = R_2^H - Tx \text{ for } x \in (x_e, x_d] \\
        & = 0 \text{ for } x \in (x_d, \infty)
    \end{aligned}
\end{equation*}

For case 5, it is:

\begin{equation*}
    \begin{aligned}
        ER_2(x) & = R_2^Q - tx / \lambda \text{ for } x \in [0, x_b], x \in (x_c, x_e] \\
        & = R_2^H - Tx \text{ for } x \in (x_b, x_c], x \in (x_e, x_d] \\
        & = 0 \text{ for } x \in (x_d, \infty)
    \end{aligned}
\end{equation*}

\subsection{First-Period Equilibrium}

We designate $\phi_1 = (x_a, x_b, x_c, R_1^H, R_1^Q, Q_1, P_1, w_1)$ and the expected present value, then, of the strategy to develop residentially at $x_c$ in period $1$ is:

\begin{equation}
    R_1^H - Tx_c + \frac{1}{1 + r}[\int_{\Gamma_1}^\infty g(\Gamma_2)(R_2^H(\Gamma_2, \phi_1) - Tx_c) d\Gamma_2]
\end{equation}

where the notation $R_2^H(\Gamma_2, \phi_1)$ indicates the dependence of $R_2^H$ on $\phi_1$ and the value $\Gamma_2$ that occurs in period $2$. Under the strategy to wait until the second period to develop at $x_c$, and the values can be designated by different cases.

Under perfect-foresight planning, landowners may preserve a ring of undeveloped land between the first-period industrial and residential zones, providing an example of leapfrog development. In the second period this land is filled-in with industrial development to accomodate increased demand for industrial production.

Urban land conversion is a dynamic process and as such should be evaluated by dynamic rather than static criteria. It does not follow just because a land-use configuration is inefficient at one moment in time, that it is inefficient in the larger scheme of things where it is evolving. Indeed in a growing city, efficiency will require that interior parcels are sometimes withheld from early development and preserved for alternative future uses.