\chapter{Scale Economics, Product Differentiation, and the Pattern of Trade}

\section{Model}

\subsection{Assumptions of the Model}

There are assumed to be a large number of potential goods, all of which enter systemtrically into demand. Specifically, we assume that all individuals in the economy have the same utility function:

\begin{equation}
    U = \sum_{i} c_i^{\theta}, \quad 0 < \theta < 1
\end{equation}

where $c_i$ is consumption of the $i$th good. The number of goods actually produced, $n$, will be assumed to be large, although smaller than the potential range of products.

There will be assumed to be only one factor of production, labor. All goods will be produced with the same cost function,

\begin{equation}
    l_i = \alpha + \beta x_i, \quad \alpha, \beta > 0, i = 1, \ldots, n
\end{equation}

where $l_i$ is labor used in producing the $i$th good, and $x_i$ is the output of that good. Average cost declines at all level of output, although at a diminishing rate.

Output of each good must equal the su of individual consumption. If we can identify individuals with workers, output must euqal consumption of a representative individualtimes the labor force:

\begin{equation}
    x_i = Lc_i, \quad i = 1, \ldots, n
\end{equation}

We assume full employment, so that the total labor force must just be exhausted by labor used in production:

\begin{equation}
    L = \sum_{i=1}^n (\alpha + \beta x_i)
\end{equation}

Finally, we assume that firms maximize profits, but that there is free entry and exit of firms, so that in equilibrium profits will always be zero.

\subsection{Equilibrium in a Closed Economy}

First, I analyze consumer behavior to derive demand functions. Then profit-maximizing behavior by firms is derived, treating the number of firms as given. Finally, the assumption of free entry is used to determine the equilibrium number of firms.

The reason that a Chamberlinian approach is useful here is that, in spite of imperfect competition, the equilibrium of the model is determinate in all essential respect because the special nature of demand rules out strategic interdependence among firms. Because firms can costlessly differentiate their products, and all products enter systemtrically into demand, two firms will never want to produce the same product; each good will be produced by only one firm.

COnsider, an individual maximizing utility subject to a budget constraint. The FOC from the maximum problem have the form:

\begin{equation}
    \theta c_i^{\theta - 1} = \lambda p_i, \quad i = 1, \ldots, n
\end{equation}

where $p_i$ is the price on the $i$th good and $\lambda$ is the shadow price on the budget constraint, that is, the marginal utility of income. The single firm faces the demand curve by:

\begin{equation}
    p_i = \theta \lambda^{-1} (x_i / L_i)^{\theta - 1} \quad i = 1, \ldots, n
\end{equation}

Each firm faces a demand curve with an elasticity $1 / (1 - \theta)$, and the profit-maximizing price is therefore,

\begin{equation}
    p_i = \theta^{-1} \beta w
\end{equation}

where $w$ is the wage rate and prices and wages can be defined in terms of any unit. Note that since $\theta, \beta$ and $w$ are the same for all firms, prices are the same for all godos and we can adopt the shorthand $p = p_i$ for all $i$.

The price $p$ is independent of output given the special assumption about cost and utility. To determine profitability, however, we need to look at outputs. Profits of the firm producing good $i$ are:

\begin{equation}
    \pi_i = p x_i - \{\alpha + \beta x_i\}w \quad i = 1, \ldots, n
\end{equation}

its profit are positive, new firms will enter, causing the marginal utility of income to rise and profits to fall until profits are driven to zero. In equilibrium, then $\pi = 0$, implying for the output of a representative firm:

\begin{equation}
    x_i = \alpha / (p / w - \beta) = \alpha \theta / \beta (1 - \theta) \quad i = 1, \ldots, n
\end{equation}

Thus output per firm is determined by the zero-profit condition. Again, since $\alpha, \beta$ and $\theta$ are the same for all firms we can use the shorthand $x = x_i$ for all $i$.

Finally, we can determine the number of goods produced by using the condition of full employment. Then we have:

\begin{equation}
    n = \frac{L}{\alpha + \beta x} = \frac{L(1 - \theta)}{\alpha}
\end{equation}