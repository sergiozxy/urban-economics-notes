\chapter{Scale Economics, Product Differentiation, and the Pattern of Trade}

\section{Model}

\subsection{Assumptions of the Model}

There are assumed to be a large number of potential goods, all of which enter systemtrically into demand. Specifically, we assume that all individuals in the economy have the same utility function:

\begin{equation}
    U = \sum_{i} c_i^{\theta}, \quad 0 < \theta < 1
\end{equation}

where $c_i$ is consumption of the $i$th good. The number of goods actually produced, $n$, will be assumed to be large, although smaller than the potential range of products.

There will be assumed to be only one factor of production, labor. All goods will be produced with the same cost function,

\begin{equation}
    l_i = \alpha + \beta x_i, \quad \alpha, \beta > 0, i = 1, \ldots, n
\end{equation}

where $l_i$ is labor used in producing the $i$th good, and $x_i$ is the output of that good. Average cost declines at all level of output, although at a diminishing rate.

Output of each good must equal the su of individual consumption. If we can identify individuals with workers, output must euqal consumption of a representative individualtimes the labor force:

\begin{equation}
    x_i = Lc_i, \quad i = 1, \ldots, n
\end{equation}

We assume full employment, so that the total labor force must just be exhausted by labor used in production:

\begin{equation}
    L = \sum_{i=1}^n (\alpha + \beta x_i)
\end{equation}

Finally, we assume that firms maximize profits, but that there is free entry and exit of firms, so that in equilibrium profits will always be zero.

\subsection{Equilibrium in a Closed Economy}

First, I analyze consumer behavior to derive demand functions. Then profit-maximizing behavior by firms is derived, treating the number of firms as given. Finally, the assumption of free entry is used to determine the equilibrium number of firms.

The reason that a Chamberlinian approach is useful here is that, in spite of imperfect competition, the equilibrium of the model is determinate in all essential respect because the special nature of demand rules out strategic interdependence among firms. Because firms can costlessly differentiate their products, and all products enter systemtrically into demand, two firms will never want to produce the same product; each good will be produced by only one firm.

COnsider, an individual maximizing utility subject to a budget constraint. The FOC from the maximum problem have the form:

\begin{equation}
    \theta c_i^{\theta - 1} = \lambda p_i, \quad i = 1, \ldots, n
\end{equation}

where $p_i$ is the price on the $i$th good and $\lambda$ is the shadow price on the budget constraint, that is, the marginal utility of income. The single firm faces the demand curve by:

\begin{equation}
    p_i = \theta \lambda^{-1} (x_i / L_i)^{\theta - 1} \quad i = 1, \ldots, n
\end{equation}

Each firm faces a demand curve with an elasticity $1 / (1 - \theta)$, and the profit-maximizing price is therefore,

\begin{equation}
    p_i = \theta^{-1} \beta w
\end{equation}

where $w$ is the wage rate and prices and wages can be defined in terms of any unit. Note that since $\theta, \beta$ and $w$ are the same for all firms, prices are the same for all godos and we can adopt the shorthand $p = p_i$ for all $i$.

The price $p$ is independent of output given the special assumption about cost and utility. To determine profitability, however, we need to look at outputs. Profits of the firm producing good $i$ are:

\begin{equation}
    \pi_i = p x_i - \{\alpha + \beta x_i\}w \quad i = 1, \ldots, n
\end{equation}

its profit are positive, new firms will enter, causing the marginal utility of income to rise and profits to fall until profits are driven to zero. In equilibrium, then $\pi = 0$, implying for the output of a representative firm:

\begin{equation}
    x_i = \alpha / (p / w - \beta) = \alpha \theta / \beta (1 - \theta) \quad i = 1, \ldots, n
\end{equation}

Thus output per firm is determined by the zero-profit condition. Again, since $\alpha, \beta$ and $\theta$ are the same for all firms we can use the shorthand $x = x_i$ for all $i$.

Finally, we can determine the number of goods produced by using the condition of full employment. Then we have:

\begin{equation}
    n = \frac{L}{\alpha + \beta x} = \frac{L(1 - \theta)}{\alpha}
\end{equation}

\subsection{Effects of Trade}

Trades can occur because in the presence of increasing returns, each good will be produced in only one country - for the same reasons that each good is produced by only one firm. The symmetry of the situation ensures that the two countries will have the same wage rate, and that the price of any good produced in either country will be the same. The number of goods produced in each country can be determined from the full employment condition:

\begin{equation}
    n = L(1 - \theta) / \alpha; \quad n^* = L^*(1 - \theta) / \alpha
\end{equation}

where $L^*$ is the labor force of the second country and $n^*$ is the number of goods produced there.

Individuals will maximize their utility but they will now distribute their expenditure over both the $n$ goods produced in the home country and the $n^*$ goods produced in the foreign country. Because of the extended range of choice, welfare will increase even though the ``real wage'' $w/p$ remains unchanged. Also, the symmetry of the problem allows us to determine trade flows. It is apparent that individuals in the home country will spend a fraction $n^* / (n + n^*)$ of their income on foreign goods, while foreigners will spend $n/(n + n^*)$ of their income on home goods. Thus the value of home country imports measured in wage units is $Ln^* / (n + n^*) = LL^* / (L + L^*)$. This equals the value of foriegn country imports, confirming that with equal wage rates in the two conutries we will have balance-of-payments equilibrium.

\section{Transport Costs}

\subsection{Individual Behavior}

Transportation costs will be assumed to be of the ``iceberg'' type, that is, only a fraction $g$ of any good shipped arrives, with $1 - g$ lost in transit.

An individual in the home country will have a choice over $n$ products produced at home and $n^*$ products produced at home and $n^*$ products produced abroad. The price of a domestic product will be the same as that received by the producer $p$. Foreign products, however, will cost more than the producer's price; if foreign firms charge $p^*$, home country consumers will have to pay the c.i.f. price $\hat{p}^* = p^* / g$. Similarly, foreign buyers of domestic products will pay $\hat{p} = p / g$.

Since the price to consumers of goods of differnet countries will in general not be the same, consumption of each imported good will differ from consumption of each domestic good.

To determine world equilibrium, we must also take into account the quantities of goods used up in transit. For determining total demand, then, we need to kow the ratio of total demand by domestic residents for each foreign product to demand for each domestic product. Letting $\sigma$ denote this ratio, and $\sigma^*$ the corresponding ratio for the other country, we can show that:

\begin{equation}
    \begin{aligned}
        \sigma & = (p/p^*)^{1/(1 - \theta)}g^{\theta / (1 - \theta)} \\
        \sigma^* & = (p^*/p)^{1/(1 - \theta)}g^{\theta / (1 - \theta)}
    \end{aligned}
\end{equation}

The overall demand pattern of each individual can then be derived from the requirement that this spending just equal his wage; that is, in the home country we must have $(np + \sigma np^*)d = w$, where $d$ is the consumption of a representative domestic good; and similarly in the foreign country. The behavior of individuals can now be used to analyze the behavior of firms. The important point to notice is that the elasticity of export demand facing any given firm is $1 / (1 - \theta)$, which is the same as the elasticity of domestic demand. Thus transportation costs have no effect on firm's policy/ Writing out these conditions again, we have:

\begin{equation}
    \begin{aligned}
        p & = w\beta / \theta; p^* = w^* \beta / \theta \\
        n & = L(1 - \theta) / \alpha; n^* = L^*(1 - \theta) / \alpha
    \end{aligned}
\end{equation}

\subsection{Determination of Equilibrium}

The only variable which can be affected is the relative wage rate $w / w^* = w$, which no longer need be equal to one. We can determine $w$ by looking at any one of three equivalent market-clearing conditions: (i) equality of demand and supply for home country labor; (ii) equality of demand and supply for foreign country labor; (iii) balance-of-payments equilibrium. It can be shown that the home country's balance of payments measued in wage units of the other country, is:

\begin{equation}
    \begin{aligned}
        B & = \frac{\sigma^* nw}{\sigma^*n + n^*}L^* - \frac{\sigma n^*}{n + \sigma n^*}wL \\
        & = wLL^* \left[ \frac{\sigma^*}{\sigma^* L + L^*} - \frac{\sigma}{L + \sigma L^*}\right]
    \end{aligned}
\end{equation}

Since $\sigma$ and $\sigma^*$ are both functions of $p/p^* = w$, the condition $B = 0$ can be used to determine the relative wage.