\chapter{When is the economy monocentric}

\section{A formal model of a spatial economy}

Consider a long-narrow country, in which area is represented by one-dimensional unbounded location space, $X = \mathbb{R}$. The quality of land is homogenous and density of land is equal to $1$ everywhere. The country has a continuum of homogenous workers with a given size, $N$. Each worker is endowed with a unit of labor, and is free to chooose any location and job in the country. The consumers of the country consist of the workers and landlords.

Each consumer consumes a homogenous agricultural good (A-good) together with a continuum of differentiated manufacturing goods (M-goods) of size $n$. Here, $n$ is to be determined endogenously. All consumers have the same utililty function given by:

\begin{equation}
    u = \alpha_A \log z + \alpha_M \log \left\{ \int_0^n q(\omega)^{\rho} d\omega \right\}^{1 / \rho}
\end{equation}

where $z$ represents the amount of A-good consumed, $q(\omega)$ is the consumption of each variety $\omega \in [0, n]$ of M-good; and $\alpha_A, \alpha_M$ and $\rho$ are positive constants such that $\alpha_A + \alpha_M = 1$ and $0 < \rho < 1$. Note that a smaller $\rho$ means that consumers have a stronger preference for variety in M-goods.

Suppose that a consumer has an income, $Y$, and faces a set of prices $p_A$ and $p_M(\omega)$. Then for choosing the consumption bundle that maximizes the equation above, subject to the budget constraint $p_A z + \int_0^n p_M(\omega) q(\omega) d\omega = Y$, demand functions of the consumer can be obtained as:

\begin{equation}
    z = (\alpha_A Y) / p_A
\end{equation}

\begin{equation}
    q(\omega) = (\alpha_M Y / p_M(\omega)) \left(p_M(\omega)^{-\gamma} / \int_0^n p_M(\omega)^{-\gamma} d\omega \right) \label{eq:demand_for_q}
\end{equation}

for each $\omega \in [0, n]$, where $\gamma = \rho / (1 - \rho)$. Note that the Equation \eqref{eq:demand_for_q} that the demand for any variety in M-good has the same price elasticity, $E$, given by:

\begin{equation}
    E = 1 / (1 - \rho) = 1 + \gamma
\end{equation}

Thus $E$ increases as $\rho$ increases. Substituting $z$ and $q(\omega)$ into the utility function yields the following indirect utility function:

\begin{equation}
    u = \log\{\alpha_A^{\alpha_A} \alpha_M^{\alpha_M}Yp_A^{\alpha_A}\} + \frac{\alpha_M}{\gamma} \log \left\{ \int_0^n (\omega)^{-\gamma} d\omega \right\}
\end{equation}

Next, the A-good is assumed to be produced under constant returns, where each unit of A-good consumes a unit of land and $a_A$ units of labor. Each M-good is produced with labor only. All types of M-good have the same production technology under increasing returns such that the total labor input, $L$, for production of quantity $Q$ of any product is given by:

\begin{equation}
   L = f + \alpha_M Q 
\end{equation}

where $f$ is the fixed labor requirement and $\alpha_M$ the marginal labor input. We assume, for simplicity, that the transport cost of each good takes Samuelson's iceberg form: if a unit of good $i$ ($i = A \text{ or } M$) is shapped over a distance $d$, only $e^{-t_i d}$ units actually arrive, where $t_i$ is a positive constant.

Owing to scale economics in production, each variety of M-good is assumed to be produced by a single, specialized firm. If a firm locates at $x \in X$ and produces an M-product, it chooses an f.o.b. price $p_M(x)$ so as to maximize its profit at the Chamberlinian equilibrium. By the assumption of iceberg transport technology, the effective (delivered) price, $p_M(y \mid x)$, for consumers at location $y \in X$ of any M-product produced at location $x$ is given by:

\begin{equation}
  p_M(y \mid x) = p_M(x) e^{t_M |y - x|}
  \label{eq:M_price_transport}
\end{equation}

Given the delivered price function above, it can be readily verified that the price elasticity of the total demand for any M-product is independent of the spatial distribution of the demand, and equals the price elasticity, $E$, of each consumer's demand given by above demand function. Thus, given the equilibrium wage rate, $W(x)$, at $x$, by the equality of the marginal revenue and marginal cost, $p_M(x) (1 - E)^{-1} = a_M W(x)$, the optimal f.o.b. price of the firm can be obtained by:

\begin{equation}
  p_M(x) = a_M W(x) / \rho
\end{equation}

which represents the familiar result that for each monopolistic firm will chare its f.o.b. price at a markup over the marginal cost $a_M W(x)$. Thus, if $Q$ is the output of the firm, its profit equals:

\begin{equation}
  \pi(x) = p_M(x) Q - W(x)(f + a_M Q) = a_M \gamma^{-1} W(x)(Q - \gamma f / a_M)
\end{equation}

Therefore, given any equilibrium configuration of the economy, if an M-firm operates at $x$, then by the zero-profit condition, its output must be equal to

\begin{equation}
  Q^* = \gamma f / a_M
\end{equation}

which is a constant independent of location.
