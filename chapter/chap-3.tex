\chapter{First Nature, Second Nature, and Metropolitan Location}

\section{A Modeling Approach}

We begin by assuming that everyone in the economy shares a utility function into which both the agricultural good and a manufactures aggregate under:

\begin{equation}
    U = C_M^{\mu} C_A^{1 - \mu}
\end{equation}

where $\mu$ is the share of manufactured goods in expenditure. The manufactures aggregate in turn is a CES function of a large number of potential varieties, not all of which will actually be available.

\begin{equation}
    C_M = \left[ \sum_i C_M^{\sigma - 1/ \sigma}\right]^{\sigma / \sigma - 1}
\end{equation}

Agricultural goods are produced by a sector-specific factor, agricultural labor. They are produced under constant returns; without loss of generality we assume that the unit labor requirement is one. In the manufacturing sector, however, there are increasing returns; we introduce these by assuming that for any variety that is actually produced, there is a fixed labor input required independent of the volume of output:

\begin{equation}
    L_{Mi} = \alpha + \beta q_{Mi}
\end{equation}

Both kinds of labor will be assumed to be fully employed. Thus

\begin{equation}
    L_M = \sum_i L_{Mi}
\end{equation}

and

\begin{equation}
    L_A = C_A.
\end{equation}

First, agricultural labor will be assumed to be distributed across space; in the next section we will assume that it is evenly spread along a line, in the section following that it is evenly spread across a disk. Second, we assume that although agricultural goods can be costlessly transported. The proportional rate of melting is assumed to be constant per unit of distance, implying that if a single unit of a manufactured good is shipped a distance $D$, the quantity that arrives is only $e^{-\tau D}$, where $\tau$ is the transport cost.

First, we note that given the absence of any transportation costs for agricultural goods, all agricultural workers will receive the same nominal wage. Second, we note that manufacturing will have a monopolistically competitive market structure; the form of the utility function together with the assumption of ``iceberg'' transportation costs implies a constant elasticity of demand, and hence, that the price of each manufactured good at the factory gate will be a constant proportional markup on the wage rate. Third, we note that a fraction $\mu$ of total expenditure will fall on manufactured goods; since profits are competed away, manufacturing workers will receive a share $\mu$ of total income, agricultural workers $1 - \mu$. Finally, we note that the elasticity of substitution between any two manufactured products is $\sigma$.

\section{A One-Dimensional Model}

We assume that agricultural workers are distributed evenly along a line of unit length. The method we will use is to posit an initial situation in which all manufacturing is concentrated at a single location, then ask whether an individual firm will find it advantageous to move to any other location. Suppose, then, that all manufacturing is concentrated at the location $x_C$ along the unit interval. We need to ask whether it is to $x_A$ site. If there is no site at which this hypothetical real wage exceeds the real wage paid a $x_C$, then a defecting firm will be unable to attract workers away from the city and still earn a profit; hence in this case, concentration at $x_C$ is an equilibrium.

Let $w$ be the ratio of the nominal wage rate that a firm at $x_A$ would have to pay to attract workers to that at $x_C$. Given the monopolistically competitive market structure, the ratio of the f.o.b. price charged by a defecting firm choosing to locate at $x_A$ to that of a typical good manufactured at $x_C$ will also be $w$. The price ratio to a consumer at location $x$ will reflect both this f.o.b. price ratio and transport costs, which depend on the consumer's relative distance from $x_A$ and $x_C$. Let $p_x$ be this relative price to a consumer at location $x$: given our assumption about transport costs, it is simply:

\begin{equation}
    p_x = w e^{\tau(|x_A - x| - |x_C - x|)}
\end{equation}

Next consider the ratio of scales of a product manufactured at $x_A$ to that of a typical good manufactured at $x_C$. Given the elasiticityy of substitution of $\sigma$, the ratio of consumption by a consumer at location $x$ is:

\begin{equation}
    c = p_x^{-\sigma}
\end{equation}

The ratio of value of sales to the consumer at $x$, however, is less sensitive to the price, because volume effects are offset by valuation effects; thus we have for the value ratio

\begin{equation}
    s = p_x^{1 - \sigma}
\end{equation}

To calculate the overall ratio of sales of a firm that defects to $x_A$ to that of a typical product from the metropolis, we note that manufacturing workers, who account for a fraction $\mu$ of demand, are all concentrated at $x_C$; while agricultural workers, who account for the rest, are spread evenly along the unit interval. This implies that the overall sales ratio is:

\begin{equation}
    S = \mu [we^{\tau |x_C - x_A|}]^{1 - \sigma}
\end{equation}
 
