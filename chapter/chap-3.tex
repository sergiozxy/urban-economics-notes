\chapter{First Nature, Second Nature, and Metropolitan Location}

\section{A Modeling Approach}

We begin by assuming that everyone in the economy shares a utility function into which both the agricultural good and a manufactures aggregate under:

\begin{equation}
    U = C_M^{\mu} C_A^{1 - \mu}
\end{equation}

where $\mu$ is the share of manufactured goods in expenditure. The manufactures aggregate in turn is a CES function of a large number of potential varieties, not all of which will actually be available.

\begin{equation}
    C_M = \left[ \sum_i C_M^{\sigma - 1/ \sigma}\right]^{\sigma / \sigma - 1}
\end{equation}

Agricultural goods are produced by a sector-specific factor, agricultural labor. They are produced under constant returns; without loss of generality we assume that the unit labor requirement is one. In the manufacturing sector, however, there are increasing returns; we introduce these by assuming that for any variety that is actually produced, there is a fixed labor input required independent of the volume of output:

\begin{equation}
    L_{Mi} = \alpha + \beta q_{Mi}
\end{equation}

Both kinds of labor will be assumed to be fully employed. Thus

\begin{equation}
    L_M = \sum_i L_{Mi}
\end{equation}

and

\begin{equation}
    L_A = C_A.
\end{equation}

First, agricultural labor will be assumed to be distributed across space; in the next section we will assume that it is evenly spread along a line, in the section following that it is evenly spread across a disk. Second, we assume that although agricultural goods can be costlessly transported. The proportional rate of melting is assumed to be constant per unit of distance, implying that if a single unit of a manufactured good is shipped a distance $D$, the quantity that arrives is only $e^{-\tau D}$, where $\tau$ is the transport cost.

First, we note that given the absence of any transportation costs for agricultural goods, all agricultural workers will receive the same nominal wage. Second, we note that manufacturing will have a monopolistically competitive market structure; the form of the utility function together with the assumption of ``iceberg'' transportation costs implies a constant elasticity of demand, and hence, that the price of each manufactured good at the factory gate will be a constant proportional markup on the wage rate. Third, we note that a fraction $\mu$ of total expenditure will fall on manufactured goods; since profits are competed away, manufacturing workers will receive a share $\mu$ of total income, agricultural workers $1 - \mu$. Finally, we note that the elasticity of substitution between any two manufactured products is $\sigma$.

\section{A One-Dimensional Model}

We assume that agricultural workers are distributed evenly along a line of unit length. The method we will use is to posit an initial situation in which all manufacturing is concentrated at a single location, then ask whether an individual firm will find it advantageous to move to any other location. Suppose, then, that all manufacturing is concentrated at the location $x_C$ along the unit interval. We need to ask whether it is to $x_A$ site. If there is no site at which this hypothetical real wage exceeds the real wage paid a $x_C$, then a defecting firm will be unable to attract workers away from the city and still earn a profit; hence in this case, concentration at $x_C$ is an equilibrium.

Let $w$ be the ratio of the nominal wage rate that a firm at $x_A$ would have to pay to attract workers to that at $x_C$. Given the monopolistically competitive market structure, the ratio of the f.o.b. price charged by a defecting firm choosing to locate at $x_A$ to that of a typical good manufactured at $x_C$ will also be $w$. The price ratio to a consumer at location $x$ will reflect both this f.o.b. price ratio and transport costs, which depend on the consumer's relative distance from $x_A$ and $x_C$. Let $p_x$ be this relative price to a consumer at location $x$: given our assumption about transport costs, it is simply:

\begin{equation}
    p_x = w e^{\tau(|x_A - x| - |x_C - x|)}
\end{equation}

Next consider the ratio of scales of a product manufactured at $x_A$ to that of a typical good manufactured at $x_C$. Given the elasiticityy of substitution of $\sigma$, the ratio of consumption by a consumer at location $x$ is:

\begin{equation}
    c = p_x^{-\sigma}
\end{equation}

The ratio of value of sales to the consumer at $x$, however, is less sensitive to the price, because volume effects are offset by valuation effects; thus we have for the value ratio

\begin{equation}
    s = p_x^{1 - \sigma}
\end{equation}

To calculate the overall ratio of sales of a firm that defects to $x_A$ to that of a typical product from the metropolis, we note that manufacturing workers, who account for a fraction $\mu$ of demand, are all concentrated at $x_C$; while agricultural workers, who account for the rest, are spread evenly along the unit interval. This implies that the overall sales ratio is:

\begin{equation}
    S = \mu [we^{\tau |x_C - x_A|}]^{1 - \sigma} + (1 - \mu) \int_0^1 [we^{\tau(|x_A - x| - |x_C - x|)}]^{1 - \sigma} dx
\end{equation}
 
or rearranging terms:

\begin{equation}
    S = w^{-(\sigma - 1)}[\mu[e^{-\tau|x_C - x_A|}]^{\sigma - 1} + (1 - \mu)\int_0^1 [e^{\tau (|x_C - x| - |x_A - x|)}]^{\sigma - 1} dx]
\end{equation}

The above equation determines relative sales as a function of the relative wage rate. First, note that by assumption all firms at $x_C$ are earning zero profits, with their operating surpuses just covering their fixed costs. A firm at $x_C$ must do the same. But the operating surplus of a firm in the Dixit-Stiglitz model is proportional to its sales, while the fixed costs are incurred in manufacturing labor, which at $x_A$ receives a relative wage $w$. It follows, then, that if there are to be zero profits we must have:

\begin{equation}
    S = w.
\end{equation}

Putting the above two equations together, we have our expression for the nominal wage rate that a firm at $x_A$ can afford to pay:

\begin{equation}
    w = [\mu [e^{-\tau |x_C - x_A|}]^{\sigma - 1} + (1 - \mu) \int_0^1 [e^{\tau (|x_C - x| - |x_A - x|)}]^{\sigma - 1} dx]^{1 / \sigma}
\end{equation}

To determine whether a concentration of manufactures at $x_C$ is an equilibrium, however, we need to compare not the relative nominal wage but the relative real wage. The difference between the two comes from the fact that manufactured goods produced at $x_C$ are part of workers' consumption basket, with a weight $\mu$. Taking this into account, we note that the relative real wage rate is:

\begin{equation}
    \omega = w e^{-\mu \tau |x_A - x_C|}
\end{equation}

equations above determine the real wage that a defecting firm could afford to pay if it were to locate at a site $x_A$ when all other manufacturing workers are concentrated at $x_C$. Rearranging above, we immediately have a simple definition of an equilibrium metropolitan location. A metropolis at $x_C$ is an equilibrium, if given that location, the maximum of the function is also at $x_C$.

It might seem that a metropolis at the center of the region is always a possible equilibrium. Unfortunately, this is not quite right, because there may exist no equilibrium with only a single metropolis. If transportation costs are high enough, then even if one posits a concentration of all manufacturing at the center, firms will find it advantageous to move away from the center to get better access to the rural market.

Symmetry ensures that the derivative of the integral is zero in this case, and all the exponential terms become nity, so that this derivative to the right is simply:

\begin{equation}
    R = -\tau \mu - \frac{\sigma - 1}{\sigma} \tau \mu
\end{equation}

If we take the derivative for $x_A$ slightly less than $x_C$, however, we find the derivative to the left to be:

\begin{equation}
    L = \tau \mu + \tau \mu \frac{\sigma - 1}{\sigma} = -R
\end{equation}

Thus, if there is a discontinuity in the slope of the market potential function.

We can solve analytically for this range by applying a criterion of local stability: a necessary condition for a metropolitan site to be an equilibrium site is that given a hypothetical metropolis at that site, the market potential has a local maximum there. Given the symmetry of the problem, we need only consider locations to the left of center, obviously in that case a more desirable alternative site, if it exists, will lie to the right. So the defining criterion for the range of potential sites is: if we posit a metropolis at some $x_C < 0.5$ then $d\omega / dx_A$ for $x_A$ slightly greater than $x_C$ must be negative.

All the absolute value terms become unambiguously signed, and when the expression is evaluated in the vicinity of $x_A = x_C$ all of the exponential terms become unity. Thus letting $R$ be the derivative of the relative wage with respect to $x_A$, when $x_A$ is just slightly greater than $x_C$, we have:

\begin{equation}
    R = \frac{\tau}{\sigma}[-\mu \sigma - \mu(\sigma - 1) + (1 - \mu)(\sigma - 1)(1 - 2x_C)]
\end{equation}

Bearing in mind that for an equilibrium metropolitan location we must have $R < 0$, we note that the above euqation contains two negative terms and one positive. The criterion $R < 0$ defines the range of potential metropolitan sites. For a metropolis to the left center, we have:

\begin{equation}
    1 - 2x_C < \frac{\mu}{1 - \mu} \frac{2\sigma - 1}{\sigma - 1}
\end{equation}

The larger is the right hand side, the wider the range of potential sites; if the right-hand side exceeds $1$, any point on the line can accommodate a metropolis.

\section{A Two-Dimensional Model}

In the one-dimensional case this range was a central portion of the line segment; in the two-dimensional case it will be a central disk within the regional disk. Let us posit a metropolis at $x_C, y_C$; assuming $y_C = 0$. We want to consider the attractiveness of an alternative location $x_A, y_A$. To calculate this, we need to know three distances. let $D_{AC}$ be the distance between the alternative location and the metropolis; we have:

\begin{equation}
    D_{AC} = \sqrt{(x_A - x_C)^2 + (y_A - y_C)^2} = \sqrt{(x_A - x_C)^2 + y_A^2}
\end{equation}

Let $D_C(x, y)$ be the distance from the metropolis to some other location $(x, y)$:

\begin{equation}
    D_C(x, y) = \sqrt{(x_C - x)^2 + y^2}
\end{equation}

and let $D_A(x, y)$ be the distance from the alternative location to $(x, y)$:

\begin{equation}
    D_A(x, y) = \sqrt{(x - x_A)^2 + (y - y_A)^2}
\end{equation}

By analogy with the one-dimensional case, the sales of a firm at the alternative location relative to those of one in the metropolis are:

\begin{equation}
    S = \mu[w e^{-\tau D_{AC}}] + \frac{1 - \mu}{\pi} \int_{-1}^1 \int_{-\sqrt{1 - x^2}}^{\sqrt{1 - x^2}} [w e^{\tau [D_X(x, y) - D_A(x, y)]}]^{\sigma - 1} dy dx
\end{equation}

The zero-profit condition once again requires that $S = w$. Thus we have:

\begin{equation}
    w = [\mu e^{-\tau(\sigma - 1) D_{AC}} + \frac{1 - \mu}{\pi} \int_{-1}^1 \int_{-\sqrt{1 - x^2}}^{\sqrt{1 - x^2}} e^{\tau(\sigma - 1) [D_X(x, y) - D_A(x, y)]} dy dx]^{1 / \sigma}
\end{equation}

and the relative real wage at the alternate location is:

\begin{equation}
    \omega = w e^{-\mu \tau D_{AC}}
\end{equation}