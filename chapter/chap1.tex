% !TEX root = ../notes_template.tex
\chapter{THE ECONOMICS OF DENSITY: EVIDENCE FROM THE BERLIN WALL}

\section{THEORETICAL MODEL}

The city consists of a set of discrete locations, index by $i = 1, \ldots, S$. Each block has an effective supply of floor space $L_i$. Floor space can be used commercially or residentially, and we denote the endogenous fractions of floor space allocated to commercial and residential use by $\theta_i$ and $1 - \theta_i$. City is populated by an endogenous measure of $H$ workers, who are perfectly mobile within the city and the larger economy, which provides a reservation level of utility $\overline{U}$. Workers decide whether or not to move to the city before observing idiosyncratic utility shocks for each possible pair of residence and employment blocks within the city.

\subsection{Workers}

Workers are risk neutral and have preferences that are linear in a consumption index: $U_{ijo} = C_{ijo}$, where $C_{ijo}$ denotes the consumption index for worker $o$ residing in block $i$ and working in block $j$. This consumption index depends on consumption of the single final good $c_{ijo}$; consumption of residential floor space ($\ell_{ijo}$); residential amenities ($B_i$) that capture common characteristics that make a block a more or less attractive place to live; the disutility from commuting from residence block $i$ to workplace block $j$ ($d_{ij} \geq 1$); and an idiosyncratic shock that is specific to individual workers and varies with the worker's blocks of employment and residence $(z_{ijo}$). This idiosyncratic shock captures the idea that individual workers can have idiosyncratic reasons for living and working in different parts of the city. In particular, the aggregate consumption index is assumed to take the Cobb-Douglas form:

\begin{equation}
    C_{ijo} = \frac{B_i z_{ijo}}{d_{ij}} (\frac{c_{ijo}}{\beta})^{\beta} (\frac{\ell_{ijo}}{1 - \beta})^{1 - \beta}, 0 < \beta < 1
\end{equation}

where the iceberg communiting cost $d_{ij} = e^{\kappa \tau_{ij}} \in [1, \infty)$ increases with the travel time $(\tau_{ij})$ between blocks $i$ and $j$. The parameter $\kappa$ controls the size of commuting costs.

For each worker $o$ living in block $i$ and commuting to block $j$, the idiosyncratic component of utility $(z_{ijo})$ is drawn from an independent Frechet distribution:

\begin{equation}
    F(z_{ijo}) = e^{-T_i E_j z_{ijo}^{-\varepsilon}}, T_i, E_j > 0, \varepsilon > 1
\end{equation}

where the scale parameter $T_i > 0$ determines the average utility derived from living in block $i$; the scale parameter $E_j$ determines the average utility derived from working in block $j$; and the shape parameter $\varepsilon > 1$ controls the dispersion of idiosyncratic utility.

The indirect utility from residing in block $i$ and working in block $j$ can be expressed in terms of the wage paid at this workplace $(w_j)$, commuting costs $(d_{ij})$, the residential floor price $(Q_i)$, the common component of amenities $(B_i)$, and the idiosyncratic shock $(z_{ijo})$:

\begin{equation}
    u_{ijo} = \frac{z_{ijo} B_i w_j Q_i^{\beta - 1}}{d_{ij}}
\end{equation}

where we have used utility maximization and the choice of the final good as numeraire. Although we model commuting costs in terms of utility, there is an isomorphic formulation in terms of a reduction in effective units of labor, because the iceberg commuting cost $d_{ij} = e^{\kappa \tau_{ij}}$ enters the indirect utility function multiplicatively. As a result, commuting costs are proportional to wages, and this specification captures changes over time in the opportunity cost of travel time.

Since indirect utility is a monotonic function of the idiosyncratic shock $(z_{ijo})$, whch has a Frechet distribution, it follows that indirect utility for workers living in block $i$ and working in block $j$ also has a Frechet distribution. Each worker chooses the bilateral commute that offers her the maximum utility, where the maximum of Frechet distributed random variables is itself Frechet distributed. Using these distributions of utility, the probability that a worker choose to  live in block $i$ and work in block $j$ is:

\begin{equation}
    \pi_{ij} = \frac{T_i E_j(d_{ij} Q_i^{1 - \beta})^{-\varepsilon} (B_i w_j)^{\varepsilon}}{\sum_{r=1}^S \sum_{s=1}^S T_r E_s(d_{rs}Q_r^{1 - \beta})^{-\varepsilon} (B_r w_s)^{\varepsilon}} \equiv \frac{\Phi_{ij}}{\Phi}.
\end{equation}

Summing these probabilities across workplaces for a given residence, we obtain the overall probability that a worker resides in block $i(\pi_{Ri})$, while summing these probabilities across residences for a given workplace, we obtain the overall probability that a worker works in block $j(\pi_{Mi})$:

\begin{equation}
    \pi_{Ri} = \sum_{j=1}^s \pi_{ij} = \frac{\sum_{j=1}^s \Phi_{ij}}{\Phi}, \pi_{Mj} = \sum_{i=1}^S \pi_{ij} = \frac{\sum_{i=1}^S \Phi_{ij}}{\Phi}
\end{equation}

These residential and workplace choice probabilities have an intuitive interpretation. The idiosyncratic shock to preferences $z_{ijo}$ implies that individual workers choose different bilateral commutes when faced with the same prices $\{Q_i, w_j\}$, commuting costs $\{d_{ij}\}$ and location characteristics $\{B_i, T_i, E_j\}$. Other things equal, the more attractive its amenities $B_i$, the higher its average idiosyncratic utility determined by $T_i$, the lower its residential floor prices $Q_i$, and the lower its commuting costs $d_{ij}$ to employment locations.

Conditional on living in block $i$, the probability that a worker commutes to block $j$ is:

\begin{equation}
  \pi_{ij \mid i} = \frac{E_j(w_j / d_{ij})^\varepsilon}{\sum_{s=1}^S E_s(w_s / d_{is})^\varepsilon}
\end{equation}

where the terms in $\{Q_i, T_i, B_i\}$ have cancelled from the numerator and denominator. Therefore, the probability of commuting to block $j$ conditional on living in block $i$ depends on wage $(w_j)$, average utility draw $(E_j)$, and commuting costs $(d_{ij})$ of employment location $j$ in the numerator as well as teh wage $(w_s)$, average utility draw $(E_s)$ and commuting costs $(d_{is})$ for all other possible employment locations $s$ in the denominator.

Using the conditional commuting probabilities, we obtain the following commuting market clearing condition that equates the measure of worker employed in block $j (H_{M_j})$ with the measure of workers choosing to commute to block $j$:

\begin{equation}
  H_{M_j} = \sum_{s=1}^S \frac{E_j(w_j / d_{ij})^{\varepsilon}}{\sum_{s=1}^S E_s(w_s / d_{is})^{\varepsilon}} H_{R_i}
\end{equation}

where $H_{R_i}$ is the measure of residents in block $i$. Since there is a continuous measure of workers residing in each location, there is no uncertainty in the supply of workers to each employment location.

Expected worker income conditional on living in block $i$ is equal to the wages in all possible employment location weighted by the probabilities of commuting to choose those locations conditional on living in $i$:

\begin{equation}
  E[w_j \mid i] = \sum_{j=1}^S \frac{E_j(w_j / d_{ij})^\varepsilon}{\sum_{s=1}^S E_s(w_s / d_{is})^\varepsilon} w_j
\end{equation}

Finally, population mobility implies that the expected utility from moving to the city is equal to the reservation level of utility in the wider economy $(\tilde{U})$:

\begin{equation}
  E[u] = \gamma [\sum_{r=1}^S \sum_{s=1}^S T_r E_s(d_{rs} Q_r^{1-\beta})^{-\varepsilon} (B_r w_s)^{\varepsilon}]^{1/\varepsilon} = \tilde{U}
\end{equation}

where $E$ is the expectation operator and expectation is taken over the distribution for the idiosyncratic component of utility: $\gamma = \Gamma(\frac{\varepsilon - 1}{\varepsilon})$ and $\Gamma(\cdot)$ is the Gamma function.

\subsection{Production}

Production of the tradable final good occurs under conditions of perfectly competition and constant returns to scale. We can assume the production technology takes the CD-form

\begin{equation}
  y_j = A_j H_{M_j}^{\alpha} L_{M_j}^{1 - \alpha}
\end{equation}

where $A_j$ is final goods productivity and $L_{M_j}$ is the measure of floor space used commercially. Firms choose their block of production and their inputs of workers and commercial floor space to maximize profits, taking as given final goods productivity $A_j$, the distribution of idiosyncratic utility, goods and factor prices, and the location decisions of other firms and workers. Profit maximization implies that equlibrium employment in block $j$ is increasing in productivity ($A_j$), decreasing in wage ($w_j$), and increasing in commercial floor space ($L_{M_j}$).

\begin{equation}
  H_{M_j} = (\frac{\alpha A_j}{w_j})^{1/(1-\alpha)} L_{M_j}
\end{equation}

where the equilibrium wage is determined by the requirement that the demand for workers in each employment location equals the supply of workers to that location.

From the first order conditions for maximization and zero profit, equilibrium commercial floor prices ($q_j$) in each block with positive employment must satisfy:

\begin{equation}
  q_j = (1 - \alpha) (\frac{\alpha}{w_j})^{\alpha / (1 - \alpha)} A_j^{1/(1 - \alpha)} L_{M_j}^{-1/(1 - \alpha)}
\end{equation}

\subsection{Land Market Cleaning}

Land market equilibrium requires no-arbitrage conditions between the commercial and residential use of floor space after the tax equivalence of land use regulations. The share of floor space used commercially $(\theta_i)$ is:

\begin{equation}
  \begin{aligned}
    \theta_i & = 1 \text{ if } q_i > \xi_i Q_i \\
    \theta_i & \in [0, 1] \text{ if } q_i = \xi_i Q_i
    \theta_i & = 0 \text{ if } q_i < \xi_i Q_i
  \end{aligned}
\end{equation}

where $\xi_i \geq 1$ captures one plus the tax equivalent of land use regulations that restrict commercial land use relative to residential land use. We assume that the observed price of floor spaces in the data is the maximum of the commercial and residential price of floor space: $\mathbb{Q} = \max\{q_i, Q_i\}$. Hence the relationship between observed, commercial, and residential floor prices can be summarized as:

\begin{equation}
  \begin{aligned}
  \mathbb{Q}_i & = q_i, q_i > \xi_i Q_i, \theta_i = 1 \\
  \mathbb{Q}_i & = q_i, q_i = \xi_i Q_i, \theta_i \in [0, 1] \\
  \mathbb{Q}_i = Q_i, q_i < \xi_i Q_i, \theta_i = 0
\end{aligned}
\end{equation}

We follow the standard approach in the urban literature of assuming that floor space $L$ is supplied by a competitive construction sector that uses land $K$ and capital $M$ a inputs. We assume that the production function takes the CD-form: $L_i = M_i^{\mu} K_i^{1 - \mu}$. Therefore, the corresponding dual cost function for floor space is $\mathbb{Q}_i = \mu^{-\mu}(1 - \mu)^{-(1 - \mu)} \mathbb{P}^{\mu} \mathbb{R}^{1 - \mu}_i$, where $\mathbb{Q}_i = \max\{q_i, Q_i\}$ is the price for floor space, $\mathbb{P}$ is the common price for capital across all blocks, and $\mathbb{R}_i$ is the price for land. Since the price for capital is the same across all locations, the relationship between the quantities and prices of floor space and land can be summarized as:

\begin{equation}
  L_i = \phi_i K_i^{1 - \mu}
\end{equation}

\begin{equation}
  \mathbb{Q}_i = \chi \mathbb{R}_i^{1 - \mu}
\end{equation}

where we refer to $\phi_i = M_i^{\mu}$ as the density of development and $\chi$ is a constant. Residential land market clearing implies that the demand for residential floor space equals the supply of floor space allocated to residential use in each location: $(1 - \theta_i)L_i$. Maximization for each worker and taking expectation over the distribution for idiosyncratic utility, the residential land market clearing condition can be expressed as:

\begin{equation}
  \mathbb{E}[\ell_i] H_{R_i} = (1 - \beta) \frac{\mathbb{E}[w_s \mid i] H_{R_i}}{Q_i} = (1 - \theta_i)L_i.
\end{equation}

Commercial land market cleaning requires that the demand for commercial floor space equals the supply of floor space allocated to commercial use in each location: $\theta_j L_j$. Using the FOC for profit maximization, the commercial land market clearing condition can be expressed as:

\begin{equation}
  (\frac{(1 - \alpha) A_j}{q_j})^{1/\alpha} H_{M_j} = \theta_j L_j
\end{equation}

We both residential and commercial land market clearing are satisfied, total demand for floor space equals the total supply of floor space:

\begin{equation}
  (1 - \theta_i)L_i + \theta_i L_i = L_i = \phi_i K_i^{1 - \mu}
\end{equation}

\subsection{General Equilibrium With Exogenous Location characteristics}

Given the model's parameters $\{\alpha, \beta, \mu, \varepsilon, \kappa\}$, the reservation level of utility in the wider economy $\overline{U}$, and vectors of exogenous location characteristics $\{T, E, A, B, \phi, K, \xi, \tau\}$, the general equilibrium is referred with six vectors $\{\pi_M, \pi_R, Q, q, w, \theta\}$ and total city population $H$.

\begin{proposition}
  Assuming exogenous finite, and strictly location characteristics $(T_i \in (0, \infty), E_i \in (0, \infty), \phi_i \in (0, \infty), K_i \in (0, \infty), \xi_i \in (0, \infty), \tau_{ij} \in (0, \infty) \times (0, \infty))$, and exogenous, finite and nonnegative final goods productivity $A_i \in [0, \infty)$ and residential amnetities $B_i \in [0, \infty)$, there exists a unique general equilibrium vector $\{\pi_M, \pi_R, Q, q, w, \theta\}$
\end{proposition}

\begin{proof}
  The proof can be decomposed into two parts: firstly, we show that under the assumptions that all blocks have strictly positive, finite, and exogenous location and we allow some blocks to be more attractive than others in terms of these characteristics. But workers draw idiosyncratic preferences from a Frechet distribution for pairs of residence and workplace locations, and therefore, since teh Frechet distribution is unbounded from above, any block with strictly positive characteristics has a positive measure of workers that prefer that location as a residence or workplace at a positive and finite price. Hence, all blocks with finite positive wages attract a positive measure of workers, and all blocks with finite positive floor prices attract a positive measure of residents.

  Then We next show that blocks with strictly positive, finite, and exogenous location characteristics must have strictly positive and finite values of both wages and floor prices in equilibrium.
\end{proof}

\subsection{Introducing Agglomenration Forces}

We now introduce endogenous agglomenration forces. We allow final goods productivity to depend on production fundamentals $(a_j)$ and production externalities $(Y_j)$. Production fundamentals capture features of physical geography that make a location more or less productive independently of the surrounding density of economic activity. Production externalities impose structure on how the productivity of a given block is affected by the characteristics of other blocks. Specifically, we follow the standard approach in urban economics of modeling these externalities as depending on the travel-time weighted sum of workplace employment density in surrounding blocks:

\begin{equation}
  A_j = a_j Y_j^{\lambda}, Y_j \equiv \sum_{s=1}^S e^{-\delta \tau_{js}} (\frac{H_{M_s}}{K_s}),
\end{equation}

where $H_{Ms}/K_s$ is workplace employment density per unit of land area; production externalities decline with travel time $(\tau_{js})$ through the iceberg factor $e^{-\delta \tau_{js})} \in (0, 1]$; $\delta$ determines their rate of spatial decay, and $\lambda$ controls their relative importance in determining overall productivity.

We model the externalities in workers' residential choices analygously to the externalities in firms' production choices. We allow residential amenities to depend on residential fundamentals $(b_i)$ and residential externalities $(\Omega_i)$. Residential fundamentals capture features of physical geography that make a location a more or less attractive place to live independently of the surrounding density of economic activity. Residential externalities again impose structure on how the amenities in a given block are affected by the characteristics of other blocks. Specifically, we adopt a symmetric specification as for production externalities, and model residential externalities as depending on the travel time weighted sum of residential employment density in surrounding blocks:

\begin{equation}
  B_i = b_i \Omega_i^{\eta}, \Omega_i \equiv \sum_{r=1}^S e^{-\rho \tau_{ir}} (\frac{H_{Rr}}{K_r})
\end{equation}

where $H_{Rr} / K_r$ is residence employment density per unit of land area; residential externalities decline with travel time $(\tau_{ir})$ through the iceberg factor $e^{-\rho \tau_{ir}} \in (0, 1]$; $\rho$ determines their rate of spatial decay; and $\eta$ controls their relative importance in overall residential amenities. The parameter $\eta$ captures the net effect of residence employment density on amenities, including negative spillover such as air pollution and crime, and positive externalities through the availability of urban amenities. Although $\eta$ captures the direct effect of higher residence employment density on utility through amenities, there are clearly other general equilibrium effects through floor prices, commuting times and wages.

\subsection{Recovering Location Characteristics}

We now show that there is a unique mapping from the observed variables to unobserved location characteristics. Since a number of these unobserved variables enter the model isomorphically, we define the following composites denoted bya tilde:

\begin{equation*}
  \begin{aligned}
    \tilde{A}_i & = A_i E_i^{\alpha / \varepsilon}, \tilde{a}_i = a_i E_i^{\alpha / \varepsilon} \\
    \tilde{B}_i & = B_i T_i^{1/\varepsilon} \zeta_{Ri}^{1 - \beta}, \tilde{b}_i = b_i T_i^{1 / \varepsilon} \zeta_{Ri}^{1 - \beta} \\
    \tilde{w}_i & = w_i E_i^{1 / \varepsilon} \\
    \tilde{\phi}_i & = \tilde{\phi}_i(\phi_i, E_i^{1 / \varepsilon}, \xi_i)
  \end{aligned}
\end{equation*}

where we use $i$ to index all blocks, and the function $\tilde{\phi}_i(\cdot)$ is a defined function; $\zeta_{Ri} = 1$ for completely specialized residential blocks; and $\zeta_{Ri} = \zeta_i$ for residential blocks with some commercial land use.

In the labor market, the adjusted wage for each employment location $(\tilde{w}_i)$ captures the wage $(w_i)$ and the Frechet scale parameter for the location $(E_i^{1 / \varepsilon})$, because these both affect the relative attractiveness of an employment location to workers. On the production side, adjusted productivity for each employment location $(\tilde{A}_i)$ captures productivity $(A_i)$ and the Frechet scale parameter for the location $(E_i^{\alpha / \varepsilon})$ because these both affect the adjusted wage consistent with zero profits. Adjusted production fundamentals are defined analogously. On the consumption side, adjusted amenities for each residence location $(\tilde{B}_i)$ capture amenities $(B_i)$, the Frechet scale parameter for that location $(T_i^{1 / ]\varepsilon})$ and the relationship between observed and residential floor prices $(\zeta_{Ri} \in \{1, \zeta_i\})$, because these all affect the relative attractiveness of a location consistent with population mobility. Adjusted residential fundamentals are defined analogously. Finally, in the land market, the adjusted density of development $(\tilde{\phi}_i)$ includes the density of development $(\phi_i)$ and other production and residential parameters that affect land market clearing.

\begin{proposition}
  \begin{enumerate}
    \item Given known values for the parameters $\{\alpha, \beta, \mu, \varepsilon, \kappa\}$ and the observed data $\{\mathbb{Q}, \bf{H}_M, \bf{H}_R, \bf{K}, \bf{\tau}\}$ there exists unique vectors of the unobserved location characteristics $\{\tilde{\bf{A}}^*, \tilde{\bf{B}}^*, \tilde{\bf{\phi}}^*\}$ that are consistent with the data being an equilibrium of the model.
    \item Given known values for the parameters $\{\alpha, \beta, \mu, \varepsilon, \kappa, \lambda, \delta, \eta, \rho\}$ and the observed data $\{\mathbb{Q}, \bf{H}_M, \bf{H}_R, \bf{K}, \bf{\tau}\}$ there exists a unique vectors of the unobserved location characteristics $\{\tilde{\bf{a}}^*, \tilde{\bf{b}}^*, \tilde{\bf{\phi}}^*\}$ that are consistent with the data being an equilibrium of the model.
  \end{enumerate}
\end{proposition}


















