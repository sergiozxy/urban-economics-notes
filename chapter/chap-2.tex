\chapter{Increasing Returns and Economic Geography}

\section{A TWo-Region Model}

All individuals in this economy, then, are assumed to share a utility function of the form:

\begin{equation}
    U = C_M^{\mu} C_A^{1 - \mu}
\end{equation}

where $C_A$ is consumption of the agricultural good and $C_M$ is consumption of a manufactures aggregate. The manufactures aggregate $C_M$ is defined by:

\begin{equation}
    C_M = \left[ \sum_{i=1}^N c_i^{(\sigma - 1)/\sigma}\right]^{\sigma/(\sigma - 1)}
\end{equation}

where $N$ is the large number of potential products and $\sigma > 1$ is the elasticity of substitution among the products. The elasticity $\sigma$ is the second parameter determining the character of equilibrium in the model.

There are two regions in the economy and two factors of production in each region. Peasants produce agricultural goods; without loss of generality, we suppose that the unit labor requirement is one. The peasant population is assumed completely immobile between regions, with a given peasant supply $(1 - \mu) / 2$ in each region. Workers may move between the regions; we let $L_1$ and $L_2$ be the workers supply in regions $1$ and $2$, respectively, and require only that the total add up to the overall number of workers $\mu$:

\begin{equation}
    L_1 + L_2 = \mu
\end{equation}

The production of an individual manufactured good $i$ involves a fixed cost and a constant marginal cost, giving rise to economies of scale.

\begin{equation}
    L_{M_i} = \alpha + \beta x_i
\end{equation}

where $L_{M_i}$ is the labor used in producing $i$ and $x_i$ is the good's output.

Two strong assumptions will be made for tractability. First, transportation of agricultural output will be assumed to be costless.