\chapter{Increasing Returns and Economic Geography}

\section{A TWo-Region Model}

All individuals in this economy, then, are assumed to share a utility function of the form:

\begin{equation}
    U = C_M^{\mu} C_A^{1 - \mu}
\end{equation}

where $C_A$ is consumption of the agricultural good and $C_M$ is consumption of a manufactures aggregate. The manufactures aggregate $C_M$ is defined by:

\begin{equation}
    C_M = \left[ \sum_{i=1}^N c_i^{(\sigma - 1)/\sigma}\right]^{\sigma/(\sigma - 1)}
\end{equation}

where $N$ is the large number of potential products and $\sigma > 1$ is the elasticity of substitution among the products. The elasticity $\sigma$ is the second parameter determining the character of equilibrium in the model.

There are two regions in the economy and two factors of production in each region. Peasants produce agricultural goods; without loss of generality, we suppose that the unit labor requirement is one. The peasant population is assumed completely immobile between regions, with a given peasant supply $(1 - \mu) / 2$ in each region. Workers may move between the regions; we let $L_1$ and $L_2$ be the workers supply in regions $1$ and $2$, respectively, and require only that the total add up to the overall number of workers $\mu$:

\begin{equation}
    L_1 + L_2 = \mu
\end{equation}

The production of an individual manufactured good $i$ involves a fixed cost and a constant marginal cost, giving rise to economies of scale.

\begin{equation}
    L_{M_i} = \alpha + \beta x_i
\end{equation}

where $L_{M_i}$ is the labor used in producing $i$ and $x_i$ is the good's output.

Two strong assumptions will be made for tractability. First, transportation of agricultural output will be assumed to be costless. Second, transportation costs for manufactured goods will be ``iceberg'' form. Suppose that tehre are a large number of manufacturing firms, each producing a single product. Then given the definition of the manufacturing aggregate and the assumption of iceberg transport costs, the elasticity of demand facing any individual firm is $\sigma$. The profit-maximizing pricing behavior of a representative firm in region $1$ is therefore to set a price equal to:

\begin{equation}
  p_1 = (\frac{\sigma}{\sigma - 1}) \beta w_1
\end{equation}

where $w_1$ is the wage rate of workers in region $1$; a similar equation applies in region $2$.

\begin{equation}
  \frac{p_1}{p_2} = \left(\frac{w_1}{w_2}\right)
\end{equation}

If there is free entry of firms into manufacturing, profits must be driven to zero. Then it must be true that:

\begin{equation}
  (p_1 - \beta w_1) x_1 = \alpha w_1,
\end{equation}

which implies:

\begin{equation}
  x_1 = x_2 = \frac{\alpha(\sigma - 1)}{\beta}.
\end{equation}

This has the useful implications that the number of manufactured goods produced in each region is proportional to the number of workers, so that:

\begin{equation}
  \frac{n_1}{n_2} = \frac{L_1}{L_2}
\end{equation}

It should be noted that in zero-profit equilibrium, $\sigma / (\sigma - 1)$ is the ratio of the marginal product of labor to its average product, that is, the degree of economics of scale. Thus although $\sigma$ is a parameter of tastes rather than technology, it can be interpreted as an inverse index of equilibrium economies of scale.

\section{Short-Run and Long-Run Equilibrium}

We suppose that workers move toward the region that offers them higher real wages, leading to either convergence between regions as they move toward equality of worker/peasant ratios or divergence as the workers all congregate in one region.

Let $c_{11}$ be the consumption in region $1$ of a representative region $1$ product, and $c_{12}$ be the consumption in region $1$ of a representative region $2$ product. The price of a local product is simply its free on board price $p_1$; the price of a product from the other region, however, is its transport cost-inclusive price $p_2 / \tau$. Thus the relative demand for representative product is:

\begin{equation}
  \frac{c_{11}}{c_{12}} = (\frac{p_1 \tau}{p_2})^{-\sigma} = (\frac{w_1 \tau}{w_2})^{-\sigma}
\end{equation}

Define $z_{11}$ as the ratio of region $1$ expenditure on local manufactures to that on manufactures from the other region. Two points should be noted about $z$. First, a 1 percent rise in the relative price of region $1$ goods, while reducing the relative quantity sold by $\sigma$ percent, will reduce the value by only $\sigma - 1$ percent because of the valuation effect. Second, the more goods produced in region $1$, the higher their share of expenditure for any given relative price. Thus,

\begin{equation}
  z_{11} = (\frac{n_1}{n_2}) (\frac{p_1 \tau}{p_2}) (\frac{c_{11}}{c_{12}}) = (\frac{L_1}{L_2}) (\frac{w_1 \tau}{w_2})^{-(\sigma - 1)}
\end{equation}

Similarly, the ratio of region $2$ spending on region $1$ products to spending on local products is:

\begin{equation}
  z_{12} = (\frac{L_1}{L_2}) (\frac{w_1}{w_2 \tau})^{-(\sigma - 1)}
\end{equation}

The total income of region $1$ workers is equal to the total spending on these products in both regions. Let $Y_1$ and $Y_2$ be the regional incomes then the income of region $1$ workers is:

\begin{equation}
  w_1 L_1 = \mu \left[ (\frac{z_{11}}{1 + z_{11}})Y_1 + (\frac{z_{12}}{1 + z_{12}})Y_2 \right]
\end{equation}

and the income of region $2$ workers is:

\begin{equation}
  w_2 L_2 = \mu \left[ (\frac{1}{1 + z_{11}})Y_1 + (\frac{1}{1 + z_{12}})Y_2 \right]
\end{equation}

The incomes of the two regions, however, depend on the distribution of workers and their wages. Recalling that the wage rate of peasants in the numeraire, we have:

\begin{equation}
  Y_1 = \frac{1 - \mu}{2} + w_1 L_1 \quad Y_2 = \frac{1 - \mu}{2} + w_2 L_2
\end{equation}

The above equations may be regarded as a system that determines $w_1$ and $w_2$. By inspection, one can see that if $L_1 = L_2, w_1 = w_2$. If labor is then shifted to region $1$, however, the relative wage rate $w_1 / w_2$ can move either way. The reason is that there are two opposing effects. One side, there is the ``home market effect'': other things equal, the wage rate will tend to be higher in the larger market. On the other side, there is the extent of competition: workers in the region with the smaller manufacutring labor force will face less competition for the local peasant market than those in the more populous region. 

In the long run, workers are interested not in nominal wages but in real wages, and workers in the region with the larger population will face a lower price for manufacturing goods. Let $f = L_1 / \mu$, the share of the manufacutring labor force in region $1$. Then the true price index of manufacutring goods for consumers residing in region $1$ is:

\begin{equation}
  P_1 = \left[f w_1^{-\sigma + 1} + (1 - f)(\frac{w_2}{\tau})^{1 - \sigma}]^{1/(1 - \sigma)}
  \end{equation}

that for consumers residing in region $2$ is:

\begin{equation}
  P_2 = \left[f (\frac{w_1}{\tau})^{-\sigma + 1} + (1 - f)(w_2^{1 - \sigma}]^{1/(1 - \sigma)}
\end{equation}

The real wage of workers in each region are:

\begin{equation}
  \omega_1 = w_1 P_1^{-\mu} \quad \omega_2 = w_2 P_2^{-\mu}.
\end{equation}

\section{Necessary Conditions for Manufacutring Concentration}

